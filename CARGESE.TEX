\input harvmac
 
\hsize37truepc\vsize58truepc
\hoffset=.2truein\voffset=-.4truein
\nopagenumbers
 
 
\vphantom{0}\vskip1.5truein
%
\leftline{MATRIX MODELS OF 2D GRAVITY AND}
\leftline{ISOMONODROMIC DEFORMATION}
%
\vskip.65truein
\hbox{\obeylines\baselineskip12pt\parskip0pt\parindent0pt\hskip1.1truein
\vbox{Gregory Moore
Department of Physics and Astronomy
Rutgers University
Piscataway, NJ 08855-0849
and 
Department of Physics
Yale University
New Haven, CT 06511-8167}}
%
\vskip .3truein
\baselineskip 14pt plus 1pt minus 1pt
 

\def\CP {{\cal P }}
\def\CL {{\cal L}}
\def\CV {{\cal V}}
\def\p {\partial}
\def\CS {{\cal S}}
\def\hb {\hbar}
\def\inbar{\,\vrule height1.5ex width.4pt depth0pt}
\def\IB{\relax{\rm I\kern-.18em B}}
\def\IC{\relax\hbox{$\inbar\kern-.3em{\rm C}$}}
\def\IP{\relax{\rm I\kern-.18em P}}
\def\IR{\relax{\rm I\kern-.18em R}}
\medskip

\newsec{Introduction} 

One of the principal goals of the theory of 
2D gravity is making sense of the formal expression
\eqn\formal{
Z(\mu,\kappa;t_i)=\sum_h\int_{MET_h}dg e^{\mu\int\sqrt{g}+\kappa\int R}
Z_{QFT(t_i)}[g] }
where we integrate over metrics $g$ on surfaces with $h$ handles 
with a weight defined by the Einstein-Hilbert action 
($\mu$ is the cosmological constant and $\kappa$ is Newton's 
constant, or, equivalently, the string coupling) together 
with the partition function of some 2D quantum field theory,
$QFT(t_i)$. The parameters $t_i$ are
coordinates on a subspace of the space of 2D field theories, 
or, equivalently, coordinates for a space of string backgrounds.

The expression \formal\ is of course extremely formal, especially when 
one includes the sum over topologies.
Nevertheless rigorous definitions of 
this expression have been proposed using the methods of 
random matrix theory. Last year these concrete definitions 
led to a beautiful result for \formal\ for a special set 
of quantum field theories
\nref\BK{E. Br\'ezin and V. Kazakov, ``Exactly solvable field
theories of closed strings,'' Phys. Lett. {\bf B236}(1990)144.}%
\nref\DS{M. Douglas and S. Shenker, ``Strings in less than one
dimension,'' Rutgers preprint RU-89-34.}%
\nref\GM{D. Gross and A. Migdal, ``Nonperturbative two dimensional
quantum gravity,'' Phys. Rev. Lett. {\bf 64}(1990)127.}%
\nref\GMi{D. Gross and A. Migdal, ``A nonperturbative treatment of
two-dimensional quantum gravity,'' Princeton preprint PUPT-1159(1989).}
\nref\newD{M. Douglas, ``Strings in less than one dimension
and the generalized KdV hierarchies,''  Rutgers preprint RU-89-51.}
\nref\bdss{T. Banks, M. Douglas, N. Seiberg, and S. Shenker,
``Microscopic and macroscopic loops in non-perturbative two dimensional
gravity,'' Rutgers preprint RU-89-50.}\refs{\BK{--}\bdss}. 
The answer is most simply expressed as 
a differential equation satisfied by 
\eqn\spfcht{u(x;t_i)={\p^2\over\p x^2} Z }
where $x$ is the scaled string coupling.
The differential equation 
satisfied by $u$ for
coupling to the $(p,q)$ 
minimal conformal field theory
\ref\bpz{A.A. Belavin, A.M. Polyakov, A.B. Zamolodchikov, 
Nucl. Phys. {\bf B241}(1984)333}\ 
was most elegantly 
formulated by M. Douglas \newD\ in terms of a differential 
operator $L=D^q+u_{q-2}(x)D^{q-2}+\cdots u_0(x)$, where 
$D=d/dx$ and $u_{q-2}$ is identified with 
$u(x)$, as the system of equations:
\eqn\dgle{[L^{p/q}_+,L]=1 }
where the subscript $+$ keeps the differential operator part of 
a pseudodifferential operator. The equations for massive 
models interpolating between the minimal models have the form
$\sum_p t_p [L^{p/q}_+,L]=1$. 

The specific case of the $(2p-1,2)$ models (corresponding 
to single hermitian matrix models) have been most intensively
studied. In this case we have simply $L=D^2+u(x)$ 
so the string equations are
\eqn\onemat{\sum_j (j+\half)t_j R_j[u(x)]=x}
where $R_j$ are conserved densitites of the KdV hierarchy
\GM\GMi\ .

Solutions to \onemat\ satisfy several interesting properties.
One remarkable 
result is that a solution to \onemat\ , as a function of the 
$t_j$ should satisfy KdV flow in the $t_j$ \bdss\ .
The original argument of Banks, Douglas, Seiberg, and 
Shenker was proposed from the 
physical point of view using the matrix model integral, 
and makes certain implicit assumptions about 
boundary conditions. Later the issue of proper
boundary conditions for physically acceptable solutions 
of \onemat\ was clarified
\ref\BMP{E. Brezin, E. Marinari, and G. Parisi, ``A Non-Perturbative 
Ambiguity Free Solution of a String Model,'' ROM2F-90-09}.
BMP argued that
the existence of physically reasonable solutions 
depends on the parity of $m$, the largest index for 
which $t_m\not=0$. 
Physical solutions should exist only for $m$ odd, 
and in this case the asymptotics should fix a unique solution of 
\onemat\ which is pole-free on the real axis. 
These arguments were supported by a numerical solution 
for the case $m=3$. 
In 
\ref\dss{M. Douglas, N. Seiberg, and S. Shenker, 
``Flow and instability in quantum gravity,'' Rutgers preprint, 
RU-90-19}\ 
it was demonstrated, again numerically, 
that one cannot use KdV flow to define ``pure gravity'' 
(an $m=2$ solution) by flowing from the well-defined 
$m=3$ solution. Essentially, the solution develops a 
shock wave and there is no well-defined
$t_2\to \infty$ limit.

This paper is a review and continuation of 
\ref\geom{G. Moore, ``Geometry of the string equations,'' 
Yale preprint YCTP-P4-90}\ where  
the formalism of isomonodromic deformation
\nref\Its{A. Its and V. Yu. Novokshenov, {\it The Isomonodromic
Deformation Method in the Theory of Painlev\'e Equations,}
Springer Lect. Notes Math. 1191.}\ and some related ideas 
were applied to the string equations. 
The purpose of our paper was threefold.
First, as described in section three,
the isomonodromic deformation formalism is well
suited to proving rigorously the statements regarding 
the properties of KdV flow and existence and uniqueness of
solutions mentioned above. 
The formalism applies equally well to the 
string equations associated with unitary-matrix models and 
with double-cut phases of the hermitian matrix models.
Second, through the work of the 
Kyoto school
\nref\Jimboi{M. Jimbo, T. Miwa, K. Ueno, ``Monodromy Preserving 
Deformation of Linear Ordinary Differential Equations with Rational
Coefficients,'' Physica {\bf 2D}(1981)306.}
\nref\Jimboii{M. Jimbo and T. Miwa, ``Monodromy Preserving 
Deformation of Linear Ordinary Differential Equations with Rational
Coefficients. II,'' Physica {\bf 2D}(1981)407.}
\nref\Jimboiii{M. Sato, T. Miwa, and M. Jimbo, ``Aspects of Holonomic
Quantum Fields Isomonodromic Deformation and Ising Model,''
in {\it Complex Analysis, Microlocal Calculus and Relativisitic
Quantum Theory}, D. Iagolnitzer, ed., Lecture Notes in Physics 126}
\nref\Jimboiv{M. Jimbo, ``Introduction to Holonomic Quantum 
Fields for Mathematicians,'' Proc. Symp. in Pure Math. {\bf 49}(1989)part
I. 379.}
\refs{\Jimboi{--}\Jimboiv}\ it is known that the isomonodromic
deformation formalism is closely related to the quantum field 
theory of free fermions in two-dimensional spacetime. This is 
extremely suggestive since hermitian matrix models can also
be written in terms of nonrelativistic 
fermions. It is worth having a good
understanding of any connection between these fermions since 
it might be indicative of a deep connection between nonperturbative
2D gravity and ``quantum field theory on the spectral curve.''
These matters are discussed in sections two and four.
Third, and more philosophically, interesting physics is usually
related to interesting geometry, and this certainly ought to 
be the case for nonperturbative quantum gravity. We would like 
to know the underlying geometrical significance of the string 
equations. By this we do {\it not} mean the geometrical 
meaning of the terms in the asymptotic expansion of solutions
to the string equations, for these have already been 
adequately understood from the point of view of topological
field theory
\nref\witt{E. Witten, ``On the structure of the topological phase of 
two dimensional gravity,'' preprint IASSNS-HEP-89/66}
\nref\distler{J. Distler, ``2D quantum gravity, topological
field theory and multicritical matrix models,'' princeton 
preprint PUPT-1161}
\nref\DiWit{R. Dijkgraaf and E. Witten, ``Mean Field Theory, Topological 
Field Theory, and Multi-Matrix Models,'' IASSNS-HEP-90/18;PUPT-1166}
\nref\newvsq{E. Verlinde and H. Verlinde, ``A Solution of two 
dimensional topological quantum gravity,'' preprint IASSNS-HEP-90/40}
\nref\dvv{R. Dijkgraaf, H. Verlinde, and E. Verlinde, Princton preprint
PUPT-1184}\refs{\witt{--}\dvv} .
This third goal, which was our primary motivation in \geom\ , remains
distant.

Closely related matters have been discussed in many recent papers.
Of these we draw particular attention to 
\nref\emil{E. Martinec, ``On the origin of integrability in matrix 
models,'' Chicago preprint, EFI-90-67}
\nref\witrev{E. Witten, ``Two dimensional gravity and
intersection theory on moduli space,'' IAS preprint,
IASSNS-HEP-90/45}\ 
\nref\morozovi{A. Gerasimov, A. Marshakov, A. Mironov, 
A. Morozov, and A. Orlov, ``Matrix Models of 2D gravity and 
Toda Theory,'' P.N. Lebedev Institute preprint, July 1990}
\nref\morozovii{A. Mironov and A. Morozov, ``On the origin of 
virasoro constraints in matrix models: lagrangian approach,''
P.N. Lebedev Institute preprint, July 1990}\refs{\emil{--}\morozovii}\ 
where, among other things,
the matrix models were shown to be equivalent to the 
infinite Toda chain - an important integrable system- 
even {\it before} taking the continuum limit.

\newsec{Matrix Models and 2D Field Theory}

In this section we outline how a ``2D field theory on 
the spectral curve'' may be seen to emerge from the 
matrix model integral. Our understanding of this phenomenon
is woefully incomplete, but some
relevant features may be seen already at this stage. 
Most importantly, 
the spectral parameter of inverse scattering 
theory is identified with the eigenvalue coordinate
of the random matrix path integral. This point of
view has also been emphasized in \morozovi\morozovii\ 
(although some details are different).

\subsec{The level-spacing problem}

The clearest example of the phenomenon we are discussing 
can be seen already in the large $N$ limit of hermitian 
matrix models. Consider the matrix model:
\eqn\gauss{\eqalign{
Z_N&=\int d^{N^2}\phi e^{- N  tr V(\phi)}\cr
&=\int \prod d\lambda_i \Delta^2 e^{-N \sum_i V(\lambda_i)}\cr} }
where $V(\lambda)$ is a polynomial in $\lambda$. 
and consider furthermore the probability $\tau(I;N)$
that no eigenvalue falls in the range $I=[\lambda_1,\lambda_2]$.
It was shown in 
\ref\dnsmt{M. Jimbo, T. Miwa, Y. Mori and M. Sato, 
``Density Matrix of an Impenetrable Bose Gas and the 
Fifth Painlev\'e Transcendent'' Physica {\bf 1D} (1980)80}
that $\tau(a_1-a_2)=\lim_{N\to \infty}
\tau([a_1/N,a_2/N];N)$ is 
the tau function for the isomonodromy problem related to 
the painlev\'e V equation. Since the context in which 
this was originally understood is (superficially) removed
from matrix models we will show how it follows
from that point of view. 

Correlation functions with the measure
\gauss\  can be interpreted 
\ref\mehta{M. L. Mehta, {\it Random Matrices} Academic Press,1967. }
\ref\itzdr{See sec. 10.3 in C. Itzykson and J.-M. Drouffe,
{\it Statistical Field Theory}, vol. 2, Cambridge Univ. Press. 1989}
\bdss\ as expectation values in a slater determinant of 
fermion one-body wavefunctions given by orthonormal functions:
%
%$$\psi_j(\lambda)= \bigl({N\over 2\pi}\bigr)^{1/4}
%\bigl({N\over j!}\bigr)^{1/2}(\lambda^j+\cdots)e^{-N\lambda^2/ 4} $$
%
$$p_n(\lambda)= P_n(\lambda)e^{-{N\over 2}V(\lambda)} $$
where $P_n$ are orthonormal polynomials for the measure 
$d\lambda e^{-NV(\lambda)}$.
As in \bdss\ we may pass to second quantized wavefunctions:
\eqn\sec{
\eqalign{\psi(\lambda)&=\sum_{n=1}^\infty p_n(\lambda)a_n\cr
\psi(\lambda)^\dagger&=\sum_{n=1}^\infty p_n(\lambda)a_n^\dagger\cr
\{\psi^\dagger(\lambda),\psi(\lambda')\}&=\delta(\lambda-\lambda')\cr} 
}
where the ground state is the fermi sea with the first 
$N$ levels filled.
As argued in \bdss\ the main contributions to correlation functions 
come from the neighborhood of the fermi level. 
For even potentials
we can rewrite the recursion relation for orthonormal
wavefunctions \BK\DS\GM\ in the form:
\eqn\newrec{
\eqalign{
\lambda p_{2n}(\lambda)&=\sqrt{r_{2n+1}}p_{2n+1} +\sqrt{r_{2n}}p_{2n-1}\cr
\lambda p_{2n+1}(\lambda)&=\sqrt{r_{2n+2}}p_{2n+2} 
+\sqrt{r_{2n+1}}p_{2n}
\cr}
}
By evaluating these at $\lambda=0$ we see that 
if we expect a continuum limit for the orthonormal wavefunctions
themselves in the neighborhood of $\lambda=0$ we should define 
\eqn\sclwv{
p_{2n+1}({\lambda\over N})=(-1)^nf_1(x,\lambda)
\qquad
p_{2n}({\lambda\over N})=(-1)^nf_2(x,\lambda)
}
where $x=n/N$. Assuming $r_n$ has an expansion of the 
form $r_n=r(x)+\epsilon^2r_1(x)+\cdots$ where $\epsilon=1/N$
we find that \newrec\ implies
\eqn\sclwvi{
\eqalign{
f_1&=(r(x))^{-1/4}sin\biggl[\lambda\int^x {dx'\over \sqrt{r(x')}}\biggr]\cr
f_2&=(r(x))^{-1/4}cos\biggl[\lambda\int^x {dx'\over \sqrt{r(x')}}\biggr]\cr
} }
Since the dominant contributions of physical quantities 
come from the neighborhood of the fermi level ($x=1$) 
the orthonormal wavefunctions become 
sines and cosines. These arguments can be checked explicitely
using hermite functions in the case of a gaussian measure.

>From the behavior of the 1-body wavefunctions it follows that
$\hat \psi(\gamma;N)\equiv {1\over \sqrt{N}}\psi(\gamma/N)$
has a smooth large $N$ limit.
For example the Darboux-Christoffel formula
\eqn\extpt{
\langle N|\psi^\dagger(\lambda_1)\psi(\lambda_2)| N\rangle
=\sqrt{r_{N+1}}{p_{N+1}(\lambda_1)p_N(\lambda_2)-
p_{N+1}(\lambda_2)p_N(\lambda_1)\over
\lambda_1-\lambda_2}
} 
implies that
\eqn\limkernel{
\langle N|\hat\psi^\dagger(\gamma)\hat\psi(\gamma')| N \rangle 
\rightarrow {1\over  \pi} {sin (\gamma-\gamma')\over
\gamma-\gamma'} \qquad .}
On the operator level we define $p=(n-2N)/2N$ and 
\eqn\osc{\eqalign{
a_1(p)&=\sqrt{N}(-1)^{n+N}\bigl(\hat a_{2n}- i\hat a_{2n+1}\bigr)\cr
a_2(p)&=\sqrt{N}(-1)^{n+N}\bigl(\hat a_{2n}+ i\hat a_{2n+1}\bigr)\cr} }
where $\hat a_n\equiv a_{N+n}$. The sum over $n$ becomes an
integral $\int^{\infty}_{-1}dp$. 
Again, assuming that the main contributions
come from the neighborhood of the fermi level we extend this to 
an integral over the entire $p$ axis.
Our main claim is therefore that $\hat \psi$ has a good large $N$
limit and is given by 
\eqn\lmfn{\eqalign{
\hat\psi(\gamma)&=
e^{i\gamma}\int^\infty_{-\infty}
dp~ a_1(p)e^{i\gamma p}
+e^{-i\gamma}\int^\infty_{-\infty}
dp~ a_2(p)e^{-i\gamma p}\cr
&=e^{i\gamma}\psi_1(\gamma)+e^{-i\gamma}\psi_2(-\gamma)\cr}
 }
and that the fermi sea becomes the ground state 
defined by 
$a_i(p)|0\rangle=0$ for $p>0$ and $a_i^\dagger(-p)|0\rangle=0$
for $p>0$. 

Now let us return to the problem of the level spacing.
In terms of the orthonormal wavefunctions $p_j$ 
we have \mehta\itzdr
\eqn\level{\eqalign{
\tau(I;N) &=det\biggl[\delta_{j,k}-\int_{\lambda_1}^{\lambda_2}d\lambda
p_j(\lambda)p_k(\lambda)\biggr]_{0\leq j,k\leq N-1} \cr
&=\langle N|:exp\biggl(-\int_{\lambda_1}^{\lambda_2}
\hat\psi^\dagger(\lambda)
\hat\psi(\lambda) d\lambda \biggr):|N\rangle\cr}\qquad .
}
where the normal ordering puts $\psi$ to the right of 
$\psi^\dagger$. 
In the $N\to\infty$ limit, taking $\lambda_i=a_i/N,$ 
and $I=[a_1,a_2]$ we obtain:
\eqn\taulevel{\tau(I)=\langle 0|:exp\biggl(
-\gamma\int_I\hat\psi^\dagger
\hat\psi\biggr) :|0\rangle}
where $\gamma=1$ and the normal-ordered exponential is 
evaluated by expanding in power series and point-splitting 
all the integrals. 
This expression with $\gamma=2$ is a correlation function in the 
theory of the one-dimensional Bose gas, also known as the 
nonlinear Schr\"odinger theory, in the completely 
impenetrable case, and can be studied by the methods of
\dnsmt
\ref\Korep{A.R. Its, A.G. Izergin, V.E. Korepin, and N.A. Slavnov,
``Differential equations for quantum correlation functions,'' 
preprint}
\ref\Korepi{A.R. Its, A.G. Izergin, and V.E. Korepin, ``Temperature
correlators of the impenetrable bose gas as an integrable system,''
ICTP preprint IC/89/120}. 
(Indeed, many integrable massive field 
theories have correlation functions related to painlev\'e 
equations
\ref\mcoy{E. Barouch, B.M. McCoy, and T.T. Wu, Phys. Rev. Lett. {\bf
31} (1973)1409; C.A. Tracy and B.M. McCoy, ``Neutron scattering and 
the correlation functions of the Ising model near $T_c$,''
Phys. Rev. Lett. {\bf 31}(1973)1500;
T.T. Wu, B.M. McCoy, C.A. Tracy, and E. Barouch, 
Phys. Rev. {\bf B13}(1976)316.}.)
In particular,
$\tau(I)$ is the fredholm determinant 
$det~(1-K)$ where $K$ is the kernel defined by 
\limkernel\ on the interval $I$
and the integral operator 
$1-K$ is in the infinite dimensional group of ``completely integrable
kernels'' described in \Korep\ . 
Following the general procedure described in \dnsmt\Korep\Korepi\ 
we define 
$\chi_1(z)=e^{iz}\psi_1(z)$ and $\chi_2(z)=e^{-iz}
\psi_2(-z)$ so that $\hat\psi=\chi_1+\chi_2$ and $\psi^\dagger=
\chi_1^\dagger + \chi_2^\dagger$, and consider the ``Baker-Akhiezer 
framing'':
\eqn\levii{
\Psi_{\alpha\beta}(\lambda,\lambda')={\langle 0|\chi_\alpha^\dagger(\lambda)
exp(-\int_I\chi^\dagger \chi)\chi_\beta(\lambda')|0\rangle\over
\langle 0|exp(-\int_I\chi^\dagger \chi)|0\rangle} 
\qquad .}
As we will see in sec. 3.1 this matrix satisfies equations
reminiscent of the Knizhnik-Zamolodchikov equation:
\eqn\levlin{\eqalign{{\p\over \p \lambda}\Psi&=M_\lambda \Psi\cr
{\p\over\p a_i}\Psi&=M_i\Psi\qquad, \cr}}
where $M_\lambda,M_i$ are matrices which are rational in 
$\lambda$.
We will see that the study of solutions of \levlin\ gives 
information on the 
determinant $\tau(I)$. 

\subsec{Multi-Cut Solutions}

We now turn to an example in which we take a double
scaling limit. Consider a special class of 
multicritical potentials
\ref\crmr{\v C. Crnkovi\'c and G. Moore,``Multi-Critical Multi-Cut
Matrix Models,'' Yale preprint YCTP-P16-90}
\eqn\mulpt{V'_m(\lambda)=k(m)\lambda^{2m+1}\bigl(1-{1\over \lambda^2}
\bigr)^{1/2}|_+ }
where the subscript $+$ means we keep the polynomial part in an
expansion about infinity and 
\eqn\const{k(m)=2^{2m+1}{(m+1)!(m-1)!\over (2m-1)!} \qquad .}
$V_m(\lambda)$ has the shape of a double well, and the bump
at $\lambda=0$ becomes progressively flatter as $m$ increases.
These potentials are characterized by the property that the eigenvalue 
distribution at tree level consists of two separate 
cuts, which, at the critical point, meet at the origin.
These theories have very interesting phase transitions investigated
in 
\ref\molinari{G.M. Cicuta, L. Molinari, and E. Montaldi, Mod. Phys.
Lett. {\bf A1}(1986)125}
\dss\crmr\ . The same critical behavior 
was discovered in the unitary-matrix ensembles
\ref\peri{V. Periwal and D. Shevitz, ``Unitary-Matrix Models 
as Exactly Solvable String Theories,'' Phys. Rev. Lett. 
{\bf 64}(1990)1326.}
\ref\neub{H.Neuberger, Nucl. Phys. {\bf B340}(1990)703}.


We consider again \newrec\ , now assuming
\molinari
\dss\ 
that $r_{2n}$ and $r_{2n+1}$ have different scaling limits:
\eqn\scli{\eqalign{r_{2n}&=r_c+a^{1/m}f(z) + a^{2/m} g(z)+\cdots\cr
r_{2n+1}&=r_c-a^{1/m}f(z)+a^{2/m}g(z)+\cdots\cr} }
where, in the standard way, $x={2n\over N}=1-a^2 (z-z_0)$ and 
$Na^{2+1/m}=1$. The tree-level string equation 
is easily shown to be \crmr\
\eqn\tree{f^{2m}= {-z\over 2^{2m-1} (m+1)} }

The critical behavior comes from the region $\lambda\cong\CO(a^{1/m})$. 
By studying the recursion 
relation \newrec\ in the neighborhood of $\lambda\cong 0$ we
expect that the orthonormal wavefunctions
\eqn\lmply{\eqalign{
f^+(z,\lambda)&\equiv j(a,\lambda)(-1)^kp_{2k}(a^{1/m}\lambda)\cr
f^-(z,\lambda)&\equiv j(a,\lambda)(-1)^kp_{2k+1}(a^{1/m}\lambda)\cr} }
will have smooth limits, where we have allowed for a possible 
``wavefunction renormalization'' $j$. 
The scaling 
which gives a smooth limit for the two-point function
is $a^{1/2m}\psi(a^{1/m}\lambda)\to \psi(\lambda)$, so from 
\extpt\ 
\eqn\newtwo{\langle z_0|\hat \psi^\dagger(\lambda_1)
\hat\psi(\lambda_2)|z_0\rangle={f^+(z_0,\lambda_1)f^-(z_0,\lambda_2)-
f^+(z_0,\lambda_2)f^-(z_0,\lambda_1)\over \lambda_1-\lambda_2}\qquad .}
This defines the kernel of a completely integrable integral operator
so, following \dnsmt\Korep\Korepi\ we study the linear ODE's 
satisfied by
$\vec\psi=(f^+(z,\lambda),f^-(z,\lambda) )$. Indeed, 
the double-scaling limit of \newrec\ becomes
\eqn\reciii{\lambda \vec \psi=\sqrt{8}\biggl(-i\sigma_2
{d\over dz}-\sigma_1 {f\over 4}\biggr)\vec \psi\qquad .}

Now consider the relation
\eqn\reciv{{d\over d\lambda}\vec p =
N\bigl(V'(\lambda)_+\bigr) \vec p }
where the subscript $+$ indicates that we keep only the 
upper triangular part of the matrix representation of the 
operator $V'(\lambda)$ in the basis of orthonormal wavefunctions.
In the double 
scaling limit, with an appropriate choice of $j$ and a
slight redefinition of $\vec \psi$, the recursion relations 
become
\eqn\laxi{
\CL\vec\psi\equiv
\biggl({d\over dz}+\sigma_3 \lambda+f \sigma_1\biggr)\vec\psi=0}
\eqn\laxii{
\biggl({d\over d\lambda}+M_m(\lambda,f)\biggr)\vec\psi=0
}
where $M_m$ is polynomial in $\lambda$ and differential polynomial
in $f$. We may find $M_m$ explicitly as follows.
Note that the commutator of $M_m$ with $\CL$ is just $\sigma_3$.
In integrable systems theory the method of 
Zakharov and Shabat  determines
the space of matrices $M$ which are polynomial in $\lambda$ 
and differential polynomial in $f$ and whose commutator with 
$\CL$ is $\lambda$-independent. The vector
space of such matrices is spanned by the matrices
occuring in the Lax pairs for the modified KdV hierarchy. 
(See, e.g., 
\ref\DrS{Drinfeld and Sokolov, ``Equations of Korteweg-de Vries 
type and simple Lie algebras,''
Sov. Jour. Math. (1985)1975, section 3.8}\ .) 

The flatness condition following from \laxi\ and\laxii\ gives an 
ordinary differential equation for $f$. Comparing with 
the result \tree\ derived at tree level we find 
\eqn\res{M_m=(2m+1) \biggl[\bigl
(C_m+(V_m'/2\zeta)+{z\over 2m+1}\bigr)\sigma_3
+V_m\sigma_1 -(V_m'/2\zeta)i\sigma_2\biggr]}
where
\eqn\uets{\eqalign{
C_m&\equiv R_m + \zeta^2 R_{m-1}+\cdots \zeta^{2m}R_0 \cr
V_m&\equiv \zeta S_{m-1} +\zeta^3 S_{m-2}+\cdots \zeta^{2m-1} S_0\cr
S_m&\equiv f R_m-\half R_m'\cr} 
}
and the KdV potentials are evaluated for $u=f^2 + f'$. The compatibility 
conditions become the (2-cut) string equation
$S_m-{z\over 2m+1} f=0$
\peri\neub\ .
(Of course, the above 
argument is a variant of Douglas' original argument.)

More generally, by adding the multicritical potentials 
$V=\sum_\ell t_\ell a^{(2m-2\ell)/m} V_\ell$ 
we simply replace
\eqn\repl{M_m\rightarrow \sum_\ell t_\ell M_\ell}
which yields the more general string equations:
\eqn\msvmd{\sum t_\ell S_\ell +{1\over 2m+1} z f=0 }
By arguments analogous to those above we find that 
the orthonormal wavefunctions must also satisfy linear equations 
in the variables $t_\ell$:
\eqn\laxiii{
\biggl({d\over d t_\ell}+ \bigl
(\zeta C_\ell+\half V'_\ell\bigr)\sigma_3
+(\zeta V_\ell+S_\ell)\sigma_1 -\half V'_\ell i\sigma_2\biggr)\vec\psi =0}
The compatibility of \laxi\ with \laxiii\ now shows that 
the solution $f$ to \msvmd\ , as a function of $t_\ell$ 
satisfies mKdV flow
\eqn\mckf{
{\p f\over \p t_\ell}={\p \over \p x}S_\ell[f]
}
while the solution $f$ to 
\msvmd\ defines a self-similar solution to the flow.

As in the case of the level-spacing problem we can describe 
the partition function in terms of a fermion correlator. 
If we change the potential by changing $t_\ell\to t_\ell+\delta t_\ell$
then, for $N$ finite we may write:
\eqn\heuri{{Z(t_\ell+\delta t_\ell)\over Z(t_\ell)}=
\langle N|exp\biggl\{-\int_{-\infty}^\infty \sum_\ell\delta t_\ell
a^{(2m-2\ell)/m}V_\ell(\lambda)\psi^\dagger\psi(\lambda)\biggr\}|N\rangle}
By analytic continuation we can drop the subscript $+$ in \mulpt\ 
and, for small $\lambda$, 
$V_\ell(\lambda)\sim i k(\ell) \lambda^{2\ell+1}/(2\ell+1)$.
Changing variables $\lambda\to a^{1/m}\lambda$ and taking
the continuum limit gives
\eqn\heurii{
{Z(t_\ell+\delta t_\ell)\over Z(t_\ell)}=
\biggl\langle :exp\biggl\{-i\int_{-\infty}^\infty \sum_\ell{k(\ell)\over
2\ell +1}\delta t_\ell
\lambda^{2\ell+1}\psi^\dagger\psi(\lambda)\biggr\} :\biggr\rangle}
Because we interchanged limits several times \heurii\ must be regarded
as heuristic.

\subsec{Single-Cut solutions}

We now consider the phase of the matrix model in which the 
eigenvalue distribution forms a single cut.
The critical behavior arises from the 
integration near a nonzero value of $\lambda=\lambda_c$, 
and for the $m^{th}$ multicritical point we have the scaling behavior
$R_n\to r_c + a^{2/m} u(z)$ for $n/N=1-a^2 (z-z_0)$, with
$Na^{2+1/m}$ fixed
\BK\DS\GM\ .
In this
case it is natural to assume that the orthonormal wavefunctions
have a limit as $a\to 0$: 
$$p_n(\lambda_c+a^{2/m}\lambda)\to a^{-1/2m}p(z,\lambda)\qquad .$$
The recursion relation becomes
\eqn\schrod{\biggl({d^2\over dz^2} + u(z)\biggr)p(z,\lambda)=\lambda 
p(z,\lambda) 
}
so the double scaling limit of the orthonormal
wavefunctions defines a Baker-Akhiezer function
\foot{We take the double scaling limit of orthonormal 
wavefunctions explicitly, for the case of a gaussian ensemble, 
in appendix A.}.
Similarly, in the limit we have the quantum 
field
$$a^{1/m}\psi(\lambda_c+a^{2/m}\lambda)\rightarrow 
\hat \psi(\lambda)=\int dz a(z) p(z,\lambda)$$
and the two point function is simply 
$$\eqalign{
\langle z_0|\hat \psi^\dagger(\lambda_1)
\hat\psi(\lambda_2)|z_0\rangle&=\sqrt{r_c}
{p'(z_0,\lambda_1)p(z_0,\lambda_2)-
p'(z_0,\lambda_2)p(z_0,\lambda_1)\over \lambda_1-\lambda_2}\cr
&\equiv K_m(\lambda_1,\lambda_2)\cr}\qquad .$$
Following the procedure of \dnsmt\Korep\Korepi\ as before
we consider the linear ODE satisfied
by $\vec\psi=(p'(x,\lambda)~p(x,\lambda))$:
\eqn\lopp{
\CL\vec\psi\equiv \biggl[-{d\over dx} + \pmatrix{0&\lambda+u\cr
1&0\cr}\biggr]\vec\psi=0 \qquad .}

>From the matrix model it is clear that 
$$\biggl({d\over d\lambda}+N_m(\lambda,u)\biggr)\vec\psi=0$$ 
where $N_m$ 
is polynomial in $\lambda$ and differential polynomial in $u$. 
The commutator of $\CL$ with $N_m$ is $\pmatrix{0&1\cr 0&0\cr}$
so, by the method of 
Zakharov-Shabat, $N_m$ is a linear combination of Lax pairs for 
the KdV hierarchy. The $\ell^{th}$ 
KdV flow can be written \DrS\ 
as the compatibility condition
$[2\p/\p t_\ell + \CP_\ell,\CL]=0,$
where the $sl(2)$ matrix 
\eqn\popp{
\CP_\ell\equiv \pmatrix{A_\ell& B_\ell\cr
C_\ell&-A_\ell\cr} }
may be expressed in terms of the conserved densities $R_\ell$ of 
KdV flow
\ref\Gelf{I.M. Gelfand and L.A. Dickii, ``Asymptotic Behavior of the 
Resolvent of Sturm-Liouville Equations and the Algebra of the 
Korteweg-De Vries Equations,'' Russian Math Surveys, {\bf 30}(1975)77.}
via
\eqn\potent{\eqalign{C_\ell&
=R_\ell+\lambda R_{\ell-1}+\cdots 
+ \lambda^{\ell-1}R_1 + \lambda^\ell R_0\cr
A_\ell&=\half C_\ell'\cr
B_\ell&=(\lambda+u)C_\ell-A_\ell'\cr} }
Comparing with the tree-level equations we learn that if we define
\eqn\wittx{
\IP=-{d\over d\lambda}-\half\sum_j (j+\half)t_{j}\CP_{j-1}
+\pmatrix{0&x-(\sum_j (j+\half) t_jR_j)\cr
0&0\cr} }
(where $\CP_{-1}=0$) 
then the orthonormal wavefunctions satisfy
\eqn\linsys{\eqalign{
\IP \vec\psi(\lambda,x,t_j)&=0\cr
\CL \vec\psi(\lambda,x,t_j)&=0\cr
(2{d\over dt_j}+\CP_j)\vec\psi(\lambda,x,t_j)&=0\cr}\qquad . }
Compatibility of the first and last pairs gives
the massive $(2l-1,2)$ and KdV equations, respectively. 
The compatibility conditions for the first and 
third equations then follow from the string and KdV 
equations. 
Similar considerations apply to the $(p,q)$ 
equations.

If we perturb around the multicritical point we 
add the potential \GM\GMi\ 
\eqn\prtrb{
\delta V= N\sum \delta t_\ell (\lambda-\lambda_c)^{\ell+1/2} a^{(2m-2\ell)/ m}
\qquad .}
As in the derivation of \heurii\ the
partition function becomes
\eqn\newtpt{
\langle z_0|:exp\biggl\{\int_{-\infty}^\infty 
\psi^\dagger(\lambda)\psi(\lambda)
\sum \delta t_\ell \lambda^{\ell+1/2}d~\lambda \biggr\}:|z_0\rangle
}
Thus, as a function of the $\delta t_\ell$ the partition function 
is simply the fredholm determinant $det(1-K)$ for the 
kernel defined by 
\eqn\kerna{\eqalign{
K(\lambda_1,\lambda_2;\delta t_\ell)
&=\sqrt{\vartheta(\lambda_1)}K_m(\lambda_1,\lambda_2)
\sqrt{\vartheta(\lambda_2)}\cr
\vartheta(\lambda)&=\sum \delta t_\ell \lambda^{\ell+1/2}\cr}
\qquad .}
The convergence of the integrals is 
rather delicate. The only case we can analyze explicitly 
is the ``topological
point'' $m=1$ in which case the relevent integrals are 
conditionally convergent.

\subsec{Relation to other work}

The emergence of a quantum field theory on the 
spectral curve has been discovered in several different
guises in recent investigations into matrix models.
In 
\ref\das{S.R. Das and A. Jevicki, ``String field 
theory and physical interpretation of d=1 strings,'' 
Brown preprint BROWN-HET-750}
\ref\sengupt{A.M. Sengupta and S.R. Wadia, ``Excitations
and interactions in d=1 string theory,'' Tata preprint}
\ref\gross{D. Gross, Lectures at the Carg\`ese workshop 
and at CERN, June 1990}\ 
it is shown that one may derive a quantum field theoretical 
representation of the $d=1$ matrix model involving 
fields $\phi(t,\lambda)$ where $t$ is the 1-dimensional 
time and $\lambda$ is the eigenvalue coordinate of the 
string. This might be the $d=1$ version 
of the phenomena discussed for $d<1$ in this paper, and
it would be extremely interesting to relate the 
$d=1$ theories to the $d<1$ theories by some analog of the 
Feigin-Fuks construction. On the other hand, we have 
analytically continued to complex $\lambda$, while 
$\lambda$ is kept real in \das\sengupt\gross\ .


There has also been a good deal of fuss over the 
discovery that the partition function of the matrix 
model is annihilated by operators forming a 
subalgebra of the Virasoro algebra
\ref\fukuma{M. Fukuma, H. Kawai, and R. Nakayama, Univ. of 
Tokyo preprint UT-562}
\dvv\morozovi\morozovii\ . We will see in section 4.3 that the 
2D quantum field theory on the spectral curve is related to 
a 2D conformal field theory, thus making the appearance of 
a Virasoro algebra completely natural
\foot{Essentially the same remark was made in a pretty
paper by A. Mironov and A. Morozov \morozovii\ .}.
Still, the relation between the two formalisms remains
to be clarified.

It would be extremely interesting to understand how
the field theory on the spectral curve 
arises from 
the sum over topologies of conformal-field-theoretic
correlators.
Some fascinating speculations along these lines are 
described in 
\ref\knizhnik{V.G. Knizhnik, ``Multiloop amplitudes in 
the theory of quantum strings and complex geometry,'' 
Usp. Fiz. Nauk. {\bf 159}(1989)401. English translation:
Sov. Phys. Usp. {\bf 32}(1989)945, section 12}
\morozovi\ .


\newsec{Isomonodromic Deformation}

In the previous section we saw the appearance of a 2D quantum field 
theory on the spectral curve, and interpreted the 
string equations and KdV flow as 
compatibility conditions for first order linear differential 
equations satisfied by 
correlation functions in the theory.
These equations are transport equations for 
a flat connection. Similar equations,
for example null vector equations or 
the Knizhnik-Zamolodchikov equation, 
appear in conformal field theory. These equations depend
on parameters, e.g. the 
moduli of some Riemann surface, and have interesting 
monodromy. Typically, deforming the parameters
leaves the monodromy 
unchanged. In conformal field theory this is 
often expressed in terms of the flatness of the Friedan-Shenker
vector bundles over moduli space
\ref\frsh{D. Friedan and S. Shenker, ``The analytic geometry of
two-dimensional conformal field theory,'' Nucl. Phys. {\bf B281}
(1987)509; D. Friedan, ``A new formulation of string theory,'' 
Physica Scripta T {\bf 15}(1987)72.}. This 
is an example of isomonodromic deformation. In this section
we will see that isomonodromic deformation is the key behind
the linear systems occuring in all the matrix model 
problems discussed above.

\subsec{The level-spacing problem}

We indicate the origin of \levlin . 
The monodromy of the two-point function \levii\ 
as a function of $\lambda$ gives information on 
the partition function \taulevel\ 
\dnsmt\ 
\foot{For readers of \dnsmt\ one can identify, for example,
$R_I^\pm(x,x';\xi)$ with the two point function of
$\hat\chi^\dagger_i(x)$ with $\hat\psi(x')$, and so on.}. 
Because of the singular terms
in the ope, $\chi_\alpha^\dagger$ has a nontrivial exchange algebra
with $exp\int_I \chi^\dagger\chi$ so $\Psi$ has monodromy
as $\lambda$ is continued around the endpoints of $I$.
Moreover, by locality of the ope 
the monodromy matrix is unchanged as we deform
the parameters $a_i$ in the complex plane. Finally, the
meromorphic matrix
$d\Psi \Psi^{-1}$
where $d$ is exterior differentiation in $\lambda,\lambda',a_i$ is determined
by its singularities at $\lambda=a_i,\lambda'=a_i,\lambda=\lambda'$ 
which are 
in turn dictated by the singular terms in the ope. 
In this way we can 
derive linear differential equations in $\lambda,a_i$ which are
analogous to the Knizhnik-Zamolodchikov equations, 
obtained 
here as isomonodromic deformation conditions.
Moreover, 
by studying the compatibility conditions for these linear 
equations we obtain nonlinear constraints which allow us
to derive information on the partition function, which is 
simply the tau function for isomonodromic deformation
(see below). Therefore, since \laxi\laxii\ and \linsys\ 
are analogues of \levlin\  we
would 
like to interpret them as isomonodomic 
deformation conditions. This requires an enrichment of our notion
of monodromy. 

\subsec{Stokes phenomenon}

The isomonodromic deformation method focuses on the monodromy
of the solutions of the equation in $\lambda$. Since 
\laxii\linsys\ is of the form
${d\over d\lambda}\Psi=\CA(\lambda)\Psi$ where $\CA$ is polynomial
in $\lambda$ 
it would appear that there can be no monodromy. In fact, the 
singularity at infinity induces a kind of monodromy in the 
form of stokes phenomenon.

To put stokes phenomenon in perspective let us consider a general
differential equation 
\eqn\lin{{d\over d\lambda}\Psi=\CA(\lambda)\Psi}
where $\CA$ is meromorphic in $\lambda$. 
Near a 
regular singular point $\lambda_0$, where $\CA$ has a pole,
we may write a formal solution to the differential equation as
\eqn\regi{\Psi(\lambda)=\hat\Psi(\lambda)e^{M~log(\lambda-\lambda_0)}}
where $\hat\Psi$ is a formal power series in $\lambda-\lambda_0$.
If we want true solutions corresponding to this formal solution
we first choose angular regions of $\lambda_0$ in which we can 
define a branch of the logarithm. 
The formal and true solutions coincide 
in the angular regions
\foot{It is a 
nontrivial property of differential equations with regular 
singularities that $\hat \Psi$ is in fact a convergent power 
series.} ,
but upon comparison in the overlaps 
of regions we conclude that the true solutions differ by
the monodromy $e^{2\pi iM}$. 

Stokes phenomenon appears when we try to find solutions to 
\lin\ and $\CA$ has an irregular singular point,
i.e., a pole of order larger than one.
For simplicity assume that 
$$\CA(\lambda)={A_{-r}\over (\lambda-\lambda_0)^{r+1}}+\cdots$$
with $A_{-r}$ diagonalizable. In this case it can be shown
\ref\Wasow{W. Wasow, {\it Asymptotic Expansions for Ordinary 
Differential Equations}, Interscience, 1965.}
that there is a formal solution 
\eqn\formi{\Psi=\hat\Psi e^{T(\lambda-\lambda_0)}} 
with 
$$T={D_{-r}\over (\lambda-\lambda_0)^r}+\cdots
{D_{-1}\over(\lambda-\lambda_0)}+M~log~(\lambda-\lambda_0)$$
where the $D_i$ are diagonal and commute with the ``formal
monodromy'' $M$. In general $\hat \Psi$ is only an asymptotic
series. It nonetheless has nontrivial analytic meaning 
since there exist angular sectors in 
the neighborhood of the singular point
% as in
%\fig\sectors{Stokes sectors in the neighborhood of an 
%irregular singular point. The two solutions $\Psi_{1,2}$ are
%asymptotic to the formal solution only in 
%the indicated sectors.}
in which there exist true solutions of \lin\ which are 
asymptotic to the formal solution \formi\ . The regions 
can be found by considering the nature of the 
essential singularity $e^T$. Notice that this is 
exponentially growing and decaying in the angular 
sectors where $\Re(\lambda-\lambda_0)^r$ is positive and negative,
respectively.
Typically a true solution asymptotic to the formal 
solution can only be defined in a sector of angular
width $\pi/r$ which contains regions of both 
growth and decay. Therefore if we compare two 
such solutions which are defined on regions with 
a nontrivial overlap they may 
differ by right-multiplication by a constant 
matrix. Such matrices are called stokes matrices, 
and, labelling the sectors by $\Omega_k$ we obtain
a set of stokes matrices $S_k$ associated with the 
differential equation. 

According to 
\ref\flasch{H. Flaschka and A. Newell, ``Monodromy and 
Spectrum-Preserving Deformations I,'' Comm. Math. Phys. 
{\bf 76}(1980)65.}\refs{\Jimboi{--}\Jimboiv}\ 
the stokes matrices are a generalization 
of the monodromy of the differential equation.
This does not mean that a solution to the differential
equation cannot be analytically continued to a single valued
solution in an entire neighborhood of the singular point.
The point is, 
such a single-valued solution has the ``wrong'' asymptotic 
behavior in all but two sectors.

A relevant example is provided by:
\eqn\airi{
W'=\biggl(2\zeta^2\sigma_3-{1\over 2 \zeta}\sigma_1\biggr)W \qquad .}
This is equivalent to the equation in $\lambda$ in \linsys\
for the ``topological'' case $\ell=0$. The equation \airi\ 
has an irregular singularity of order 3 at infinity, 
and $T={2\over 3}\zeta^3\sigma_3$ so there are six stokes
sectors defined by the rays $\theta=\pm{\pi\over 6},\pm{\pi\over 2},
\pm{5\pi\over 6}$. We can solve \airi\ exactly in terms of 
Airy functions. 
For example, defining $Ai_1(z)\equiv Ai(ze^{-2\pi i/3})$ we may 
write the fundamental solution, valid in the sector 
$\Omega_0\cup\Omega_1=\{\zeta|-{\pi\over 6}<
arg~\zeta< {\pi\over 2}\}$, 
\eqn\airii{
W_1=\sqrt{\pi\over\zeta}
\pmatrix{Ai_1'(\zeta^2)+\zeta Ai_1(\zeta^2)&
Ai'(\zeta^2)+\zeta Ai(\zeta^2)\cr
Ai_1'(\zeta^2)-\zeta Ai_1(\zeta^2)&
Ai'(\zeta^2)-\zeta Ai(\zeta^2)\cr}
\pmatrix{e^{-i\pi/6}&\cr &-1\cr} 
}
whereas in the sector $\Omega_1\cup\Omega_2=\{\zeta|{\pi\over 6}<
arg~\zeta< {5\pi\over 6}\}$ we must use:
\eqn\airiii{\eqalign{
W_2 &=\sqrt{\pi\over 4\zeta}
\pmatrix{-i(Ai'(\zeta^2)+\zeta Ai(\zeta^2))
&-(Ai'(\zeta^2)+\zeta Ai(\zeta^2))\cr
-i(Ai'(\zeta^2)-\zeta Ai(\zeta^2))&
-(Ai'(\zeta^2)-\zeta Ai(\zeta^2))\cr}\cr\cr
&+
\sqrt{\pi\over 4\zeta}
\pmatrix{Bi'(\zeta^2)+\zeta Bi(\zeta^2)&
i(Bi'(\zeta^2)+\zeta Bi(\zeta^2))\cr
Bi'(\zeta^2)-\zeta Bi(\zeta^2)&
+i(Bi'(\zeta^2)-\zeta Bi(\zeta^2))\cr}\cr}
}
These two solutions are single-valued and holomorphic in the 
region $arg~\zeta\not=\pi$ but are only asymptotic to the 
formal solution in the angular regions described above. 
Moreover, 
they are related by $W_2=W_1S_1$ where the stokes matrix is 
easily seen to be
\eqn\airiv{S_1=\pmatrix{1&i\cr 0&1\cr} \qquad .}
Similarly, from standard formulae one can find the 
the stokes matrices for the other sectors.

\subsec{Asymptotic analysis of the PII family}

We now show that the linear ODE's of sec. 2 are
isomonodromic deformation conditions.
The differential equation
\eqn\lasaga{{d\over d\lambda} \Psi=\sum t_{\ell} M_\ell\Psi }
depends on parameters $z,t_j,f,f_z,f_{zz},\dots$, where, 
for the moment, we consider $f,f_z,f_{zz},\dots $ to be 
independent quantities.
We will see in section four that $z,t_j$ may be thought
of as generalized moduli so it is 
natural to ask what conditions $f,f_z,f_{zz},\dots$
must satisfy if the monodromy(=stokes data) 
is to be independent of the 
moduli(=$z,t_j$). We may answer this question as follows.

As we vary $z,t_j$ we obtain a family of invertible matrix
solutions $\Psi(\lambda,z,t_j)$ to \lasaga\ 
so we may consider the quantity ${\p \Psi\over \p z}\Psi^{-1}$, which 
is a rational matrix in $\lambda$. The only singularities 
can be at $\lambda=\infty$, so we need only know the behavior 
of a solution to \lasaga\ in this limit. Therefore we 
perform an asymptotic solution of \lasaga\ 
by relating the matrix elements of \res\ 
to the resolvent $R(x,\lambda^2)$ 
of the Schr\"odinger operator $L=D^2+u$. The result is 
that $\Psi\sim\hat\Psi e^T$
where  
$$T=\half\biggl(\sum_\ell{2m+1\over 2\ell+1}t_\ell 
\lambda^{2\ell+1}\biggr)\sigma_3
$$
and 
$$\eqalign{
\hat\Psi&=1+{\hat \Psi_1\over \lambda}+{\hat \Psi_2\over \lambda^2}+\cdots\cr
\hat\Psi_1&={-if\over 2}\sigma_2+H\sigma_3\cr}$$
where $H'=\half f^2$. Since the stokes data is unchanged under
deformation of $z$ we can compute $\p_z\Psi \Psi^{-1}$ 
by substituting the asymptotic expansion and we find
the condition \laxi\ . From this condition it follows that, 
as functions of $z$, $f_z$ must be the derivative of $f$ etc., 
and moreover
$f(z)$ must satisfy the string equation.
Similarly, one may 
compute ${\p \over\p t_\ell}\Psi~ \Psi^{-1}mod (1/\lambda)$
to obtain the linear condtion \laxiii\ .This shows that if one 
considers a solution to \msvmd\ as a function of the $t_\ell$ then,
if the stokes data are fixed, $f$ must satisfy mKdV flow.
 
\subsec{Asymptotic analysis of the PI family}

A very similar computation can be carried out for 
the $\lambda$ equation in \linsys\ .
There is one (important)
technical change.
Since the highest power of $\lambda$ does not
multiply an invertible matrix one must make a 
transformation \Jimboii\
\ref\Kapaev{A. Kapaev, ``Asymptotics of solutions of the Painlev\'e
equation of the first kind,'' Differential Equations, 
{\bf 24}(1988)1107.}
$\lambda=\zeta^2$ and
\eqn\zee{\eqalign{
\Psi(\lambda)&=\zeta^{1/2}\pmatrix{1&1\cr
1/\zeta&-1/\zeta\cr} W(\zeta)\cr
W(\zeta)&=\zeta^{-1/2}\half \pmatrix{1&\zeta\cr
1&-\zeta\cr}\Psi(\lambda)\cr} }
The leading singularity in the equation for $W$ now 
multiplies $\sigma_3$ and $W\sim \hat W e^{T}$ where 
\eqn\asymp{
\eqalign{
T&=
\biggl(-{1\over 4}
\sum_{j=1} t_j \zeta^{2j+1} +\zeta x\biggr)
\sigma_3\cr
\hat W&=1+ {H_1\over \zeta}\sigma_3 +
{u\over 4\zeta^2}\sigma_1 +\CO(1/\zeta^3)\cr} }
and $H_1'=\half u(x)$. 
Again,
this may be proved by relating the matrix elements in 
the differential equation to the resolvent of the Schr\"odinger
operator $L=-D^2+u$. As before, if we deform $x$ keeping the 
stokes data fixed then we can evaluate the expression 
${\p \Psi\over \p x}\Psi^{-1}$ from its asymptotics.
(In this case we must take into account 
the regular singularity at zero.)
The result is just:
\eqn\laxiv{{\p\Psi\over \p x}\Psi^{-1}=\pmatrix{0&\lambda+u\cr
1&0\cr}
}
and comparing with section 2.3 we see that this means
$u(x)$ must satisfy the string equation. Similarly we
can ask for the condition that deformations in $t_j$
keep the stokes data fixed, and we find remaining linear
equations in \linsys\ so that $u(x;t_j)$ satisfies 
KdV flow. 

The string equation and KdV flow are compatible equations.
This does not mean that, as we change the $t_j$ a solution 
to the string equation
automatically satisfies KdV flow. Nevertheless,
we show in sec. 3.6 that the physical 
asymptotic conditions on $u(x)$ 
fix the stokes data uniquely, 
thus proving that 
$u(x;t_j)$ satisfies KdV flow. 
The following argument for KdV flow has been 
proposed in \emil\witrev\morozovi\ .
In the matrix model, before the continuum limit is taken,
the jacobi matrix representing multiplication by 
$\lambda$ in the space of orthogonal polynomials 
satisfies toda flow. It was proposed  some time ago
\ref\moser{J. Moser, ``Finitely many mass points on the line under
the influence of an exponential potential--an integrable system,''
in  Lecture Notes in Physics {\bf 38}, J. Moser, ed., p. 467}
that the continuum limit of toda flow should give 
(m)KdV flow.

\subsec{$\tau$ functions}

One of the beautiful results of the Kyoto school 
\refs{\Jimboi{--}\Jimboiv}\ 
is the definition of the 
$\tau$ function for isomonodromic deformation, which 
motivated the perhaps better known tau function 
of the KP hierarchy. Applying \Jimboi\Jimboii\ to 
our case gives a closed one-form
\eqn\clsdfrm{
\omega=Res_{\zeta=\infty} tr\Biggl[ 
\bigl(\hat W^{-1}{d\hat W\over d\zeta}\bigr) d T\Biggr] }
on the space of deformation parameters.
Since $\omega$ is closed 
one can define (locally)
the {\it tau function} via $\omega=d(log \tau)$.
Substituting \asymp\  into \clsdfrm\ 
we get ${\p\over \p x}(log\tau)=-2H_1$. 
On the other hand, \asymp\ implies that $H_1'=u/2$ 
hence
\eqn\freeeng{
u(x;t_j)=-{d^2\over dx^2}log \tau }
so the partition function of the matrix model
(whose logarithm is the partition function of 2D 
gravity) is simply the tau function for isomonodromic
deformation.
Since the tau function is holomorphic
\ref\Miwa{T. Miwa, ``Painlev\'e property of monodromy
preserving deformation equations and the analyticity of 
$\tau$ functions,'' Publ. Res. Inst. Math. Sci. {\bf 17}
(1981)703}\ 
we see that all the string equations have the 
painlev\'e property:
the {\it only} singularities of a solution
$u$ to the string equations are second order poles. 
This result may be easily extended to the entire hierarchy 
of $(p,q)$ string equations.

Applying the definition \clsdfrm\ to the asymptotic 
expansion for the PII family gives
\eqn\tauf{f^2=2H'={\p^2\over \p z^2}log~ \tau \qquad .}
On the other hand, computing the connected two-point function of 
$tr\phi^2$ one can show \peri\dss\crmr\ 
\eqn\twopt{
\langle tr\phi^2~tr\phi^2\rangle_c =R_N\bigl(R_{N+1}+R_{N-1}\bigr)\to
{1\over 8} - a^{2/m} f^2 }
so that for this universality class we also can 
identify the partition function with the tau function. 


\subsec{Stokes matrices for the PI family}

In order to obtain physical solutions to the 
string equation we need to choose proper 
boundary conditions. In the isomonodromic 
deformation literature it is shown that the 
initial conditions 
may be taken to be the stokes
matrices of the associated linear problem. 
In this section we derive the stokes data 
for the PI family for the ``physical'' solutions
to the string equations.

The issue of the correct choice of boundary conditions 
for a physically acceptable solution to the string equation
is dictated by 
the matrix model integral
\BMP
\ref\David{F. David, ``Phases of the large N matrix model
and non-perturbative effects in 2d gravity,'' Saclay
preprint SPhT/90-090}. 
For example, in the case of the $m=3$ member of the PI family,
i.e., $R_3[u(x)]=x$, the physical asymptotics are given by
$u\sim \CO(x^{1/3})$ as $x\to\pm\infty$. Since each asymptotic
condition fixes two boundary conditions we expect that these
asymptotics uniquely specify the solution.
>From the numerical 
integration in \BMP\ it appears that the solution is pole-free. 

We will show that the above asymptotics 
uniquely determine the stokes parameters in the monodromy 
problem associated to $R_3=x$. Since one can reconstruct 
a solution from the stokes data (via the star operator, see below)
it follows that the solution is unique.
This solution is real for all $x$. 

We investigate the direct monodromy problem following closely
the treatment in
\ref\Itsi{See reference \Its, especially, chapter 5.}
\Kapaev . After the transformation \zee\
our equation can be written as
\eqn\dforz{
 {d W\over d\zeta}=\Biggl[(B+\zeta^2 C+\Delta)
\sigma_3-(B-\zeta^2 C+\Delta)i\sigma_2+(2\zeta A
-{1 \over 2\zeta})\sigma_1\Biggr]W}
where 
$$\eqalign{
C\equiv -\half\sum_j (j+\half)t_{j}C_{j-1}&
\qquad \qquad \Delta= x -\sum (j+\half)t_jR_j\cr
A= \half C' &\qquad\qquad B=(\lambda+u)C-A'\cr}
$$
Equation \dforz\ 
has an irregular singularity of order $2\ell+3$ at infinity
and a regular singularity at the origin. 

Consider first some general properties of the stokes 
matrices. If $\ell$ is the largest index with 
$t_{\ell+1}\not=0$ there will be
$4\ell+6$ stokes sectors
$\Omega_k$ each containing a unique ray 
$\theta={\pi\over 4\ell+6}(2k-1)$, $k=0,\dots 4\ell+5$
along which 
$cos[(2\ell+3)\theta]=0$, thus we may take neighborhoods of 
infinity defined by:
\eqn\sectors{
\Omega_k\equiv \{\zeta| {\pi\over 4\ell+6}+ {\pi\over 2\ell+3}(k-2)
< arg \zeta < {\pi\over 4\ell+6}+ {\pi\over 2\ell+3}k\} }
for $k=0,\dots, 4\ell+5 ~mod~ 4\ell+6$. 
On the overlap $\Omega_k\cap\Omega_{k+1}$ the two solutions 
$W_{k+1}$ and $W_k$ asymptotic to $\hat W e^T$ must be 
related by stokes matrices $W_{k+1}=W_k S_k$.
The asymptotic expansion constrains the 
$S_k$ to be of the form
\eqn\sto{
S_{2k}=\pmatrix{1&0\cr s_{2k}&1\cr}\qquad\qquad
S_{2k+1}=\pmatrix{1&s_{2k+1}\cr 0&1\cr}\qquad .  }

The stokes parameters $s_i$ are not all independent but 
satisfy:
\eqn\stoper{s_{k+2\ell+3}=s_k}
\eqn\mncnst{S_1\cdots S_{2\ell+3}=i \sigma_1 }
\eqn\storl{s_{2\ell+3-k}=-\bar s_k\qquad {\rm if\  u\  is\  real} }

We may prove these properties as follows.
For \stoper\ we use
the symmetry of equation \dforz\ to conclude that
$W_{k+2\ell+3}(\zeta)=\sigma_1W_k(-\zeta)\sigma_1$ and hence that 
$S_{k+2\ell+3}=\sigma_1 S_k\sigma_1=S_k^{tr}$.
For \mncnst\  we remark that the original 
equation in $\lambda$ is regular throughout the $\lambda$ plane.
The solution is simply $Pexp\int^\lambda A(\lambda')d\lambda'$ for
an appropriate matrix $A$ hence 
the only singularities in $\Psi$ can occur at infinity. Thus, near
$\zeta=0$ we have
\eqn\nrzr{
W(\zeta)\cong \half\zeta^{-1/2}\pmatrix{1&\zeta\cr 1&-\zeta\cr} 
}
By
\nrzr\ , if we analytically continue $W_1$ from $\Omega_1$ then 
$W_1(-\zeta)=i\sigma_1W_1(\zeta)$ from which we obtain \mncnst\ .
Finally if $u$ is real then all the coefficients of powers of 
$\zeta$ in \dforz\ are real so that $\bar W_k(\zeta)=W_{-k}(\bar\zeta)$
implying \storl\ .

Since the determinant of \mncnst\ is automatically satisfied, 
\mncnst\ only imposes three independent constraints on the 
stokes matrices so
we have $2\ell+3-3=2\ell$ independent stokes parameters.
Note that this is the number of initial conditions in the 
string equation. In fact, the two sets of parameter spaces 
may be regarded as the same
\Its\ref\itsnovok{A.R. Its and V. Yu. Novokshenov, ``Effective
Sufficient Conditions for the Solvability of the Inverse Problem
of Monodromy Theory for Systems of Linear Ordinary Differential
Equations,'' Funct. Anal. and Appl. {\bf 22}(1988)190.}.


Now we simplify \dforz\ using the physical asymptotics. 
Consider \dforz\ 
as $x\to \pm\infty$ first in the case where $t_j=0$ for 
$j\not= \ell+1,0$. 
Since $R_\ell=\kappa_\ell u^\ell +\cdots $ with 
$$\kappa_\ell=(-1)^\ell{(2\ell-1)!!\over 2^{\ell+1}\ell!}$$
perturbation theory predicts that the solution to
the string equation satisfies
\eqn\ascses{\eqalign{
u\sim \bigl(-1/2\kappa_{\ell+1}
\bigr)^{1/(\ell+1)
}x^{1/(\ell+1)} &\qquad \ell\quad
{\rm even},\quad x\to\pm\infty\cr
u\sim 
\pm\bigl(1/2\kappa_{\ell+1}\bigr)^{1/(\ell+1)}(-x)^{1/(\ell+1)} 
&\qquad \ell\quad
{\rm odd}, \quad x\to -\infty\cr}
}
In fact, perturbation theory tells us to take the $+$ root for $u$ 
in the case of odd $\ell$, but we may easily examine both cases at once.
Rescaling variables and only keeping leading order terms 
\dforz\ becomes
\eqn\limeqt{
{dW\over d\xi}=\tau\Biggl[ (2\xi^2\pm 1)p_\ell^\pm(\xi)\sigma_3
\mp p_\ell^\pm(\xi)i\sigma_2
-{1\over 2\xi \tau}\biggl({8\ell\kappa_{\ell+1}
\over \ell+1}\xi^2 p_\ell(\xi)+1 \biggr)
\sigma_1
+\CO(1/\tau^2)\Biggr]W
}
where
$p_\ell^\pm(\xi)=\sum_{p=0}^\ell (\pm 1)^p\kappa_p\xi^{2\ell-2p}$.
For $\ell$ odd we obtain this equation with $x\to -\infty$, 
where: 
$$\zeta=(-x/2\kappa_{\ell+1})^{1/(2\ell+2)}\xi \qquad\qquad 
\tau=(-x/2\kappa_{\ell+1})^{(2\ell+3)/(2\ell+2)}\qquad .$$ 
For $\ell$ even we obtain this equation with $\pm x\to +\infty$, where
$$\zeta=(\pm x/(-2\kappa_{\ell+1}))^{1/2\ell+2}\xi \qquad\qquad 
\tau=(\pm x/(-2\kappa_{\ell+1}))^{(2\ell+3)/(2\ell+2)}\qquad .$$
When $t_j\not=0$ for $j\leq \ell$ we scale $\zeta^2=u_0\xi^2$
where $u_0$ is the solution of the tree-level equation, so 
$\tau=|u_0|^{\ell+3/2}$. Expression \limeqt\ still holds but
$p^\pm_\ell(\xi)$ receives corrections of order $\CO\bigl(t_j
\tau^{-2(\ell+1-j)/(\ell+1)}\bigr)$. These will not affect 
our argument below. 

The evaluation of the stokes matrices is carried out by doing 
a WKB analysis in the $\tau\to\infty$ limit as in \Its . Thus 
true solutions to \limeqt\ are asymptotic as
$\tau\to\infty$ to the WKB ansatz:
\eqn\wkbansatz{ W^{WKB}\sim
T exp\Biggl[\tau\int^\zeta\Lambda(\zeta')d\zeta'
\Biggr]
}
where $T$ diagonalizes \limeqt to $\Lambda=\mu\sigma_3$. 
There are several WKB solutions in different regions defined by 
the turning points and conjugate stokes lines. The turning 
points are simply the roots $\xi_i$ of $\mu$, and the  
conjugate stokes lines are the lines defined by the vanishing real 
part:
\eqn\skis{
\Re\int_{\xi_i}^{\xi}\mu(\xi')d\xi'=0
}
The general procedure for finding the stokes matrices is described in 
\Its . A simple consequence of this procedure allows us to 
obtain certain necessary conditions on 
the stokes matrices which, in some cases, fix the parameters 
uniquely. The main observation is that
if, at a turning point 
which is a root of $p_\ell$ the conjugate stokes lines form three large 
regions each abutting an open region at infinity 
then the stokes matrix for the transition function associated with the middle 
region $\tilde \Omega_2$ is trivial \geom\ .

Therefore we describe the stokes lines.
The lines for
the case $\ell=2$ have the form
\vfill\eject
\centerline{}
\vskip 3.5in
The limit $x\to +\infty$ gives $s_1=s_6=0$ and
the limit $x\to -\infty$ gives $s_2=s_5=0$.
>From the monodromy constraints we get
$s_0=s_3=s_4=i$.
Unfortunately,
if we move on to higher $\ell$ 
the configuration of stokes lines becomes
too complicated to use our observation immediately to set 
half the stokes parameters to zero. 
Nevertheless, the pattern for small $\ell$ leads to a 
natural guess for the stokes parameters in 
the general case. For 
$\ell$ even, comparing the constraints from the 
physical asymptotics
at either end 
of the axis should fix two disjoint sets of stokes
parameters. We expect that $s_1=s_2=\cdots =s_\ell=0$
while $s_{\ell+1}=s_{\ell+2}=s_{2\ell+3}=i$ so that the solution 
is unique, and the stokes data is always concentrated on 
the wedges abutting the $x$ and $y$ axes in the 
$\zeta$ (=$\sqrt{\lambda}$) plane. Similar considerations
hold for $\ell$ odd. In particular, we expect
(and can prove for the case $\ell=1$) that \ascses\ 
forces $s_0=s_2=\cdots =s_{\ell-3}=s_{\ell-1}=0$,
$s_{\ell+1}=s_{\ell+2}=i$. 

We have not used the reality 
constraints in an essential way. Repeating
the analysis for the triply truncated
solution, characterized by $u\sim +(-x)^{1/2}$ for 
$x\to-\infty$ and $u\sim\pm i x^{1/2}$ for $x\to \infty$,
one easily shows that in this
case the constraints $s_2=s_3=i$, $s_0=s_5=0$, 
$s_1+s_4=0$ are supplemented by $s_1=0$ or $s_4=0$, 
depending on the sign of the imaginary part. In either
case we confirm that the solution is unique.

An analog of the BMP solutions exists for the 
PII family
\ref\Hast{S.P. Hastings and J.B. McLeod, ``A boundary value 
problem associated with the second painlev\'e transcendent and the 
Korteweg-de Vries equation,'' Arch. Rat. Mech. and Anal. {\bf 73}(1980)31}
\ref\cdm{\v C. Crnkovi\'c, M. Douglas, and G. Moore, 
``Physical solutions for unitary-matrix models,'' Yale preprint
YCTP-P6-90.}. 
In this case physical asymptotics specify
that $u(x)$ grows algebraically at one end of the 
axis and decays exponentially at the other end
\dss\cdm\ . Applying the above techniques one finds 
that in this case (at least for the first two members
of the PII family) the stokes data is concentrated 
entirely on the $y$-axis, but the necessary conditions
leave a single parameter undetermined. In this case,
however, the inverse monodromy problem is equivalent
to an integral equation (the self-similar Gelfand-Levitan-Marchenko
equation) which can be examined directly. 
This is done in \cdm\ where it is shown that 
the physical solution is unique.

\subsec{KdV orbits are disconnected}

Since KdV flow is compatible with the string equations
we can consider the KdV orbits of solutions of the 
string equation. These orbits are parametrized by the 
largest index $\ell$ for which $t_\ell\not=0$.
We may wonder whether the
orbits of physical solutions are connected. 
Certainly they are formally connected. For example,
consider the equation
\eqn\flowi{
\half(m_1+\half)R_{m_1}+\half(m_2+\half)T R_{m_2}=x
}
with $m_1>m_2$. As discussed in \bdss
\ref\cgmr{\v C. Crnkovi\'c, P. Ginsparg, and G. Moore, ``The Ising model,
the Yang-Lee edge singularity, and $2D$ quantum gravity,'' Phys. 
Lett. {\bf 237B}(1990)196.}\ 
if one scales a solution $u(x;T)$ to \flowi\ using 
$v(y;T)=T^{2/(2m_2+1)}u\bigl(T^{1/(2m_2+1)}y;T\bigr)$, then 
the large $T$ limit $v(y)=\lim_{T\to\infty}v(y;T)$ must
be a solution of the lower order string equation 
$\half(m_2+\half)R_{m_2}[v]\sim x$, {\it provided the limit
exists.} The existence of this limit is a very delicate 
issue, and in \dss\ numerical evidence is presented that
the flow from $m=3$ to $m=2$ does not exist. 

Since physical asymptotics fixes the stokes 
data, and since KdV flow is isomonodromic, we are in a position 
to investigate analytically the result in \dss\ .
Suppose the $T\to\infty$ limit does exist.
Then we can scale $\zeta\to T^{-1/(2m_2+1)}\zeta$ in \dforz\ 
to obtain an equation with a smooth $T\to\infty$ limit. Since
solutions can in principle be obtained from the path-ordered exponential
of the ``gauge field'' on the rhs of \dforz\ ,
solutions to \dforz\ will also have smooth 
$T\to\infty$ limits. Moreover, from the asymptotics in $\zeta$
we see that the coefficients have smooth $T\to\infty$ limits
and in fact approach the asymptotics of the lower order equation.
Thus, fundamental solutions smoothly approach fundamental solutions for the
lower order equation, although they will be defined on small regions of
angular width ${2\pi\over 4m_1+2}$ and hence only define part of a
fundamental solution for the lower order equation which is defined on the
larger regions of angular width ${2\pi\over 4m_2+2}$. Because of this 
we find two rules governing flows:

\noindent
1. A large region associated with a trivial stokes matrix for the 
``$m_2$ equation'' cannot contain a small region with a nontrivial
stokes matrix for the ``$m_1$ equation.''

\noindent
2. A large region associated with a nontrivial stokes matrix for the 
``$m_2$ equation'' must contain at least one small region with a 
nontrivial stokes matrix for the ``$m_1$ equation.''

Note that for even and odd $\ell$ the nontrivial stokes data disagrees
on the real axis. Hence, by rule 1 it is impossible to use KdV flow 
to go from an even $\ell$ to an odd $\ell$ model, confirming the result of 
\dss\ . Note that flow from an even $\ell$ to a smaller even $\ell$
is consistent with rules 1 and 2. An alternative argument for the 
nonexistence of certain flows is given in \David\ .

\subsec{Stokes phenomenon and the eigenvalue distribution}

The above WKB analysis is very similar to the work
of F. David \David\ , who studies carefully the 
saddle-point eigenvalue distribution, taking into account global
stability criteria. The reason for the similarity is that, at finite
$N$ the eigenvalue density $\rho_N(\lambda)$ and the fermion 
two-point functions are connected through
\eqn\evi{\eqalign{
\rho_N(\lambda)&={1\over N}\langle N|\psi^\dagger\psi(\lambda)|N\rangle\cr
&={\sqrt{r_N}\over N}\biggl(p_N{\p\over\p\lambda}p_{N+1}-p_{N+1}
{\p\over\p \lambda}p_N\biggr)\cr} }
and in the double scaling limit we therefore have
\eqn\evii{
\rho_N(\lambda_c+a^{2/m}\lambda)\longrightarrow
a^{2-1/m}\biggl[\bigl({\p\over\p z}p\bigr)\bigl({\p\over\p \lambda}p\bigr)
-p\bigl({\p\over\p z}{\p\over\p \lambda}p\bigr)\biggr] 
\qquad .}
The eigenvalue distribution vanishes algebraically along a 
cut $\CC$ in the $\lambda$ plane, ending at $\lambda_c$
and vanishes exponentially beyond $\lambda_c$. This has 
important consequences for the large $\lambda$ asymptotics
of $p(z,\lambda)$. Introducing a second solution of \schrod\ ,
$q(z,\lambda)$ so that $p,q$ have unit wronskian we may 
write a matrix solution to \linsys\ 
\eqn\eviii{\Psi=\pmatrix{\p_z p&\p_z q\cr p&q\cr} }
According to \wkbansatz\ the large $\lambda$ asymptotics 
of $\Psi$ have the form
\eqn\eviv{
\Psi\sim{1\over \lambda^{1/4}}\pmatrix{\lambda^{1/2}&-\lambda^{1/2}\cr
1&1\cr}\biggl(1+\CO(\lambda^{-1/2})\biggr)
exp\biggl[-\sigma_3\int^\lambda \mu(\lambda')d~\lambda'\biggr] N}
where $\mu^2=A^2+BC$ and $N$ is an undetermined constant matrix.
Note that the rhs of \evii\ is the $21$ matrix element of 
$\Psi^{-1}\p_\lambda\Psi$. Therefore, just beyond $\lambda_c$
along the cut $\CC$, $p$ must decrease exponentially, 
and we can take $N=1$. On the other hand, along the cut 
$\CC$ the algebraic behavior of $\rho$ implies that $p$ 
is not pure exponential but instead is a cosine or a sine. 
This asymptotic behavior requires a different matrix
\eqn\evv{N\propto \pmatrix{1&1\cr 1&-1\cr} }
Thus, the exponential vanishing of the eigenvalue density 
outside of $\CC$ and the algebraic vanishing of the 
density along $\CC$ implies that the Baker-Akhiezer functions
{\it must} exhibit stokes phenomenon. Moreover, it follows 
that the eigenvalue
density in the scaling region of $\lambda_c$ is given by 
$-i\mu(\lambda)$. In particular,
using the tree level approximation to $\mu$ 
we may obtain 
the tree level effective potential $Re[G(\lambda)]$ for eigenvalues 
from 
\eqn\evvi{G'(\lambda)=C[u,\lambda]\sqrt{\lambda+u} }
where we use the tree level approximation for $C,u$. 
Thus, David's stability sectors coincide with the sectors
\skis\ around the point $\xi_i=\lambda_c^{1/2}$
and his global stability condition is essentially the 
condition that the integration over eigenvalues must not 
pass through a region in which the double scaling limit of the 
orthogonal polynomials has exponential growth.


%David also finds an obstruction in the flow
%from $m=3$ to $m=2$ if $t_2$ is kept real, but his 
%obstruction can be evaded by making $t_2$ complex. On the other
%hand, the stokes data argument makes no distinction between
%real and complex $t_2$. Since the absence of David's obstruction
%does not guarantee the existence of a flow the results are not
%(yet) in contradiction.
%These issues merit closer scrutiny.

\newsec{Isomonodromy and Free Fermions}

In this section we will 
interpret the isomonodromy problem connected with the 
string equations in conformal-field-theoretic terms. 
Our paradigm will be the solution of the Riemann-Hilbert
problem for the case of regular singular points given ten 
years ago by the Kyoto school
\ref\holoii{M. Sato, T. Miwa, M. Jimbo, ``Holonomic Quantum Field Theory
II,'' Publ. RIMS {\bf 15}(1979)201}.
We review their construction first, in the light of subsequent
developments in CFT. Then we consider the case of irregular 
singular points. Developing further some work of Miwa
\ref\miwai{T. Miwa, ``Clifford operators and Riemann's monodromy problem,''
Publ. Res. Inst. Math. Sci. {\bf 17}(1981)665}, we find
that the theory of irregular singular points can be included 
at the expense of the introduction of a new kind of operator.
We are 
extending conformal field theory by expanding 
the class of functions admitted in the theory from analytic functions 
with algebraic singularities to analytic functions with essential 
singularities. This is reflected in the need to expand the 
class of operators from twist operators to star operators.

\subsec{Regular Singular Points}

The basic idea of \holoii\ is that the solution to an $m\times m$
matrix differential equation 
\eqn\diffl{
{d\Psi\over dz}=A(z)\Psi(z) }
may be characterized uniquely by its monodromy properties. 
More precisely, suppose $A$ has only simple poles at points 
$a_\nu$ and the residue can be diagonalized to $L_\nu$. Then
the matrix $\Psi$ can be uniquely characterized by 
the requirement that 

(i.) $\Psi(z_0)=1$

(ii.) $\Psi(z)$ is holomorphic and invertible 
in $z\in\IP^1-\{a_1,\dots , a_n\}$

(iii.) $\Psi(z)=\hat \Psi^{(\nu)}(z)
e^{L_\nu log(z-a_\nu)}$ for, $z\cong a_\nu$
where $\hat \Psi^{(\nu)}$ is holomorphic and invertible in a neighborhood
of $a_\nu$. 

Conversely, any such matrix defines a rational matrix
$A=\Psi_z\Psi^{-1}$ with at most simple poles.
Thus, 
if one can construct appropriate ``twist operators'' 
$\varphi_i$
such that the correlation function 
\eqn\corr{\Psi_{\beta\alpha}(z_0;z)=
(z_0-z)
{\langle
\bar \psi_\beta(z_0)\psi_\alpha(z) \varphi_n(a_n)\cdots \varphi_1(a_1)
\rangle\over
\langle \varphi_n(a_n)\cdots\varphi_1(a_1)\rangle } 
}
has the correct monodromy properties, then it must be a solution of 
\diffl . Thus we have reduced the global Riemann-Hilbert
problem to the {\it local} problem
of finding conformal fields $\varphi_{L}(a)$ with the operator 
product expansion 
\eqn\twis{
\psi_\alpha(z)\varphi_L(a)\sim \bigl[\CO^0_{L,\alpha}(a)+
(a-z)\CO^1_{L,\alpha}(a)+\cdots\bigr](z-a)^{L_\alpha} }

The construction of these operators proceeds by
choosing a basis of curves
$\gamma_\nu$ circling once around $a_\nu$ and generating the 
fundamental group $\pi_1(\IP^1-\{a_\nu\};z_0)$ 
and defining:
\eqn\gentwis{
\varphi_{M}(a)=exp\Biggl[\int_\CC^a tr\biggl\{log(M)J(y)\biggr\}
{dy\over 2\pi}\Biggr] 
}
where $J_{\beta\alpha}=\bar\psi_\beta\psi_\alpha$ 
and $\CC$ is a contour (a branch cut
for $\Psi$) emanating from $a$. Consider now \corr\  with such
operators inserted. As we analytically continue $z$ around
$a$ the simple pole in the ope of $\psi$ with $J$
gives rise to the monodromy $\psi_\alpha\to \psi_\gamma M_{\gamma\alpha}$
in the Fermi field. Therefore \corr\ solves 
conditions ({\it i-iii}). 

Let us now consider isomonodromic deformation of \diffl\ .
It is clear from locality of the ope that changing the 
$a_\nu$ leaves the monodromy data unchanged. 
Necessary and sufficient conditions
for isomonodromic deformation are given by:
\eqn\schles{\eqalign{
{\p \Psi\over \p z_0}&=-\sum {A_\nu\over z_0-a_\nu}\Psi\cr
{\p \Psi\over \p a_\nu}&=\biggl(-{A_\nu\over z-a_\nu}+{A_\nu\over z_0-a_\nu}
\biggr)\Psi\cr}
}
The
compatibility conditions for these linear equations give the 
nonlinear Schlesinger equations
\foot{For an appropriate choice of matrices, for example, 
these equations reduce to PVI \Jimboii .}.
>From \corr\ we see that the linear equations of
isomonodromic deformation theory should 
be thought of as transport equations on moduli space, 
analogous to the Knizhnik-Zamolodchikov equations,
so that the theory of isomonodromic deformation
for regular singular points fits nicely 
into the framework of Friedan-Shenker modular geometry.

The tau function associated to the deformation parameters $a_\nu$
is defined to be \Jimboi\Jimboii\Jimboiii\
\eqn\taufn{
d~log~\tau(a_1,\dots, a_n)=-\sum_\nu Res_{z=a_\nu}tr
\Biggl[ (\hat \Psi^{(\nu)})^{-1}
{\p \hat \Psi^{(\nu)}\over\p z} d \biggl(log(z-a_\nu)L_\nu\biggr)\Biggr]
}
where the $d$ is a differential in the parameters $a_\nu$. 
In fact the $\tau$ function is given by
$\tau=\langle \varphi_1(a_1)\dots\varphi_n(a_n)\rangle$\ 
\Jimboi\Jimboii\Jimboiii\ .
We will now rederive this
using general principles of conformal field theory
\foot{Exactly the same derivation was given 3 years 
ago by V. Knizhnik, \knizhnik\ , ch. 4. I was unaware of this work when 
I published \geom .}.

We have normalized \corr\ so that it is equal to $\delta_{\alpha\beta}$
at $z=z_0$. Taking the 
operator product expansion as $z\to z_0$ and matching this with 
an expansion of a solution to \diffl\ around $z_0$ we find
\eqn\corrii{\eqalign{
{\langle
J_{\beta\alpha}(z_0) \varphi_n(a_n)\cdots \varphi_1(a_1)
\rangle\over
\langle \varphi_n(a_n)\cdots\varphi_1(a_1)\rangle }
&=-A(z_0)_{\beta\alpha} \cr
{\langle
\CT(z_0) \varphi_n(a_n)\cdots \varphi_1(a_1)
\rangle\over
\langle \varphi_n(a_n)\cdots\varphi_1(a_1)\rangle } 
&=\half tr A^2(z_0)\cr} }
where 
$\CT$ is the stress energy tensor. Since $L_{-1}$ always takes a
derivative with respect to position we have
$${\p\over\p a_\nu}log[\langle\varphi(a_1)\cdots\varphi(a_n)\rangle]
=
\oint_{a_\nu}dz_0
{\langle
\CT(z_0) \varphi_n(a_n)\cdots \varphi_1(a_1)
\rangle\over
\langle \varphi_n(a_n)\cdots\varphi_1(a_1)\rangle } 
=
\oint_{a_\nu}dz_0 \half trA^2(z_0)$$
On the other hand, substituting the local expansion 
$\Psi=\hat\Psi(z-a)^L$ we get
\eqn\lcexp{\hat \Psi^{-1}\hat \Psi_z + L/(z-a)
=\hat \Psi^{-1}A\hat \Psi\qquad .}
Squaring this equation we find
\eqn\sqaring{
{\p\over\p a}\bigl(log~\tau\bigr)=Res~tr~\biggl(\hat\Psi^{-1}\hat\Psi_z
{L\over z-a}\biggr)=\half Res~tr~A^2
\qquad ,}
and hence the tau function is simply the correlation function 
of twist operators. 

\subsec{Irregular Singular Points}

Let us now attempt to repeat the 
previous discussion for the case of 
a differential equation \diffl\ where 
$A$ is rational but can have irregular singularities. 
Our treatment is the same in spirit as the discussion of 
Miwa \miwai\ , although there are some
differences of detail.

Recall that
at an irregular singular point we divide up a neighborhood 
of the point into sectorial domains, each containing 
a fundamental solution with asymptotics 
\eqn\stoii{
\Psi\sim\biggl(\sum_{l\geq 0}\hat\Psi^{(l)}(z-a)^l\biggr)(z-a)^L
e^{T(z-a)} }
where $L$ and 
$$T(z-a)=\sum_{i=1}^{r}{t_r\over (z-a)^r}$$ 
are diagonal, and $\hat \Psi^{(0)}$ 
is invertible. In 
particular, 
the analytic continuation of $\Psi_1$ will have the asymptotic 
expansion 
\eqn\stoki{
\Psi_1\sim\biggl[\sum_{l\geq 0}
\hat\Psi^{(l)}(z-a)^l\biggr](z-a)^L e^T(S_1\cdots
S_{k-1})^{-1} }
in the sector $\Omega_k$. 
A solution to the differential equation can be uniquely characterized
by the conditions $(i)-(iii)$ above except that $(iii)$ must
be replaced by the requirement that $\Psi$ have the asymptotic
expansions \stoki\ . 
Assume there is only one
irregular singular point and
define $\tilde \Psi\equiv \Psi_1 e^{-T(z-a)}$. 
We now search for quantum 
field operators $V_{S,T,L}(a)$, which we call ``star'' operators,
such that 
\eqn\corstar{
\tilde\Psi_{\beta\alpha}(z_0;z)= 
(z_0-z)
{\langle
\bar \psi_\beta(z_0)\psi_\alpha(z) V_{S_1,t_1,L_1}(a_1)\cdots \rangle\over
\langle V_{S_1,t_1,L_1}(a_1)\cdots\rangle } 
}
Evidently,
a star operator is characterized by its operator product expansion 
with $\psi$, $\bar\psi$, e.g., for $z\in \Omega_k$ we have
\eqn\starope{
\psi_\alpha(z)V_{S,T,L}(a)\sim \bigl[\CO^0_{\gamma}(a)+
(a-z)\CO^1_\gamma(a)+\cdots\bigr](z-a)^{L_\gamma}\bigl[
e^T(S_1\cdots S_{k-1})^{-1}e^{-T}\bigr]_{\gamma\alpha}
}
>From this description it looks very unlikely that star operators
exist 
\foot{For example, it is often claimed that operator product expansions
in CFT are convergent. Note that \starope\ is only asymptotic.
The reason for this is ultimately to be found in the fact that 
the string coupling has become dimensionful
\DS . Note that it is $x$ and the masses $t_j$ 
which multiply the terms giving the essential 
singularity at infinity. }.
We now give at least a formal construction of these operators.

Consider a ray $\CC$ emanating from a point $a$. Consider the product 
of operators 
$$exp\bigl[\int^a_\CC dy\psi_\alpha(y) M_{\alpha\beta}(y)\bar\psi_\beta(y)
\bigr]\psi_\alpha(z)$$
where $M$ is some matrix defined along the line. 
If we analytically continue in $z$ through 
the curve $\CC$ and compare with the other operator ordering 
it is a simple consequence of Cauchy's theorem and the 
operator product expansion that we have the exchange algebra:
\eqn\galg{\eqalign{
exp\bigl[\int^a_\CC dy\psi_\alpha(y) M_{\alpha\beta}(y)\bar\psi_\beta(y)
\bigr]\psi_\alpha(z-\epsilon)&=\cr\cr
\psi_\gamma(z+\epsilon) (e^{M(z)})_{\gamma\alpha}&
exp\bigl[\int^a_\CC dy\psi_\alpha(y) M_{\alpha\beta}(y)\bar\psi_\beta(y)
\bigr]\cr}
}
where $z$ is a point on $\CC$ and $z\pm\epsilon$ are points above and 
below $z$. Thus, defining $\CS_k\equiv e^T S_k e^{-T}$ we may define, 
at least formally, 
\eqn\stardef{
V_{S,T,L}(a)=\varphi_L(a)\prod_k exp\Biggl[\int^a_{\CC_k}
tr\biggl\{(log \CS_l(y))J(y)\biggr\}dy \Biggr]
}
where we choose contours $\CC_k$ in $\Omega_k$ such that the 
matrix $\CS_k$ approaches the identity rapidly. 
In \miwai\ Miwa obtained formulae for star operators using 
a slightly different formalism. Comparing his formulae in 
terms of expansions evaluated by Wick's theorem we 
obtain the same result.
As shown in \miwai\ contours of integration can be defined so 
that for small enough stokes data the integrals make 
sense, thus giving a more precise definition to the star 
operator.

It follows from locality of the operator product 
expansion that the differential equation 
satisfied by \corstar\ has the property that the monodromy
data $S,L$ are preserved if we vary the parameters $a_i,t_i$. 
In strict analogy with the case of regular singularities,
the tau function for 
isomonodromic deformation with irregular singular
points
is given by the correlation function
of star operators \miwai\ . One may give a formal 
argument for this following steps analogous to those 
leading from \corrii to \sqaring\ . The analog of \corrii\ is
\eqn\stress{\eqalign{
{\langle
J_{\beta\alpha}(z_0) V_{S_1,t_1,L_1}(a_1)\cdots 
\rangle\over
\langle V_{S_1,t_1,L_1}(a_1) \cdots\rangle }
&=-A(z_0)+T'(z_0)\cr
{\langle
\CT(z_0) V_{S_1,t_1,L_1}(a_1)\cdots 
\rangle\over
\langle V_{S_1,t_1,L_1}(a_1)\cdots \rangle } 
&=\half tr (A-T')^2(z_0)\cr}
}
where $\CT$ is the stress-energy tensor.
Using formal manipulations with the ope one can show that
\eqn\gobpe{
-Res_{z_0=a} tr\delta T(z_0) A(z_0)=
{\sum_k\int^a_{\CC_k}dy tr\biggl(\delta T(y){\delta\over
\delta T(y)}\biggr)\langle V_{S,T,L}\cdots\rangle\over
\langle V_{S,T,L}(a)\cdots \rangle}
}
Putting together \stress\ and \gobpe\ we then find
\eqn\taustar{
{d\over da}~log~\langle V\cdots V\rangle= \half Res_{z_0=a} 
tr~A^2(z_0)
={d\over da}~log~\tau
}
where the second equality follows from an argument similar to 
the case of regular singularities. Similarly, one can show
that the dependence
of $log~\tau$ and $log\langle V\cdots V\rangle$ on other 
parameters is the same.

\subsec{$\tau$ functions for 2D gravity}

As a special case of the above formalism we can express the solution 
$u$ of the string equations in terms of a fermion correlation function. 
We represent the solutions $W$ to \dforz\ and $\Psi$ to \lasaga\ as 
fermion two-point functions in the presence of star operators. For 
the $\tau$ function of the PII family we may define
$t(y)\equiv \half
\sum_{\ell\leq m} {2m+1\over 2\ell +1}t_\ell \lambda^{2\ell+1}$, so that 
$\tau(t_\ell)=\langle V(\infty;s_k,t_\ell)\rangle$ where
\eqn\srptw{\eqalign{
V(a;s_k,t_\ell)=\prod_k &
exp\biggl[s_{2k+1}\int^a_{\CC_{2k+1}}e^{2t(y)}\psi_1\bar\psi_2(y)\biggr]\cr
&\qquad
exp\biggl[s_{2k}\int^a_{\CC_{2k}}e^{-2t(y)}\psi_2\bar\psi_1(y)\biggr]\cr}
}
For the PI family we have a fermion twist operator at the origin 
(from the regular singularity in \dforz\ ). If we bosonize the 
two fermions $e^{i\phi_i}=\psi_i$ then the twist 
operator at the origin is just $e^{i\omega/\sqrt{2}}$
where $-i\sqrt{2}\p\omega=\bar\psi\sigma_3\psi$. The 
$\tau$ function is now 
\eqn\srpon{
\tau(t_\ell)=\langle e^{-i\omega(\infty)/\sqrt{2}}
V(\infty;s_l,t_\ell) e^{i\omega(0)/\sqrt{2}}\rangle}
where the star operator is the same as in the PII case but 
$t(y)=-{1\over 4}\sum t_j\zeta^{2j+1}$ and the stokes data is 
nonzero for sectors along the $x,y$ axes. 
The bosonized
currents $\bar\psi_2\psi_1,\bar\psi_1\psi_2=e^{\pm i\sqrt{2}\omega}$
involve only one scalar field $\omega$
so we have a correlator
in a $c=1$ system. The expression \srpon\ may be written explicitly
using Wick's theorem. The contour integrals diverge near $y=0$
but this divergence may be regulated and is $t_\ell$-independent.

These expressions are, of course, rather formal. It would be worthwhile
making rigorous sense of them since, at least formally, they make
transparent some interesting properties of the 2D gravity 
partition function. For example, a standard corollary of the operator 
formalism is that a $\tau$ function satisfies certain 
Virasoro constraints. Applying \stress\ to this case
with $A,T$ obtained from \dforz\lasaga\ we find:
\eqn\vircni{
\eqalign{
L_n\tau&={1\over 4}\delta_{n,0}\tau\qquad n\geq-1\qquad{\rm for~
PI}\cr
L_n\tau&=0\qquad \qquad n\geq-1\qquad{\rm for~
PII}\cr} }

Similarly since commutation with $J=\bar\psi\sigma_3\psi$
rotates the fermions $\psi_{1,2}$ oppositely we may imagine 
that there is an identity like
\eqn\cohi{
e^{\int_\CC\bigl(t^{(2)}(y)-t^{(1)}(y)\bigr)J(y)dy}
V(a;s,t^{(1)}_\ell)
e^{-\int_\CC\bigl(t^{(2)}(y)-t^{(1)}(y)\bigr)J(y)dy}
=V(a;s,t^{(2)}_\ell)}
where $\CC$ surrounds $a$.
Hence we would have
\eqn\cohii{
\tau(t_\ell)=\langle t_\ell-\bar t_\ell|\Omega_{\bar t}\rangle}
where $|\Omega_{\bar t}\rangle$ is the state created by 
the star operator at $\bar t$, and $\langle t_\ell-\bar t_\ell|$
is a coherent state for the scalar $\omega$ where only the 
odd oscillator modes are excited. (This follows since 
$t(y)$ involves only odd powers of $y$.) Combining with 
\vircni\ we may obtain, formally, expressions similar 
(but not equal) to those
in \fukuma\dvv\morozovi\morozovii\ .

Finally, let us compare with the fermion formalism of the 
matrix model described in section 2. There we found that the 
partition function at couplings $t_\ell$ can be expressed in 
terms of a correlation function in the ground state for 
couplings $\bar t_\ell$ according to:
\eqn\mtrxtaus{
\eqalign{
\tau_{PI}&=\biggl\langle exp\bigl(\int_\IR\sum_{\ell}\bigl(t_\ell-\bar
t_\ell\bigr)\lambda^{\ell +1/2}\psi^\dagger\psi(\lambda) d\lambda\bigr)
\biggr\rangle\cr
\tau_{PII}&=\biggl\langle exp\bigl(i\int_\IR\sum_{\ell}\bigl(t_\ell-\bar
t_\ell\bigr)\lambda^{2 \ell +1}\psi^\dagger\psi(\lambda) d\lambda\bigr)
\biggr\rangle\cr} }
While heuristic, these expressions 
are strikingly similar to \cohii\ and could explain why,
for matrix model asymptotics, the stokes data is 
nontrivial, and concentrated along the $y$ axis, for the PII
family and along the $x,y$ axes for the PI family.

\newsec{Grassmannians, Krichever's construction, and all that}

When the connection to the KdV hierarchy was discovered in 
2D gravity the theory of the KdV hierarchy, as presented in 
\ref\BMN{Dubrovin, Matveev, and Novikov, ``Non-Linear Equations 
of Korteweg-De Vries Type, Finite-Zone Liner Operators, and 
Abelian Varieties,'' Russian Math Surveys, {\bf 31} (1976)59}
\ref\qpKdV{E. Date, M. Jimbo, M. Kashiwara, and 
T. Miwa, ``Transformations Groups for Soliton Equations,''
I. Proc. Japan Acad. {\bf 57A}(1981)342; II. Ibid., 387;III. J. Phys. Soc.
Japan {\bf 50}(1981)3806;IV. Physica {\bf 4D}(1982)343;V. Publ. RIMS, 
Kyoto Univ. {\bf 18}(1982)1111;VI. J. Phys. Soc. Japan {\bf 50}
(1981)3813;VII. Publ RIMS, Kyoto Univ. {\bf 18}(1982)1077.}
\ref\segal{G. Segal and G. Wilson, ``Loop Groups and Equations of 
KdV Type,'' Publ. I.H.E.S. {\bf 61}(1985)1.} , 
was already familiar to string theorists. 
Indeed, one of the main points of the so-called operator
formalism
\foot{Representative papers include
\ref\luis{L. Alvarez-Gaum\'e, C. Gomez, G. Moore, and 
C. Vafa, Nucl. Phys. {\bf B303}(1988)455}
\ref\wittgr{E. Witten, ``Conformal field theory, Grassmanians, 
and algebraic curves,'' Commun. Math. Phys. {\bf 113}(1988)189}\
and references therein.}
is the equivalence of 
the Krichever theory with the 
theory of free fermions on an algebraic curve.
In the previous section we related the $\tau$
function of 2D gravity to a correlation function of 
free fermions.
In this section we will see that 
because of stokes phenomenon we require an 
extension of the the theory in \BMN\qpKdV\segal\ .

\subsec{Quasiperiodic KdV flow and isomonodromy}

We begin by reviewing a remark of M. Jimbo and T. Miwa
that the tau function of the quasiperiodic
solutions to the KdV equations is a special case of 
the isomonodromic tau function \Jimboii\ .

Recall that quasiperiodic KdV flow is straightline 
motion along $Pic_{g-1}(X)$ for a riemann surface $X$
of genus $g$, and, fixing an origin $\CL_0$ for
$Pic_{g-1}$ the tau function is just 
\eqn\qptau{\tau(\CL)={Det~\bar\p_{\CL}\over Det~\bar\p_{\CL_0}} \qquad .}
If $\CL\otimes\CL_0^{-1}$ has divisor $P_1+\cdots P_g-Q_1-\cdots-Q_g$
then by the insertion theorem
\ref\bost{L. Alvarez-Gaum\'e, J.-B. Bost, G. Moore, P. Nelson, 
and C. Vafa, Phys. Lett. {\bf 178B}(1986)41;Commun. Math. 
Phys. {\bf 112}(1987)503}
we have
\eqn\qpti{\tau(\CL)=\langle\psi(P_1)\cdots\psi(P_g)\bar\psi(Q_1)\cdots
\bar\psi(Q_g)\rangle_{(X,\CL_0)} \qquad .}
Choosing a point $P_\infty$ ``at infinity'' and a local coordinate
$1/z$ near $P_\infty$ the Baker function is essentially
\eqn\qpbki{\langle\bar\psi(P_\infty)
\psi(z)\psi(P_1)\cdots\psi(P_g)\bar\psi(Q_1)\cdots
\bar\psi(Q_g)\rangle_{(X,\CL_0)} \qquad .}
Now suppose $\pi:X\to \IP^1$ is an $m$-fold branched covering.
As is well-known from the theory of orbifolds 
\nref\zam{Al. B. Zamolodchikov, ``Conformal scalar field on the 
hyperelliptic curve and critical Ashkin-Teller multipoint 
correlation functions,'' Nucl. Phys. {\bf B285}(1987)481.}
\nref\berrad{M. Bershadsky and A. Radul, ``Conformal field theories
with additional $Z_N$ symmetry,'' Int. Jour. of Mod. Phys. 
{\bf A2}(1987)165.}
\nref\lance{L. Dixon, D. Friedan, E. Martinec and S. Shenker, 
``The conformal field 
theory of Orbifolds,'' Nucl. Phys. {\bf B282}(1987)13.}
\nref\vafa{ S. Hamidi and C. Vafa, ``Interactions on Orbifolds,'' 
Nucl. Phys. {\bf B279}(1987)465}\refs{\zam{--}\vafa}
\knizhnik\ 
we can represent one weyl fermion on $X$ by $m$ weyl fermions 
$\psi_\alpha,\bar\psi_\alpha$ on $\IP^1$, where 
$\alpha=1,\dots, m$ labels the sheets, in the presence of 
twist operators at the branch points. For example,
denoting the branch points on $\IP^1$ by $b_i$ we have
the twist field correlator
$Det~\bar\p_{\CL_0}=\langle \prod\varphi_i(b_i)\rangle.$
Similarly, if $P_i$, $Q_i$ lie on branches $\alpha_i,\beta_i$, 
respectively then the Baker function descends to the 
``Baker framing''
\eqn\qpbkfr{\tilde Y_{\alpha\beta}(\lambda)=
\langle\bar\psi_\alpha(\infty)
\psi_\beta(\lambda)\psi_{\alpha_1}(\pi(P_1))\cdots
\psi_{\alpha_g}(\pi(P_g))\bar\psi_{\beta_1}(\pi(Q_1))\cdots
\bar\psi_{\beta_g}(\pi(Q_g))\prod_i\varphi_i(b_i)
\rangle_{\IP^1} }
where $\lambda=\pi(z)$. (The actual Baker framining $Y$ differs
from $\tilde Y$ by an invertible 
diagonal matrix with an essential singularity
at infinity of the form $\sim exp\bigl(\sum t_j z^j\bigr)$.)
Regarding the fermion insertions as special cases of twist operators
and following the reasoning of the previous section it follows that 
$Y$ satisfies a linear ODE in $\lambda$ with regular singularities
at $\pi(P_i)$. Clearly we have isomonodromic deformation in 
these parameters, so the isomonodromic $\tau$ function is 
just
\eqn\qptii{
\biggl\langle
\psi_{\alpha_1}(\pi(P_1))\cdots
\psi_{\alpha_g}(\pi(P_g))\bar\psi_{\beta_1}(\pi(Q_1))\cdots
\bar\psi_{\beta_g}(\pi(Q_g))\prod_i\varphi_i(b_i)
\biggr\rangle_{\IP^1} \qquad ,}
but this is the same as \qpti\ which is the tau function of 
the quasiperiodic KdV equations.

>From this discussion we conclude that the required generalization of 
the operator formalism is the generalization from twist operators 
to star operators. In the next section we will arrive at the 
same conclusion from a different point of view.

\subsec{Noncommutative Burchnall-Chaundy-Krichever theory}

One way to understand better the geometrical 
meaning of star operators is to investigate directly 
the required generalization of the Burchnall-Chaundy-Krichever
theory through which one associates a riemann surface with a 
line bundle to a solution of the stationary KdV equations. 

Recall that in the quasiperiodic theory we have 
$[P,L]=0$ where $L=D^2-u(x)$ and $P=\sum t_j L^{(2j+1)/2}|_+$.
By simultaneously diagonalizing $P,L$ we find that $P,L$ 
must be algebraically related $P^2-Q(L)=0$ for some polynomial 
$Q$, defining a hyperelliptic curve $\Sigma$. Indeed, in
terms of $A,B,C$ defined in \potent\ we have the curve 
$\Sigma=\{(\mu,\lambda)|\mu^2=A^2+BC\}$. The simultaneous 
eigenfunction satisfies
$L\psi(\lambda,x)=\lambda \psi(\lambda,x)$, i.e., is a
Baker-Akhiezer function and is 
a section of a line bundle $\CL\to \Sigma$. KdV flows 
starting from $u(x)$ are elegantly described as straightline
flows of $\CL$ along the Jacobian of $\Sigma$. Note especially
that the parameters $t_j$ are coordinates of hyperelliptic 
moduli space and are {\it unchanged} under KdV flow.

In 2D gravity we have the equation $[P,L]=1$. This means we 
cannot diagonalize $P,L$ simultaneously. What should we do?
One idea
\foot{which also occured to C. Itzykson, C. Vafa, and 
possibly others}\ is that we should try to 
define a noncommutative spec
in the spirit of noncommutative geometry, but
this has not yet 
been pursued very far. An easier route is to consider
the equation $[P,L]=\hbar$ and try to understand the 
$\hbar\to 0$ limit
\geom
\ref\novikov{S.P. Novikov, ``On the equation $[L,A]=\epsilon\cdot 1$,''
preprint.}
\ref\krich{I. Krichever, ``On heisenberg reltions for the 
ordinary linear differential operators,'' ETH preprint}\ . 
Although the details in these papers are very similar the 
authors draw rather different conclusions. We now briefly sketch 
some salient points of these papers.

Consider a family of solutions $u_\hbar$ to the 
string equations $\sum (j+\half)t_jR_j=\hbar x$, where
$t_j=t_j^{(0)}+\hbar t_j^{(1)}+\cdots$.
By naive scaling one might think that the 
limit $\hbar\to 0$ corresponds to the limit $x\to 0$ 
of a solution at $\hbar=1$ but 
this ignores possible $\hbar$ dependence of the boundary 
conditions. Using the analysis of Boutroux (and its 
extension to the higher order string equations) one may show
that it is possible to choose boundary conditions
so that $u_\hbar = u^{(0)}+\hbar u^{(1)}+\cdots$ and 
$u^{(0)}$ is a nontrivial solution of the stationary 
KdV equations (e.g. it is a Weierstrass $\wp$ function 
in the analysis of Boutroux). 

In \geom\ we attempted to generalize the BCK theory 
by reinterpreting the equation $[P,L]=0$ as equations 
on the matrix Baker-Akhiezer function.
It was shown 
that in the case $\hbar\not=0$ 
the equations generalize directly and are the first two compatibility 
equations of \linsys\ . 
In particular the Baker-Akhiezer function exhibits 
stokes' phenomenon (as we saw in section 3.6). Any
geometrical interpretation must take account of this 
fact. One attempt, in terms of framings of push-forwards
of Krichever's line bundle, was made in \geom .
An interesting consequence of this proposal is 
that the parameters $t_j^{(0)}$ and $t_j^{(1)}$ 
defined by $t_j=t_j^{(0)}+\hbar t_j^{(1)}+\cdots$
are, respectively, 
coordinates for hyperelliptic moduli space and the 
jacobian of the curve $\Sigma(t^{(0)}_j)$. This reconciles 
the peculiar fact that in the stationary case KdV flow 
is a flow in $t^{(1)}_j$ while in the $\hbar\not=0$ case 
the flow is in $t_j$. 

In \novikov\krich\ the equation $\mu^2=A^2+BC$ is 
taken to define a riemann surface $\Sigma$. 
The recursion relations for kdv 
potentials imply that
${\p\over \p x}(A_\ell^2+B_\ell C_\ell)=-2 R_{\ell+1}' C$.
Therefore, 
in the stationary 
case the curve $\Sigma$ is independent of 
$x$ (indeed, $x$ parametrizes a direction along 
$Jac(\Sigma)$). When $\hbar\not=0$ we have instead
\eqn\dercurv{
{\p\over \p x} \biggl(A^2+BC[\lambda,u_\hbar]\biggr)
=\hbar C[\lambda,u_\hbar] }
so that the curve $\Sigma$ now depends on $x,t_j$, and 
we have a family of Riemann surfaces. Now, {\it if} 
the standard BCK theory applied then we could
identify \BMN\  
\eqn\thetaf{u(x;t_j)
=2{\p^2\over \p x^2}log~\vartheta(\vec v x+\vec w|\tau)}
where $\tau$ itself depends on $x,t_j$. In fact,
substitution into \dercurv\ then yields 
a nontrivial transcendental equation for the moduli 
as a function of $x,t_j$ \novikov\krich\ . 

Unfortunately, it is not clear that solutions of the type 
\thetaf\ will have physically relevent asymptotics. 
For example, the transcendental equations in \novikov\krich\ 
might not have solutions. (It seems that, at best, the 
curve must be completely degenerate.)
There is also a technical snag
in applying the BCK theory to the family of 
curves $\Sigma(x)$. Since solutions 
to \linsys\ for physical asymptotics have nontrivial 
stokes matrices
the expansion of the Baker-Akhiezer 
function in $1/\lambda^{1/2}$ (or, equivalently, the 
expansion in $\hbar$) is only asymptotic while in standard
kdv theory it is important that the expansion be convergent.
This technicality also means that the potential 
$u(x;t_j)$ and the associated $\tau$ function
is outside the class of functions $\CC$ \segal\ for which 
the Segal-Wilson theory is valid.
Nevertheless, the interpretation of the 
tau function of 2D gravity in terms of star operators, 
while formal, suggests that the Grassmannian 
picture of KdV flows should carry over more or 
less unchanged. Defining precisely the
enlarged Grassmannian required to incorporate the
new class of KdV flows encountered in 2D gravity 
is an interesting open problem.

\bigskip
\bigskip
\centerline{\bf Acknowledgements}

In addition to those acknowledged in \geom\ I would 
like to thank L. Alvarez-Gaum\'e, T. Banks, F. David,
C. Gomez, C. Itzykson,
B. McCoy, E. Martinec, T. Miwa, H. Neuberger,
G. Segal, S. Shatashvili, S. Shenker, M. Staudacher, 
E. Verlinde, and E. Witten
for conversations. I am also grateful to 
the organizers of the 
Yukawa International Seminar in Kyoto, the
Carg\`ese Workshop on Random Surfaces and 2D Gravity,
and the Trieste conference on 
Topological Methods in Quantum Field Theories for the 
opportunity to present this material.
I thank the Rutgers Dept. of Physics for hospitality 
while this paper was completed.
This work was supported by DOE grants DE-AC02-76ER03075, 
and DE-FG05-90ER40559,
and by a Presidential Young Investigator Award.

\appendix{A}{The double scaling limit of Hermite functions}

In the case of a gaussian matrix potential $e^{-N~ tr\phi^2}$
the orthonormal wavefunctions are simply
\eqn\herfun{
p_n(\lambda)={N^{1/4}\over 2^{n/2}\pi^{1/4}\sqrt{n!}} H_n(\sqrt{N}\lambda)
e^{-N\lambda^2/2} }
where $H_n$ is a Hermite polynomial, and has the integral 
representation
\eqn\intrep{
H_n(x)= {2^n\over \sqrt{\pi}}\int_{-\infty}^{\infty}(x+it)^ne^{-t^2}dt}
If we let $n/N=1-a^2 z$, then we can estimate the 
integral by stationary phase approximation. 
At $\lambda=\lambda_c=\sqrt{2}$
the two saddle-points coalesce, and, expanding about the 
saddle point to third order we find
\eqn\herdsl{
p_n\bigl(\sqrt{2}(1+a^2\lambda)\bigr)
\sim a^{-1/2}\int_{-\infty}^{\infty} 
e^{-it(\lambda +z)-it^3/3} dt}
up to numerical constants. Thus the double scaling limit 
of the Hermite functions are
Airy functions, which satisfy \linsys\ for the 
case $\ell=0$. Using \herdsl\ one may easily reproduce 
the result of Br\'ezin and Kazakov for the double scaling 
limit of the gaussian model resolvent \BK\ .

\listrefs
\bye



