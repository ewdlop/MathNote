\input harvmac.tex
\def\CP {{\cal P }}
\def\CL {{\cal L}}
\def\CV {{\cal V}}
\def\p {\partial}
\def\CS {{\cal S}}
\def\hb {\hbar}
\def\inbar{\,\vrule height1.5ex width.4pt depth0pt}
\def\IB{\relax{\rm I\kern-.18em B}}
\def\IC{\relax\hbox{$\inbar\kern-.3em{\rm C}$}}
\def\IP{\relax{\rm I\kern-.18em P}}
\def\IR{\relax{\rm I\kern-.18em R}}


\Title{\vbox{\baselineskip12pt\hbox{YCTP-P17-90}\hbox{}}}
{\vbox{\centerline{Matrix Models of 2D Gravity}
\centerline{}
\centerline{and}
\centerline{}
\centerline{Isomonodromic Deformation$^1$}}}

\centerline{Gregory Moore}

\bigskip{\baselineskip14pt
\centerline{Department of Physics}\centerline{Yale University}
\centerline{New Haven, CT 06511-8167}}
\bigskip
\bigskip
\bigskip
\noindent
We review the relation between the string equations of 
the matrix model approach to 2D quantum gravity and the 
method of isomonodromic deformation. 

\footnote{}{$^1$ Based on lectures given at the workshop of the 
Yukawa International Seminar, Kyoto, 10-19 May 1990, The 
Carg\`ese Workshop on Random Surfaces and 2D Gravity,
May 28-June 1 1990, and the Trieste conference on 
``Topological Methods in Quantum Field Theories,'' 11-15 June 1990}

\draft

\newsec{Introduction} 

One of the principle goals of the theory of 
2D gravity is making sense of the formal expression
\eqn\formal{
Z(\mu,\kappa;T_i)=\sum_h\int_{MET_h}dg e^{\mu\int\sqrt{g}+\kappa\int R}
Z_{QFT(T_i)}[g] }
where we integrate over metrics $g$ on surfaces with $h$ handles 
with a weight defined by the Einstein-Hilbert action 
($\mu$ is the cosmological constant and $\kappa$ is Newton's 
constant, or, equivalently, the string coupling) together 
with the partition function of some 2D quantum field theory,
$QFT(T_i)$. The parameters $T_i$ should be thought of as 
coordinates on a subspace of the space of 2D field theories, 
or, equivalently, coordinates for a space of string backgrounds.

The expression \formal\ is of course extremely formal, especially when 
one includes the sum over topologies.
Nevertheless, in the physical literature, rigorous definitions of 
this expression can be proposed using the methods of 
random matrix theory. Last year these concrete definitions 
led to a beautiful result for \formal\ for an appropriate set 
of quantum field theories
\nref\BK{E. Br\'ezin and V. Kazakov, ``Exactly solvable field
theories of closed strings,'' Phys. Lett. {\bf B236}(1990)144.}%
\nref\DS{M. Douglas and S. Shenker, ``Strings in less than one
dimension,'' Rutgers preprint RU-89-34.}%
\nref\GM{D. Gross and A. Migdal, ``Nonperturbative two dimensional
quantum gravity,'' Phys. Rev. Lett. {\bf 64}(1990)127.}%
\nref\GMi{D. Gross and A. Migdal, ``A nonperturbative treatment of
two-dimensional quantum gravity,'' Princeton preprint PUPT-1159(1989).}
\nref\newD{M. Douglas, ``Strings in less than one dimension
and the generalized KdV hierarchies,''  Rutgers preprint RU-89-51.}
\nref\bdss{T. Banks, M. Douglas, N. Seiberg, and S. Shenker,
``Microscopic and macroscopic loops in non-perturbative two dimensional
gravity,'' Rutgers preprint RU-89-50.}\refs{\BK{--}\bdss}. 
The answer depends only on a particular 
combination $x=x(\mu,\lambda)$ and is most simply expressed as 
a differential equation satisfied by 
\eqn\spfcht{u(x;T_i)={\p^2\over\p x^2} Z .}
The differential equation 
satisfied by $u$ for
coupling to the $(p,q)$ 
minimal conformal field theory
\ref\bpz{Belavin, Polyakov, Zamalodchikov, Nucl. Phys.}\ 
was most elegantly 
formulated by M. Douglas in \newD\ in terms of a differential 
operator $L=D^q+u_{q-2}(x)D^{q-2}+\cdots u_0(x)$ where 
$D=d/dx$ and $u_{q-2}$ is identified with the ``specific heat'' 
$u(x)$ of \spfcht . The system of equations:
\eqn\dgle{[L^{p/q}_+,L]=1 }
where the subscript $+$ indicates the differential operator part of 
a pseudodifferential operator. The equations for massive 
models interpolating between the minimal models have the form
$\sum_p T_p [L^{p/q}_+,L]=1$. 

The specific case of the $(2p-1,2)$ models (corresponding 
to single hermitian matrix models) have been most intensively
studied. In this case we have simply $L=D^2+u(x)$ so that 
hierararchy of kdv equations $\dot L=[L^{(2p-1)/2}_+,L]$
may be written as 
\eqn\kdv{{\p u\over \p t}={\p\over\p x}R_p [u(x)] }
where the conserved densities $R_p$ of the kdv flow are 
known as ``Gelfand-Dickii'' potentials in the 2d gravity 
literature. The string equations then take the perhaps 
more familiar form
\eqn\onemat{\sum_j (j+\half)T_j R_j[u(x)]=x}

Several interesting properties of these equations were
discovered shortly after the initial formulation of the 
double scaling limit. One of the more remarkable 
results is that a solution to \onemat\ , as a function of the 
$T_j$ should satisfy kdv flow in the $T_j$ \bdss\ .
The original argument for this was proposed from the 
physical point of view using the matrix model integral, 
which makes certain implicit assumptions about 
boundary conditions. Later the issue of proper
boundary conditions for physically acceptable solutions 
of \onemat\ were clarified, again from the matrix model
point of view, by Br\'ezin, Marinari, and Parisi
\ref\BMP{E. Brezin, E. Marinari, and G. Parisi, ``A Non-Perturbative 
Ambiguity Free Solution of a String Model,'' ROM2F-90-09}.
These authors argued, heuristically, 
that the existence of physically reasonable solutions 
depends on the parity of $m$, the largest index for 
which $T_m\not=0$. In particular, they proposed that 
physically reasonable solutions exist only for $m$ odd, 
and in this case the asymptotics fixes a unique solution of 
\onemat\ which is pole-free on the real axis. 
They backed up their arguments with a numerical solution 
for the case $m=3$. This development
led to the very interesting papers
\ref\dss{M. Douglas, N. Seiberg, and S. Shenker, 
``Flow and instability in quantum gravity,'' Rutgers preprint, 
RU-90-19}
\ref\bahnot{G. Bahnot, G. Mandal and O. Narayan,
``Phase transitions in 1-matrixc models,'' IAS preprint
IASSNS-HEP-90/52}\ in which it was demonstrated, again numerically, 
that one cannot use kdv flow to define ``pure gravity'' 
(an $m=2$ solution) by flowing from the well-defined 
$m=3$ solution. Essentially, the solution develops a 
shock wave and there is no well-defined
$T_2\to \infty$ limit.

This paper is a review and continuation of 
\ref\geom{G. Moore, ``Geometry of the string equations,'' 
Yale preprint YCTP-P4-90}\ where  
the formalism of isomonodromic deformation
\nref\Its{A. Its and V. Yu. Novokshenov, {\it The Isomonodromic
Deformation Method in the Theory of Painlev\'e Equations,}
Springer Lect. Notes Math. 1191.}\ and some related ideas 
were applied to the string equations. 
The purpose of our paper \geom\ was threefold.
First, the isomonodromic deformation formalism is well
suited to proving rigorously the statements regarding 
the properties of kdv flow and existence and uniqueness of
solutions mentioned above. In particular, in section ???
below we review how this formalism can be used to prove 
these properties. The formalism applies equally well to the 
string equations associated with unitary-matrix models and 
with double-cut phases of the hermitian matrix models so 
we outline the equations for this case too.
Second, through the work of the 
Kyoto school
\nref\Jimboi{M. Jimbo, T. Miwa, K. Ueno, ``Monodromy Preserving 
Deformation of Linear Ordinary Differential Equations with Rational
Coefficients,'' Physica {\bf 2D}(1981)306.}
\nref\Jimboii{M. Jimbo and T. Miwa, ``Monodromy Preserving 
Deformation of Linear Ordinary Differential Equations with Rational
Coefficients. II,'' Physica {\bf 2D}(1981)407.}
\nref\Jimboiii{M. Sato, T. Miwa, and M. Jimbo, ``Aspects of Holonomic
Quantum Fields Isomonodromic Deformation and Ising Model,''
in {\it Complex Analysis, Microlocal Calculus and Relativisitic
Quantum Theory}, D. Iagolnitzer, ed., Lecture Notes in Physics 126}
\nref\Jimboiv{M. Jimbo, ``Introduction to Holonomic Quantum 
Fields for Mathematicians,'' Proc. Symp. in Pure Math. {\bf 49}(1989)part
I. 379.}
\refs{\Jimboi{--}\Jimboiv}\ it is known that the isomonodromic
deformation formalism is closely related to the quantum field 
theory of free fermions in two-dimensional spacetime. This is 
extremely suggestive since hermitian matrix models can also
be written in terms of free fermions. It is worth having a good
understanding of any connection between these fermions since 
it might be an indication of a deep connection between nonperturbative
2D gravity and ``quantum field theory on the spectral curve.''
These matters are discussed in section three below.
Third, and more philosophically, interesting physics is usually
related to interesting geometry, and this certainly ought to 
be the case for nonperturbative quantum gravity. We would like 
to know the underlying geometrical significance of the string 
equations. By this we do {\it not} mean the geometrical 
meaning of the terms in the asymptotic expansion of solutions
to the string equations, for these have already been 
adequately understood from the point of view of topological
field theory
\ref\witt{E. Witten, ``On the structure of the topological phase of 
two dimensional gravity,'' preprint IASSNS-HEP-89/66}
\ref\distler{J. Distler, ``2D quantum gravity, topological
field theory and multicritical matrix models,'' princeton 
preprint PUPT-1161}
\ref\DiWit{R. Dijkgraaf and E. Witten, ``Mean Field Theory, Topological 
Field Theory, and Multi-Matrix Models,'' IASSNS-HEP-90/18;PUPT-1166}
\ref\newvsq{E. Verlinde and H. Verlinde, ``A Solution of two 
dimensional topological quantum gravity,'' preprint IASSNS-HEP-90/40}
\ref\dvv{Dijkgraaf, Verlinde, Verlinde}.
Some proposals for a geometrical meaning of the string
equations can be found in section four below.

Closely related matters have been discussed in many recent papers.
Of these we draw particular attention to 
\ref\emil{E. Martinec, unpublished}
\ref\witrev{E. Witten, ``Two dimensional gravity and
intersection theory on moduli space,'' IAS preprint,
IASSNS-HEP-90/45}\ 
\ref\morozovi{A. Gerasimov, A. Marshakov, A. Mironov, 
A. Morozov, and A. Orlov, ``Matrix Models of 2D gravity and 
Toda Theory,'' P.N. Lebedev Institute preprint, July 1990}
\ref\morozovii{A. Mironov and A. Morozov, ``On the origin of 
virasoro constraints in matrix models: lagrangian approach,''
P.N. Lebedev Institute preprint, July 1990}
where, among other things,
the matrix models were shown to be equivalent to the 
infinite Toda chain - an important integrable system- 
even {\it before} taking the continuum limit.

\newsec{Matrix Models and 2D Field Theory}

In this section we outline how a 2D field theory ``on 
the spectral curve'' may be seen to emerge from the 
matrix model integral. Our understanding of this phenomenon
is woefully incomplete, but we believe some
key features may be seen already at this stage. 
Most importantly we will 
see that the spectral parameter of inverse scattering 
theory is identified with the eigenvalue coordinate
of the random matrix path integral. This point of
view has also been emphasized in \morozovi\morozovii\ 
(although some details are different).

\subsec{The level-spacing problem}

The clearest example of the phenomenon we are discussing 
can be seen already in the large $N$ limit of hermitian 
matrix models. Consider the matrix model:
\eqn\gauss{\eqalign{
Z_N&=\int d^{N^2}\phi e^{- N  tr V(\phi)}\cr
&=\int \prod d\lambda_i \Delta^2 e^{-N \sum_i V(\lambda_i)}\cr} }
where $V(\lambda)$ is a polynomial in $\lambda$. 
and consider furthermore the probability $\tau(I;N)$
that no eigenvalue falls in the range $I=[\lambda_1,\lambda_2]$.
It was shown in 
\ref\dnsmt{M. Jimbo, T. Miwa, Y. Mori and M. Sato, 
``Density Matrix of an Impenetrable Bose Gas and the 
Fifth Painlev\'e Transcendent'' Physica {\bf 1D} (1980)80}
that $\tau(a_1-a_2)=\lim_{N\to \infty}
\tau([a_1/N,a_2/N];N)$ is the partition function of 
a quantum field theory of free fermions (and is, moreover
the tau function for the isomonodromy problem related to 
the Painlev\'e V equation.) Since the context in which 
this was originally understood is (superficially) removed
from matrix models we will show how this may be understood 
from the matrix model point of view. 

We begin by explaining the origin of the quantum field theory.
Correlation functions with the measure
\gauss\  can be interpreted 
\ref\mehta{M. L. Mehta, {\it Random Matrices} Academic Press,1967. }
\ref\itzdr{See sec. 10.3 in C. Itzykson and J.-M. Drouffe,
{\it Statistical Field Theory}, vol. 2, Cambridge Univ. Press. 1989}
\bdss\ as expectation values in a slater determinant of 
fermion one-body wavefunctions given by orthonormal functions:
%
%$$\psi_j(\lambda)= \bigl({N\over 2\pi}\bigr)^{1/4}
%\bigl({N\over j!}\bigr)^{1/2}(\lambda^j+\cdots)e^{-N\lambda^2/ 4} $$
%
$$\psi_j(\lambda)= P_n(\lambda)e^{-{N\over 2}V(\lambda)} $$
where $P_n$ are orthonormal polynomials for the measure 
$d\lambda e^{-NV(\lambda)}$.
As in \bdss\ we may pass to second quantized wavefunctions:
\eqn\sec{
\eqalign{\psi(\lambda)&=\sum_{n=1}^\infty \psi_n(\lambda)a_n\cr
\psi(\lambda)^\dagger&=\sum_{n=1}^\infty \psi_n(\lambda)a_n^\dagger\cr
\{\psi^\dagger(\lambda),\psi(\lambda')\}&=\delta(\lambda-\lambda')\cr} 
}
where the ground state is the Fermi sea with the first 
$N$ levels filled.
As argued in \bdss\ the main contributions to correlation functions 
come from the neighborhood of the Fermi level. 
For even potentials
we can rewrite the recursion relation for orthogonal
polynomials \BK\DS\GM\ in the form:
\eqn\newrec{
\eqalign{
\lambda p_{2n}(\lambda)&=\sqrt{r_{2n+1}}p_{2n+1} +\sqrt{r_{2n}}p_{2n-1}\cr
\lambda p_{2n+1}(\lambda)&=\sqrt{r_{2n+2}}p_{2n+2} 
+\sqrt{r_{2n+1}}p_{2n}
\cr}
}
By evaluating these at $\lambda=0$ we see that quite generally 
if we expect a continuum limit for the orthonormal wavefunctions
themselves in the neighborhood of $\lambda=0$ we should define 
\eqn\sclwv{
p_{2n+1}({\lambda\over N})=(-1)^nf_1(x,\lambda)
\qquad
p_{2n}({\lambda\over N})=(-1)^nf_2(x,\lambda)
}
where $x=n/N$. Assuming $r_n$ has an expansion of the 
form $r_n=r(x)+\epsilon^2r_1(x)+\cdots$ where $\epsilon=1/N$
we find that \newrec\ implies
\eqn\sclwvi{
\eqalign{
f_1&=(r(x))^{-1/4}sin\biggl[\lambda\int^x {dx'\over \sqrt{r(x')}}\biggr]\cr
f_2&=(r(x))^{-1/4}cos\biggl[\lambda\int^x {dx'\over \sqrt{r(x')}}\biggr]\cr
} }
Since the dominant contributions of physical quantities 
come from the neighborhood of the Fermi level ($x=1$) we see 
that quite generally the orthonormal wavefunctions become 
sines and cosines. THese arguments can be checked explicitely
using hermite functions in the case of a gaussian measure.

From the behavior of the 1-body wavefunctions we see that
$\hat \psi(\gamma;N)\equiv {1\over \sqrt{N}}\psi(\gamma/N)$
has a smooth large $N$ limit.
For example, 
for the two-point function one easily verifies
the exact formula:
\eqn\extpt{
\langle N|\psi^\dagger(\lambda_1)\psi(\lambda_2)| N\rangle
=\sqrt{r_{N+1}}{\psi_{N+1}(\lambda_1)\psi_N(\lambda_2)-
\psi_{N+1}(\lambda_2)\psi_N(\lambda_1)\over
\lambda_1-\lambda_2}
} 
so that we have
\eqn\limkernel{
\langle N|\hat\psi^\dagger(\gamma)\hat\psi(\gamma')| N \rangle 
\rightarrow {1\over  \pi} {sin (\gamma-\gamma')\over
\gamma-\gamma'} \qquad .}
On the operator level we define $p=(n-2N)/2N$ and 
\eqn\osc{\eqalign{
a_1(p)&=\sqrt{N}(-1)^{n+N}\bigl(\hat a_{2n}- i\hat a_{2n+1}\bigr)\cr
a_2(p)&=\sqrt{N}(-1)^{n+N}\bigl(\hat a_{2n}+ i\hat a_{2n+1}\bigr)\cr} }
where $\hat a_n\equiv a_{N+n}$. The sum over $n$ becomes an
integral $\int^{\infty}_{-1}dp$. 
Again, assuming that the main contributions
come from the neighborhood of the Fermi level we extend this to 
an integral over the entire $p$ axis.
Our main claim is therefore that $\hat \psi$ has a good large $N$
limit and is given by 
\eqn\lmfn{\eqalign{
\hat\psi(\gamma)&=
e^{i\gamma}\int^\infty_{-\infty}
dp~ a_1(p)e^{i\gamma p}
+e^{-i\gamma}\int^\infty_{-\infty}
dp~ a_2(p)e^{-i\gamma p}\cr
&=e^{i\gamma}\psi_1(\gamma)+e^{-i\gamma}\psi_2(-\gamma)\cr}
 }
and that the Fermi sea becomes the ground state 
defined by 
$a_i(p)|0\rangle=0$ for $p>0$ and $a_i^\dagger(-p)|0\rangle=0$
for $p>0$. 

Now let us return to the problem of the level spacing.
In terms of the orthogonal polynomials $\psi_j$ 
we have \mehta\itzdr
\eqn\level{\eqalign{
\tau(I;N) &=det\biggl[\delta_{j,k}-\int_{\lambda_1}^{\lambda_2}d\lambda
\psi_j(\lambda)\psi_k(\lambda)\biggr]_{0\leq j,k\leq N-1} \cr
&=\langle N|exp\bigl(-\int_{\lambda_1}^{\lambda_2}
\hat\psi^\dagger(\lambda)
\hat\psi(\lambda) d\lambda \bigr)|N\rangle\cr}\qquad .
}
In the $N\to\infty$ limit, taking $\lambda_i=a_i/N,$ 
and $I=[a_1,a_2]$ we obtain:
$$\tau(I)=\langle 0|:exp\biggl(
-\int_I\hat\psi^\dagger
\hat\psi\biggr) :|0\rangle$$
where the normal-ordered exponential is 
evaluated by expanding in power series and point-splitting 
all the integrals so that the delta functions in the 
two-point correlation functions don't contribute.
This is simply a correlation function in the 
theory of the one-dimensional Bose gas, also known as the 
nonlinear Schr\"odinger theory, in the completely 
impenetrable case,  as studied in  
\dnsmt
\ref\Korep{A.R. Its, A.G. Izergin, V.E. Korepin, and N.A. Slavnov,
``Differential equations for quantum correlation functions,'' 
preprint}
\ref\Korepi{A.R. Its, A.G. Izergin, and V.E. Korepin, ``Temperature
correlators of the impenetrable bose gas as an integrable system,''
ICTP preprint IC/89/120}. 
In particular,
$\tau(I)$ is simply the Fredholm determinant 
$det~(1-K)$ where $K$ is the kernel defined by 
\limkernel\ on the interval $I$.
and the integral operator 
$I+K$ is in the infinite dimensional group of ``completely integrable
kernels'' described in \Korep\ . 
Following the general procedure described in \dnsmt\Korep\Korepi\ 
we define 
$\chi_1(z)=e^{iz}\psi_1(z)$ and $\chi_2(z)=e^{-iz}
\psi_2(-z)$ so that $\hat\psi=\chi_1+\chi_2$ and $\psi^\dagger=
\chi_1^\dagger + \chi_2^\dagger$, and consider the ``Baker-Akhiezer 
framing'':
\eqn\levii{
\Psi_{\alpha\beta}(\lambda,\lambda')={\langle 0|\chi_\alpha^\dagger(\lambda)
exp(-\int_I\chi^\dagger \chi)\chi_\beta(\lambda')|0\rangle\over
\langle 0|exp(-\int_I\chi^\dagger \chi)|0\rangle} 
\qquad .}
As we will see below this matrix satisfies a linear ODE in 
$\lambda$ reminiscent of the Knizhnik-Zamalodchikov equation 
whose study allows us to derive information on the 
determinant $\tau(I)$. 

\subsec{Multi-Cut Solutions}

We now turn to an example in which we take a double
scaling limit. We consider a special class of 
multicritical potentials
\ref\molinari{Molinari} 
\ref\crmr{C. Crnkovic and G. Moore,``Multi-Critical Multi-Cut
Matrix Models,'' Yale preprint YCTP-??-90}
\eqn\mulpt{V'_m(\lambda)=k(m)\lambda^{2m+1}\bigl(1-{1\over \lambda^2}
\bigr)^{1/2}|_+ }
where the subscript $+$ means we keep the polynomial part in an
expansion about infinity and 
\eqn\const{k(m)=2^{2m+1}{(m+1)!(m-1)!\over (2m-1)!} \qquad .}
The potential
$V_m(\lambda)$ has the shape of a double well, and the bump
at $\lambda=0$ becomes progressively flatter as $m$ increases.
These potentials are characterized by the property that the eigenvalue 
distribution at tree level consists of two separate 
cuts, which, at the critical point, meet at the origin.
These theories have very interesting phase transitions investigated
in \molinari\dss\crmr\ .

We now consider the double scaling limit and the recursion 
relations for these models. We denote orthonormal polynomials 
by $\CP_k$ and we have the standard relation 
\eqn\reci{\lambda \CP_n=\sqrt{R_{n+1}}\CP_{n+1}+\sqrt{R_n}\CP_{n-1} }
As in 
\ref\molinari{molinari}
\dss\ we assume 
that $R_{2n}$ and $R_{2n+1}$ have different scaling limits:
\eqn\scli{\eqalign{R_{2n}&=r_c+a^{1/m}f(z) + a^{2/m} g(z)+\cdots\cr
R_{2n+1}&=r_c-a^{1/m}f(z)+a^{2/m}g(z)+\cdots\cr} }
where, in the standard way, $x={2n\over N}=1-a^2 (z-z_0)$ and 
$Na^{2+1/m}=1$. The tree-level string equation 
is easily shown to be \crmr\
\eqn\tree{f^{2m}= {-z\over 2^{2m-1} (m+1)} }
Since $V$ is even we can group the orthogonal 
polynomials into even and odd classes and the operation 
of multiplication by $\lambda$ maps even orthogonal polynomials
to odd polynomials and vice versa. By studying the recursion 
relation \reci\ in the neighborhood of $\lambda\cong 0$ we
expect that the orthogonal polynomials 
\eqn\lmply{\eqalign{
f^+(z,\lambda)&\equiv j(a,\lambda)(-1)^k\CP_{2k}(a^{1/m}\lambda)\cr
f^-(z,\lambda)&\equiv j(a,\lambda)(-1)^k\CP_{2k+1}(a^{1/m}\lambda)\cr} }
will have smooth limits, where we have allowed for a possible 
``wavefunction renormalization'' $j$ which is $z$-independent.

Thus, following the reasoning before we now consider a doublet 
of second quantized Fermi fields....etc.

The double-scaling limit of \reci\ becomes
\eqn\reciii{\lambda \vec \psi=\sqrt{8}\biggl(-i\sigma_2
{d\over dz}-\sigma_1 {f\over 4}\biggr)\vec \psi}
where $\vec\psi=(f^+(z,\lambda),f^-(z,\lambda) )$. We now consider
the double scaling limit of the full recursion relation
\eqn\reciv{{d\over d\lambda}\vec\CP =
N\bigl(V'(\lambda)_+\bigr) \vec\CP }
where the subscript $+$ indicates that we keep only the 
upper triangular part of the matrix representation of the 
operator $V'(\lambda)$ in the basis of orthonormal polynomials. 
In the double 
scaling limit, with an appropriate choice of $j$ and a
slight redefinition of $\vec \psi$, the recursion relations 
become
\eqn\laxi{
\CL\vec\psi\equiv
\biggl({d\over dz}+\sigma_3 \lambda+f \sigma_1\biggr)\vec\psi=0}
\eqn\laxii{
\biggl({d\over d\lambda}-M_m(\lambda,f)\biggr)\vec\psi=0
}
where $M_m$ is polynomial in $\lambda$ and differential polynomial
in $f$. We may find $M_m$ explicitly as follows.
Note that the commutator of $M_m$ with $\CL$ is just $\sigma_3$.
In integrable systems theory the method of 
Zakharov and Shabat  determines
the space of matrices $M$ which are polynomial in $\lambda$ 
and differential polynomial in $f$ and whose commutator with 
$\CL$ is $\lambda$-independent. The vector
space of such matrices is spanned by the matrices
occuring in the Lax pairs for the modified KDV hierarchy. 
(See, e.g., 
\ref\DrS{Drinfeld and Sokolov, ``Equations of Korteweg-de Vries 
type and simple Lie algebras,''
Sov. Jour. Math. (1985)1975, section 3.8}\ .) 

The flatness condition following from \laxi\ and\laxii\ gives an 
ordinary differential equation for $f$. Comparing with 
the result \tree\ derived at tree level we find 
\eqn\res{M_l=(2l+1) \biggl[\bigl
(U_l+(V_l'/2\zeta)\bigr)\sigma_3
+V_l\sigma_1 -(V_l'/2\zeta)i\sigma_2\biggr]\Psi =0}
where
\eqn\uets{\eqalign{
U_l&\equiv R_l + \zeta^2 R_{l-1}+\cdots \zeta^{2l}R_0 +{z\over 2l+1}\cr
V_l&\equiv \zeta S_{l-1} +\zeta^3 S_{l-2}+\cdots \zeta^{2l-1} S_0\cr
S_l&\equiv f R_l-\half R_l'\cr} 
}
and the KDV potentials are evaluated for $u=f^2 + f'$. The compatibility 
conditions become $S_m+{z\over 2m+1} f=0$, which is 
the equation found in 
\ref\peri{V. Periwal and D. Shevitz, ``Unitary-Matrix Models 
as Exactly Solvable String Theories,'' Phys. Rev. Lett. 
{\bf 64}(1990)1326.}
for unitary-matrix ensembles. 
This differential equation is the equation for a 
self-similar solution to the MKdV flow
\eqn\mckf{
{\p f\over \p t_l}={\p \over \p x}S_l[f]
}
that is, if $\psi(x,t)=t^\alpha f(z)$
where $z=t^\alpha x$ and $\alpha=-1/(2l+1)$. 
then $f$ must satisfy the above ODE. 

More generally, by adding the multicritical potentials 
$V=\sum_\ell t_\ell a^{(2m-2\ell)/m} V_\ell$ 
we simply replace
\eqn\repl{M_m\rightarrow \sum_\ell t_\ell M_\ell}
which yields the more general string equations:
\eqn\msvmd{\sum t_l S_l +{1\over 2m+1} z f=0 }
By arguments analogous to those above we find that 
the orthogonal polynomials must also satisfy linear equations 
in the variables $t_\ell$:

WRITE OUT LAX FOR MKDV

\subsec{Single-Cut solutions}

The double scaling limit was originally defined for a 
different universality class of phase transitions.
In these cases the critical behavior arises from the 
integration near a nonzero value of $\lambda=\lambda_c$, 
and for the $m^{th}$ multicritical point we have the scaling behavior
$R_n\to r_c + a^{2/m} u(z)$ for $n/N=1-a^2 (z-z_0)$, with
$Na^{2+1/m}$ fixed.
In this
case it is natural to assume that the orthogonal polynomials 
have a limit of the kind:
$$p_n(\lambda_c+a^{2/m}\lambda)\to j(a,\lambda)p(z,\lambda)$$
where the wavefunction renormalization will be determined below.
The limit of the recursion relation becomes
\eqn\schrod{\bigl({d^2\over dz^2} + u(z)\bigr)p(z,\lambda)=\lambda 
p(z,\lambda) 
}
showing that the orthogonal polynomials define a Baker-Akhiezer 
function. Similarly, in the limit we have the quantum 
field
$$a^{1/m}\psi(\lambda_c+a^{2/m}\lambda)\rightarrow 
\hat \psi(\lambda)=\int dz a(z) p(z,\lambda)$$
and the two point function is simply 
$$\langle z_0|\hat \psi^\dagger(\lambda_1)
\hat\psi(\lambda_2)|z_0\rangle={p'(z_0,\lambda_1)p(z_0,\lambda_2)-
p'(z_0,\lambda_2)p(z_0,\lambda_1)\over \lambda_1-\lambda_2}$$
If we send one field to infinity we obtain simply the 
Baker function, reminiscent of the situation in conformal 
field theory.
 
Moreover, if we perturb around the multicritical point we 
add the potential
\eqn\prtrb{
\delta V= N\sum t_l (\lambda-\lambda_c)^{l+1/2} a^{(2m-2l)/ m}}
and the partition function now becomes
\eqn\newtpt{
\langle 0|e^{\int \psi^\dagger(\lambda)\psi(\lambda)
\sum t_l^{l+1/2}\lambda^{l+1/2} }|0\rangle
}
Thus, as a function of the $t_l$ the partition function 
is simply the Fredholm determinant $det(1-K)$ for the 
kernel defined by 
\eqn\kerna{\eqalign{
K(\lambda_1,\lambda_2)&=\sqrt{\vartheta(\lambda_1)}K_0(\lambda_1,\lambda_2)
\sqrt{\vartheta(\lambda_2)}\cr
K_0(\lambda_1,\lambda_2)&={p'(z,\lambda_1)p(z,\lambda_2)-
p'(z,\lambda_2)p(z,\lambda_1)\over \lambda_1-\lambda_2}\cr
\vartheta(\lambda)&=\sum t_l \lambda^{l+1/2}\cr}
}
Again, $1+K$ is
in the class of completely integrable integral operators
\Korep (in this case the convergence of the integrals is 
rather delicate. the only case we can analyze is the ``topological''
point $m=1$ in which case the relevent integrals are 
conditionally convergent). 
Following the procedure of \dnsmt\Korep\Korepi\ as before
we consider the linear matrix equation satisfied
by $\vec\psi=(p'(x,\lambda)~p(x,\lambda))$:
\eqn\lopp{
\CL\vec\psi\equiv -{d\over dx} + \pmatrix{0&\lambda+u\cr
1&0\cr}\vec\psi=0 \qquad .}
On the other hand, 
from their origin in the matrix model it is clear that 
${d\over d\lambda}\vec\psi$ may be expressed as a matrix operator 
on $\vec\psi$ which is  polynomial in $\lambda$ with $x$-dependent
coefficients as in \laxii\ . As in the previous discussion 
we can determine $M_m$ by the method of 
Zakharov and Shabat \DrS\ . Recall that the $l^{th}$ 
KdV flow can be written as the compatibility condition
$[2\p/\p t_l + \CP_l,\CL]=0\qquad ,$
where the $sl(2)$ matrix 
\eqn\popp{
\CP_l\equiv \pmatrix{A_l& B_l\cr
C_l&-A_l\cr} }
may be expressed in terms of the conserved densities $R_l$ of 
KDV flow
\ref\Gelf{I.M. Gelfand and L.A. Dickii, ``Asymptotic Behavior of the 
Resolvent of Sturm-Liouville Equations and the Algebra of the 
Korteweg-De Vries Equations,'' Russian Math Surveys, {\bf 30}(1975)77.}
via
\eqn\potent{\eqalign{
C_l&=R_l+\lambda R_{l-1}+\cdots + \lambda^{l-1}R_1 + \lambda^l R_0\cr
A_l&=\half C_l'\cr
B_l&=(\lambda+u)C_l-A_l'\cr} }
It follows that if we define
\eqn\wittx{
\IP=-\hbar{d\over d\lambda}-\half\sum_j (j+\half)T_{j}\CP_{j-1}
+\pmatrix{0&\hbar x-(\sum_j (j+\half) T_jR_j)\cr
0&0\cr} }
(where $\CP_{-1}=0$) 
then the equation $[\IP_l,\CL]=0$ is equivalent to the 
string equation.
Thus the compatibility conditions for the first pair and 
last pair of linear systems
\eqn\linsys{\eqalign{
\IP \Psi(\lambda,x,T_j)&=0\cr
\CL \Psi(\lambda,x,T_j)&=0\cr
(2\hbar{d\over dT_j}+\CP_j)\Psi(\lambda,x,T_j)&=0\cr} }
give the massive $(2l-1,2)$ and KdV equations, respectively. It can be 
shown that the compatibility conditions for the first and 
third equations above follow from the string and kdv 
equations. 
Similar considerations allow us to write the 
string equations for arbitrary $(p,q)$ in first order
form.

\subsec{Relation to other work}

The emergence of a quantum field theory on the 
spectral curve has been discovered in several different
guises in recent investigations into matrix models.
In 
\ref\das{S.R. Das and A. Jevicki, ``String field 
theory and physical interpretation of d=1 strings,'' 
Brown preprint BROWN-HET-750}
\ref\sengupt{A.M. Sengupta and S.R. Wadia, ``Excitations
and interactions in d=1 string theory,'' Tata preprint}\ 
it is shown that one may derive a quantum field theoretical 
representation of the $d=1$ matrix model involving 
fields $\phi(t,\lambda)$ where $t$ is the 1-dimensional 
time and $\lambda$ is the eigenvalue coordinate of the 
string. 
This should be regarded as the $d=1$ version 
of the phenomena discussed for $d<1$ in this paper. 
It would be extremely interesting to related the 
$d=1$ theories to the $d<1$ theories by some analog of the 
Feigin-Fuks construction. 
polchinski?

There has also been a good deal of fuss over the 
discovery that the partition function of the matrix 
model is annihilated by operators forming a 
subalgebra of the Virasoro algebra
\ref\fukuma{Fukuma, Kawai, ???}
\dvv\morozovi\morozovii\ . We will see later that the 
2D quantum field theory on the spectral curve is related to 
a 2D conformal field theory, thus making the appearance of 
a Virasoro algebra completely natural
\foot{Essentially the same remark was made in a pretty
paper by A. Mironov and A. Morozov \morozovii\ .}.

It would be extremely interesting to understand 
the origin of the field theory on the spectral curve 
in the sum over topologies of conformal-field-theoretic
correlators.
Some curious speculations along these lines are 
described in 
\ref\knizhnik{V.G. Knizhnik, ``Multiloop amplitudes in 
the theory of quantum strings and complex geometry,'' 
Usp. Fiz. Nauk. {\bf 159}(1989)401. English translation:
Sov. Phys. Usp. {\bf 32}(1989)945, section 12}
\morozovi\ .

In this context 
it is interesting to note that many integrable massive field 
theories have correlation functions related to Painlev\'e 
equations
\ref\mcoy{E. Barouch, B.M. McCoy, and T.T. Wu, Phys. Rev. Lett. {\bf
31} (1973)1409; T.T. Wu, B.M. McCoy, C.A. Tracy, and E. Barouch, 
Phys. Rev. {\bf B13}(1976)316.}
\Jimboiii\
\Korep\ .

\newsec{Isomonodromic Deformation}

In the previous section we saw the appearance of a 2D quantum field 
theory on the spectral curve, and interpreted the 
string equations and kdv flow as 
compatibility conditions for linear equations satisfied by 
correlation functions in the theory.
These linear differential equations are essentially 
flatness conditions, and very similar equations have
appeared in conformal field theory.
In conformal field theory we often meet linear differential equations, 
for example null vector equations or 
the Knizhnik-Zamalodchikov equation, which depend 
on parameters, e.g. the positions of various fields and the 
moduli of some Riemann surface, and which have interesting 
monodromy. It is quite typical that deforming the parameters, 
e.g., the moduli of the surface, leaves the monodromy 
unchanged. In the conformal field theory literature this is 
often expressed in terms of the flatness of the Friedan-Shenker
vector bundles over moduli space. This flatness condition 
is an example of isomonodromic deformation. In this section
we will see that isomonodromic deformation is the key behind
the linear systems occuring in all the matrix model 
problems discussed above.

\subsec{The Level spacing problem}

Following \dnsmt\ we show that $\Delta_I$ is simply the 
$\tau$ function for isomonodromic deformation associated with 
$gl(2)$ fermions $\psi_{1,2}$, as in the previous sections.
At this point one may easily follow all the steps in 
section seven of \dnsmt\ using the interpretation in terms of a doublet
of fermions
\foot{For readers of \dnsmt\ one can identify, for example,
$R_I^\pm(x,x';\xi)$ with the two point function of
$\hat\chi^\dagger_i(x)$ with $\hat\psi(x')$, and so on.}. 
In particular, we can compute the monodromy of 
\levii\ under analytic continuation of $x$ around $a_i$. From the 
operator product expansion we compute $d\Psi \Psi^{-1}$
where $d$ is exterior differentiation in $x,a_i$. This gives once more
Knizhnik-Zamalodchikov type equations, interpreted here as 
isomonodromic deformation equations. From the general arguments of the 
previous section we deduce that $\Delta_I$ is the tau function 
for isomonodromic deformation.

work out more details--motivates isomonodromy

\subsec{Stokes phenomenon}

The isomonodromic deformation method focuses on the monodromy
of the solutions of the equation in $\lambda$. Since our 
equation is of the form
${d\over d\lambda}\Psi=\CA(\lambda)\Psi$ where $\CA$ is polynomial
in $\lambda$ (and 
we consider $\lambda$ as a coordinate on $\IP^1$) 
it would appear that there can be no monodromy. In fact, the 
singularity at infinity induces a kind of monodromy in the 
form of Stokes phenomenon.

To put Stokes phenomenon in perspective let us consider again
the differential equation 
\eqn\lin{{d\over d\lambda}\Psi=\CA(\lambda)\Psi}
where $\CA$ is meromorphic in $\lambda$. 
Near a 
regular singular point $\lambda_0$ where $\CA$ has a pole 
we may write a formal solution to the differential equation as
\eqn\regi{\Psi(\lambda)=\hat\Psi(\lambda)e^{M~log(\lambda-\lambda_0)}}
where $\hat\Psi$ is a formal power series in $\lambda-\lambda_0$.
It is a 
nontrivial property of differential equations with regular 
singularities that $\hat \Psi$ is in fact a convergent power 
series, so we may conclude that the monodromy is $e^{2\pi iM}$. 

Stokes phenomenon appears when we try to find solutions to 
our equation and $\CA$ has a pole of order larger than one.
For simplicity we assume that 
$$\CA(\lambda)={A_{-r}\over (\lambda-\lambda_0)^{r+1}}+\cdots$$
with $A_{-r}$ diagonalizable. In this case it can be shown
\ref\Wasow{W. Wasow, {\it Asymptotic Expansions for Ordinary 
Differential Equations}, Interscience, 1965.}
that we have a formal solution 
\eqn\formi{\Psi=\hat\Psi e^{T(\lambda-\lambda_0)}} 
with 
$$T={D_{-r}\over (\lambda-\lambda_0)^r}+\cdots
{D_{-1}\over(\lambda-\lambda_0)}+M~log~(\lambda-\lambda_0)$$
where the $D_i$ are diagonal and commute with the ``formal
monodromy'' $M$. In general $\hat \Psi$ is only an asymptotic
series. It nonetheless has nontrivial analytic meaning in 
the following sense. There exist angular sectors in 
the neighborhood of the singular point as in
\fig\sectors{Stokes sectors in the neighborhood of an 
irregular singular point. The two solutions $\Psi_{1,2}$ are
asymptotic to the formal solution only in 
the indicated sectors.}
in which there exist true solutions of \lin\ which are 
asymptotic to the formal solution \formi\ . The regions 
can be found from considering the nature of the 
essential singularity $e^T$. Notice that this is 
exponentially growing and decaying in the angular 
sectors where $cos r\theta$ is positive and negative,
respectively, where $\theta$ is an angular coordinate 
around $\lambda_0$. 
Typically a true solution asymptotic to the formal 
solution can only be defined in a sector of angular
width $\pi/r$ which contains regions of both 
growth and decay. Therefore if we compare two 
such solutions which are defined on regions with 
a nontrivial overlap, as in \sectors\ they may 
differ by right-multiplication by a constant 
matrix. Such matrices are called Stokes matrices, 
and, labelling the sectors by $\Omega_k$ we obtain
a set of Stokes matrices $S_k$ associated with the 
differential equation. 

A basic example: Airy functions

According to the works
\ref\flasch{H. Flaschka and A. Newell, ``Monodromy and 
Spectrum-Preserving Deformations I,'' Comm. Math. Phys. 
{\bf 76}(1980)65.}
\Jimboi\Jimboii\Jimboiii\Jimboiv\ the 
Stokes matrices should be considered as a generalization 
of the notion of the monodromy of the differential equation.
This is not meant so suggest that a solution to the differential
equation cannot be analytically continued to a single valued
solution in an entire neighborhood of the singular point,
and this happens in many classical examples. The point is, 
such a single-valued solution has the ``wrong'' asymptotic 
behavior in all but one sector.

\subsec{Asymptotic analysis of the PII family}

We now apply the isomonodromic deformation formalism
to the compatibility conditions for the string equations.
Since the equation in $\lambda$  depends
polynomially on $\lambda$ it cannot have interesting monodromy 
in the complex plane. It does have an irregular singular 
point at infinity and, it turns out, physically interesting
solutions to \laxii\linsys\ exhibit Stokes phenomenon. 
The differential equations \laxii\linsys\
depend on parameters $x,T_j,f,f_z,f_{zz},\dots$, where, 
for the moment, we consider $f,f_z,f_{zz},\dots $ to be 
independent quantities.
In the case of 2D gravity, 
the parameters $x,T_j$ have the dual physical interpretation of
coordinates
in the space of 2D field theories and moduli of a generalization 
of a riemann surface (discussed in section 5 below). 
NEED BETTER MOTIVATION
It is 
therefore natural to ask what conditions $f,f_z,f_{zz},\dots$
must satisfy if the monodromy is to be independent of the 
moduli. We may answer this question as follows.

As we vary $z,T_j$ we obtain a family of invertible matrix
solutions $\Psi(\lambda,z,T_j)$ to 
\eqn\lasaga{{d\over d\lambda} \Psi=\sum t_{\ell} M_\ell\Psi }
so we may consider the quantity ${\p \Psi\over \p z}\Psi^{-1}$, which 
is a rational matrix in $\lambda$. The only singularities 
can be at $\lambda=\infty$, so we need only know the behavior 
of a solution to \lasaga\ in this limit. Therefore we 
perform an asymptotic solution of \lasaga\ 
by relating the matrix elements of \res\ 
to the resolvent $R(x,\lambda^2)$ (??square?)
of the Schr\"odinger operator $L=D^2+u$. The result is 
that $\Psi=\hat\Psi e^T$
where  
$$T=\half\biggl(\sum_l{2m+1\over 2l+1}t_l \lambda^{2l+1}\biggr)\sigma_3
$$
and 
$$\eqalign{
\hat\Psi&=1+{\hat \Psi_1\over \lambda}+{\hat \Psi_2\over \lambda^2}+\cdots\cr
\hat\Psi_1&={-if\over 2}\sigma_2+H\sigma_3\cr}$$
where $H'=\half f^2$. Since the Stokes data is unchanged under
deformation of $z$ we can compute $\p_z\Psi \Psi^{-1}$ 
by substituting the asymptotic expansion and we find
the condition \laxi\ . From this condition it follows that, 
as functions of $z$, $f_z$ must be the derivative of $f$ etc., 
and $f(z)$ must satisfy the string equation.
Similarly, one may 
compute $\hat\Psi{\p T\over\p t_l}\hat\Psi^{-1}mod (1/\lambda)$
to obtain the linear condtion ????. This shows that if one 
considers a solution to \msvmd\ as a function of the $t_\ell$ then,
if the stokes data are fixed, $f$ must satisfy mkdv flow.
 
\subsec{Asymptotic analysis of the PI family}

A very similar computation can be carried out for 
the equation $\IP\Psi=0$. There is one (important)
technical change.
Since the highest power of $\lambda$ does not
multiply an invertible matrix one must make a 
transformation \Jimboii\
\ref\Kapaev{A. Kapaev, ``Asymptotics of solutions of the Painlev\'e
equation of the first kind,'' Differential Equations, 
{\bf 24}(1988)1107.}
$\lambda=\zeta^2$ and
\eqn\zee{\eqalign{
\Psi(\lambda)&=\zeta^{1/2}\pmatrix{1&1\cr
1/\zeta&-1/\zeta\cr} W(\zeta)\cr
W(\zeta)&=\zeta^{-1/2}\half \pmatrix{1&\zeta\cr
1&-\zeta\cr}\Psi(\lambda)\cr} }
The leading singularity in the equation for $W$ now 
multiplies $\sigma_3$ and we may easily find
$W\sim \hat W e^{T/\hbar}$ where 
\eqn\asymp{
\eqalign{
T&=
\biggl(-{1\over 4}
\sum_{j=1} T_j \zeta^{2j+1} +\zeta x\biggr)
\sigma_3\cr
\hat W&=1+ {H_1\over \zeta}\sigma_3 +
{u\over 4\zeta^2}\sigma_1 +\CO(1/\zeta^3)\cr} }
Again,
this may be proved by relating the matrix elements in 
the differential equation to the resolvent of the Schr\"odinger
operator $L=-D^2+u$. As before, if we deform $x$ keeping the 
Stokes data fixed then we can evaluate the expression 
${\p \Psi\over \p x}\Psi^{-1}$ from its asymptotics.
(In this case one must take into account that
the equation for $W$ has a regular 
singularity at zero.)
The result is just:
\eqn\laxi{{\p\Psi\over \p x}\Psi^{-1}=\pmatrix{0&\lambda+u\cr
1&0\cr}
}
and comparing with section 3.1 we see that this means
$u(x)$ must satisfy the string equation. Similarly we
can ask for the condition that deformations in $T_j$
keep the Stokes data fixed, and we find remaining linear
equations in \linsys\ so that $u(x;T_j)$ satisfies 
KDV flow. 

We will indicate below how the physical 
asymptotic conditions on $u(x)$ as a function of $x$
fix the Stokes data uniquely. Thus we have proven that 
$u(x;T_j)$ satisfies kdv flow. The original proposal
of Banks, Douglas, Seiberg, and Shenker
\bdss\ , was based on more physical (but entirely reasonable)
arguments. The following argument for KDV flow has also been 
proposed in \emil\witrev\morozovi\ .
In the matrix model, before the continuum limit is taken,
the jacobi matrix representing multiplication by 
$\lambda$ in the space of orthogonal polynomials 
satisfies toda flow. It was proposed  some time ago
\ref\moser{J. Moser, ``Finitely many mass points on the line under
the influence of an exponential potential--an integrable system,''
in  Lecture Notes in Physics {\bf 38}, J. Moser, ed., p. 467}
that the continuum limit of toda flow should give 
(m)kdv flow.

\subsec{$\tau$ functions}

One of the beautiful results of the Kyoto school on
isomonodromic deformation was the definition of the 
$\tau$ function for isomonodromic deformation, which 
motivated the perhaps better known tau function 
of the KP hierarchy. Applying \Jimboi\Jimboii\ to 
our case we see that 
\eqn\clsdfrm{
\omega=Res_{\zeta=\infty} tr\Biggl[ 
\bigl(\hat W^{-1}{d\hat W\over d\zeta}\bigr) d T\Biggr] }
is a closed one-form on the space of deformation parameters.
Since $\omega$ is closed 
one can then define (locally, in the space of 
deformation parameters) 
the {\it tau function} via $\omega=d(log \tau)$.
We have just seen that
$dH_1/dx=u$, and substituting into the equation for $d(log\tau)$
we get $d(log\tau)=-2H_1$, hence
\eqn\freeeng{
u(x;T_j)=-{d^2\over dx^2}log \tau }
and hence the partition function of the matrix model
(whose logarithm is the partition function of 2D 
gravity) is simply the tau function.
Since the tau function for isomonodromic deformations
is known always to be holomorphic
\ref\Miwa{T. Miwa, ``Painlev\'e property of monodromy
preserving deformation equations and the analyticity of 
$\tau$ functions,'' Publ. Res. Inst. Math. Sci. {\bf 17}
(1981)703}\ 
we see that the {\it only} singularities of a solution
$u$ to the string equations are second order poles. 
This result may be easily extended to the entire hierarchy 
of $(p,q)$ equations.

If we apply the definition \clsdfrm\ to the asymptotic 
expansion for the PII family we find
\eqn\tauf{f^2=2H'={\p^2\over \p z^2}log~ \tau \qquad .}
On the other hand, computing the connected two-point function of 
$tr\phi^2$ one can show \peri\crmr\ 
\eqn\twopt{
\langle tr\phi^2~tr\phi^2\rangle_c =R_N\bigl(R_{N+1}+R_{N-1}\bigr)\to
{1\over 8} - a^{2/m} f^2 }
so that the partition function of the unitary matrix model is just the 
tau function. (WE HAVE USED $g=f^2$ WHICH WE ONLY PROVED TO TREE LEVEL
SO FAR!)


\subsec{Stokes matrices for the PI family}

In order to obtain physical solutions to the 
string equation we need to choose proper 
boundary conditions. In the isomonodromic 
deformation literature it is shown that the 
``initial conditions'' of a solution to 
the nonlinear compatibility condition (the 
string equation) may be taken to be the stokes
matrices of the associated linear problem. 
In this section we discuss the stokes data 
for the PI family for the ``physical'' solutions
to the string equations.

The issue of the correct choice of boundary conditions 
for a physically acceptable solution to the string equation
is dictated by the origin of the specific heat in 
the matrix model integral
\BMP
\ref\David{F. David, ``Phases of the large N matrix model
and non-perturbative effects in 2d gravity,'' Saclay
preprint SPhT/90-090}. 
For example, in the case of the m=3 member of the PI family,
i.e., $R_3[u(x)]=x$, the physical asymptotics are given by
$u\sim \CO(x^{1/3})$ as $x\to\pm\infty$. Since each asymptotic
condition fixes two boundary conditions we expect that these
asymptotics uniquely specify the solution.
From the numerical 
integration in \BMP\ it appears that the solution is pole-free. 

We will show that the above asymptotics 
uniquely determines the Stokes parameters in the monodromy 
problem associated to $R_3=x$. Since one can reconstruct 
a solution from the stokes data (via the * operator, see below)
it follows that the solution is completely unique.
This unique solution is real for all $x$. 

We investigate the direct monodromy problem following closely
the treatment in
\ref\Itsi{See reference \Its, especially, chapter 5.}
\Kapaev . Recall that, after the transformation \zee\
our equation can be written as
\eqn\dforz{\eqalign{
\hbar {d W\over d\zeta}&=\Biggl[(B+\zeta^2 C+\Delta_l)
\sigma_3-(B-\zeta^2 C+\Delta)i\sigma_2+(2\zeta A
-{\hbar \over 2\zeta})\sigma_1\Biggr]W\cr\cr
&=\biggl[(\zeta^{2l+2}+{u^2\over 8}\zeta^{2l-2}+\cdots)\sigma_3 +
(-{u'\over 4}\zeta^{2l+1}+\cdots)\sigma_1 
+ (-{u\over 2}\zeta^{2l}+\cdots)(-i\sigma_2)\biggr]W
\cr}  }
where 
$$\eqalign{
C\equiv -\half\sum_j (j+\half)T_{j}C_{j-1}&
\qquad \qquad \Delta=\hbar x -\sum (j+\half)T_jR_j\cr
A_l= \half C' &\qquad\qquad B=(\lambda+u)C-A'\cr}
$$
Equation \dforz\ 
has an irregular singularity of order $2l+3$ at infinity
and a regular singularity at the origin. 

We consider some general properties of the stokes 
matrices for these equations.
For the $(2l-1,2)$ string equation we will have $4l+6$ Stokes sectors
$\Omega_k$ each containing a unique ray 
$\theta={\pi\over 4l+6}(2k-1)$, $k=0,\dots 4l+5$
along which 
$cos[(2l+3)\theta]=0$, thus we may take neighborhoods of 
infinity defined by:
\eqn\sectors{
\Omega_k\equiv \{\zeta| {\pi\over 4l+6}+ {\pi\over 2l+3}(k-2)
< arg \zeta < {\pi\over 4l+6}+ {\pi\over 2l+3}k\} }
for $k=0,\dots, 4l+5$. We will consider $k$ as an integer
defined modulo $4l+6$. 
The Stokes sectors have the property that in $\Omega_k$ 
there is a unique solution $W_k$ to \dforz\ with the
asymptotic behavior $\hat W e^{T/\hbar}$. 
On the overlap $\Omega_k\cap\Omega_{k+1}$ the two solutions 
$W_{k+1}$ and $W_k$ must be 
related by Stokes matrices
\eqn\sto{W_{k+1}=W_k S_k \qquad. }
Since $W_{k+1}$ and $W_k$ have the same asymptotic expansion
in their respective sectors, the $S_k$ are constrained
to be triangular. In fact
\eqn\sto{
S_{2k}=\pmatrix{1&s_{2k}\cr 0&1\cr}\qquad\qquad
S_{2k+1}=\pmatrix{1&0\cr s_{2k+1}&1\cr}  }

The stokes parameters $s_i$ are not all independent but 
satisfy the constraints:

\noindent
1. $s_{k+2l+3}=s_k$.

\noindent
2. \eqn\mncnst{S_1\cdots S_{2l+3}=-i \sigma_1 }

\noindent
3. $s_{2l+3-k}=-\bar s_k$ if $u$ is real.

We may prove these properties as follows.
For property 1, we use
the symmetry of equation \dforz\ to conclude that
$W_{k+2l+3}(\zeta)=\sigma_1W_k(-\zeta)\sigma_1$ and hence that 
$S_{k+2l+3}=\sigma_1 S_k\sigma_1=S_k^{tr}$, from which 1 follows.
Next, for 2 we remark that the original 
equation in $\lambda$ is regular throughout the $\lambda$ plane.
The solution is simply $Pexp\int^\lambda A(\lambda')d\lambda'$ for
an appropriate matrix $A$ hence 
the only singularities in $\Psi$ can occur at infinity. Thus, near
$\zeta=0$ we have
\eqn\nrzr{
W(\zeta)\cong \half\zeta^{-1/2}\pmatrix{1&\zeta\cr 1&-\zeta\cr} 
}
By
\nrzr\ , if we analytically continue $W_1$ from $\Omega_1$ then 
$W_1(-\zeta)=-i\sigma_1W_1(\zeta)$ from which we obtain 2. 
Finally if $u$ is real then all the coefficients of powers of 
$\zeta$ in $\IP$ are real so that $\bar W_k(\zeta)=W_{-k}(\bar\zeta)$
implying 3. 

Since the determinant of \mncnst\ is automatically satisfied, 
\mncnst\ only imposes three independent constraints on the 
Stokes matrices so
we have $2l+3-3=2l$ independent Stokes parameters.
Note that this is the number of initial conditions in the 
string equation. In fact, the two sets of parameter spaces 
may be regarded as the same.

Let us now consider the implications of physical asymptotics. 
Consider the equation $\IP\Psi=0$
as $x\to \pm\infty$. We must distinguish several cases.
Since $R_l=\kappa_l u^l +\cdots $ with 
$$\kappa_l=(-1)^l{(2l-1)!!\over 2^{l+1}l!}$$
we see that perturbation theory predicts that the solution to
the string equation is given by
\eqn\ascses{\eqalign{
u\sim \bigl(-1/2\kappa_{l+1}\bigr)^{1/(l+1)}x^{1/(l+1)} &\qquad l\quad
{\rm even},\quad x\to\pm\infty\cr
u\sim \pm\bigl(1/2\kappa_{l+1}\bigr)^{1/(l+1)}(-x)^{1/(l+1)} &\qquad l\quad
{\rm odd}, \quad x\to -\infty\cr}
}
where $l$ is the largest index for which $t_{l+1}\not=0$. 
In fact, perturbation theory tells us to take the $+$ root for $u$ 
in the case of odd $l$, but we may easily examine both cases at once.
Rescaling variables and only keeping leading order terms we 
may rewrite the equation $\IP \Psi=0$ as
\eqn\limeqt{
{dW\over d\xi}=\tau\Biggl[ (2\xi^2\pm 1)p_l^\pm(\xi)\sigma_3
\mp p_l^\pm(\xi)i\sigma_2
-{1\over 2\xi \tau}\biggl({8l\kappa_{l+1}
\over l+1}\xi^2 p_l(\xi)+1 \biggr)
\sigma_1
+\CO(1/\tau^2)\Biggr]W
}
where
$p_l^\pm(\xi)=\sum_{p=0}^l (\pm 1)^p\kappa_p\xi^{2l-2p}$.
EXPLAIN WHY LOWER l's IRREL
For $l$ odd we obtain this equation with $x\to -\infty$, 
where: 
$$\zeta=(-x/2\kappa_{l+1})^{1/(2l+2)}\xi \qquad\qquad 
\tau=(-x/2\kappa_{l+1})^{(2l+3)/(2l+2)}\qquad .$$ 
For $l$ even we obtain this equation with $\pm x\to +\infty$, where
$$\zeta=(\pm x/(-2\kappa_{l+1}))^{1/2l+2}\xi \qquad\qquad 
\tau=(\pm x/(-2\kappa_{l+1}))^{(2l+3)/(2l+2)}\qquad .$$

The evaluation of the Stokes matrices is carried out by doing 
a WKB analysis in the $\tau\to\infty$ limit as in \Its . Thus one 
obtains true solutions to \limeqt\ which are asymptotic as
$\tau\to\infty$ to the WKB ansatz:
$$ W^{WKB}\sim
T exp\Biggl[\tau\int^\zeta\Lambda(\zeta')d\zeta'
\Biggr]
$$
where $T$ diagonalizes \limeqt to $\Lambda=\mu\sigma_3$. 
There are several WKB solutions in different regions defined by 
the turning points and conjugate Stokes lines. The turning 
points are simply the roots $\xi_i$ of $\mu$, and the  
conjugate Stokes lines are the lines defined by the vanishing real 
part:
\eqn\skis{
\Re\int_{\xi_i}^{\xi}\mu(\xi')d\xi'=0
}
The general procedure for finding the stokes matrices is described in 
\Its . A simple consequence of this procedure allows us to 
obtain certain necessary conditions on 
the Stokes matrices which, in some cases, fix the parameters 
uniquely. The main observation is that
if, at a turning point 
which is a root of $p_l$ the conjugate Stokes lines form three large 
regions each abutting an open region infinity as in fig. 6,
then the Stokes matrix for the transition function associated with the middle 
region $\tilde \Omega_2$ is trivial \geom\ .

Our first task is therefore to describe the Stokes lines.
For the the case $l=2$ we have fig. 9 and fig. 10. 
From the limit $x\to +\infty$ we obtain that $s_1=s_6=0$. 
From the limit $x\to -\infty$ we obtain $s_2=s_5=0$, and 
from the monodromy constraints we get
$s_0=s_3=s_4=-i$.
Unfortunately,
if we move on to higher $l$ we meet a technical problem 
which is that the configuration of Stokes lines becomes
too complicated to use our observation immediately to set 
half the Stokes parameters to zero. 
Nevertheless, studying small $l$ leads to a 
natural guess for the stokes parameters in 
the general case. For 
$l$ even, comparing the constraints from the 
physical asymptotics
at either end 
of the axis will fix two disjoint sets of Stokes
parameters. We expect that $s_1=s_2=\cdots =s_l=0$
while $s_{l+1}=s_{l+2}=s_{2l+3}=-i$ so that the solution 
is unique, and the stokes data is always concentrated on 
the wedges abutting the $x$ and $y$ axes in the 
$\zeta$ (=$\sqrt{\lambda}$) plane.

We have not used the reality 
constraints in a very essential way. An interesting 
solution of the Painlev\'e equation is the triply truncated
solution, characterized by $u\sim +(-x)^{1/2}$ for 
$x\to-\infty$ and $u\sim\pm i x^{1/2}$ for $x\to \infty$. 
Using the above technique one easily shows that in this
case the constraints $s_2=s_3=-i$, $s_0=s_5=0$, 
$s_1+s_4=0$ are supplemented by $s_1=0$ or $s_4=0$, 
depending on the sign of the imaginary part. In either
case we confirm that the solution is unique.

An analog of the BMP solutions exists for the 
PII family
\ref\Hast{S.P. Hastings and J.B. McLeod, ``A boundary value 
problem associated with the second painlev\'e transcendent and the 
Korteweg-de Vries equation,'' Arch. Rat. Mech. and Anal. {\bf 73}(1980)31}
\ref\cdm{\v C. Crnkovi\'c, M. Douglas, and G. Moore, 
``Physical solutions for unitary-matrix models,'' Yale preprint
YCTP-P6-90.}. 
In this case physical asymptotics specifies 
that $u(x)$ grows algebraically at one end of the 
axis and decays exponentially at the other end
\dss\cdm\ . Applying the above techniques one finds 
that in this case (at least for the first two members
of the PII family) the stokes data is concentrated 
entirely on the $y$-axis, but the necessary conditions
leave a single parameter undetermined. In this case,
however, the inverse monodromy problem is equivalent
to an integral equation (the self-similar Gelfand-Levitan-Marchenko
equation) which can be examined directly. 
This is done in \cdm\ where it is shown that 
the physical solution is unique.


Again it is interesting examine the Stokes data.
We may again apply our lemma for the first two 
cases in the series. 
In the first two cases in the PII 
hierarchy these two constraints 
lead to a configuration of Stokes lines
from which we may conclude 
{\it necessary} conditions that,
together with the monodromy constraints only determine 
the Stokes parameters up to a single real parameter.
The nontrivial Stokes matrices are again concentrated 
the y-axis.

\subsec{KdV orbits are disconnected}

The string equation and kdv flow are compatible equations.
This does not mean that, as we change the $T_j$ the solution 
automatically satisfies kdv flow. As we have seen above, 
if the stokes data are held fixed then we do have kdv 
flow. If we consider the kdv orbits of solutions of the 
string equation we may wonder whether they are connected. 
Certainly they are formally connected. For example,
consider the equation
\eqn\flowi{
\half(m_1+\half)R_{m_1}+\half(m_2+\half)T R_{m_2}=x
}
with $m_1>m_2$. As discussed in \bdss
\ref\cgmr{C. Crnkovic, P. Ginsparg, and G. Moore, ``The Ising model,
the Yang-Lee edge singularity, and $2D$ quantum gravity,'' Phys. 
Lett. {\bf 237B}(1990)196.}\ 
if one scales a solution $u(x;T)$ to \flowi\ using 
$v(y;T)=T^{-2/(2m_2+1)}u\bigl(T^{-1/(2m_2+1)}y;T\bigr)$, then 
the large $T$ limit $v(y)=\lim_{T\to\infty}v(y;T)$ must
be a solution of the lower order string equation 
$\half(m_2+\half)R_{m_2}[v]\sim x$, provided the limit
exists. However, the existence of this limit is a very delicate 
issue. Indeed, in \dss\ convincing evidence is presented that
the flow from $m=3$ to $m=2$ does not exist. 

Since, as we have seen, physical asymptotics fixes the stokes 
data, and since kdv flow is isomonodromic, we are in a position 
to investigate analytically the result in \dss\ .
Suppose the $T\to\infty$ limit does exist.
Then we can scale $\zeta\to T^{-1/(2m_2+1)}\zeta$ in \dforz\ 
to obtain a connection with a smooth $T\to\infty$ limit. Since
solutions can in principle be obtained from the path-ordered exponential
of the connection, solutions to \dforz\ will also have smooth 
$T\to\infty$ limits. Moreover, from the asymptotics in $\zeta$
we see that the coefficients have smooth $T\to\infty$ limits
and in fact approach the asymptotics of the lower order equation.
Thus, fundamental solutions smoothly approach fundamental solutions for the
lower order equation, although they will be defined on small regions of
angular width ${2\pi\over 4m_1+2}$ and hence only define part of a
fundamental solution for the lower order equation which is defined on the
larger regions of angular width ${2\pi\over 4m_2+2}$. Because of this 
we find two rules governing flows:

\noindent
1. A large region associated with a trivial Stokes matrix for the 
``$m_2$ equation'' cannot contain a small region with a nontrivial
Stokes matrix for the ``$m_1$ equation.''

\noindent
2. A large region associated with a nontrivial Stokes matrix for the 
``$m_2$ equation'' must contain at least one small region with a 
nontrivial Stokes matrix for the ``$m_1$ equation.''

Note that for even and odd $l$ the nontrivial Stokes data disagrees
on the real axis. Hence, by rule 1 it is impossible to use KdV flow 
to go from an even $l$ to an odd $l$ model, confirming the result of 
\dss\ . Note that flow from an even $l$ to a smaller even $l$
is consistent with rules 1 and 2. 


COMPARE DAVID'S RESULTS!!!
CONTRADICTION???

\newsec{Isomonodromy and Free Fermions}

In this section we will 
interpret the isomonodromy problem connected with the 
string equations in conformal-field-theoretic terms. 
Our paradigm will be the solution of the Riemann-Hilbert
problem for the case of regular singular points given ten 
years ago by the Kyoto school
\ref\holoii{M. Sato, T. Miwa, M. Jimbo, ``Holonomic Quantum Field Theory
II,'' Publ. RIMS {\bf 15}(1979)201}.
We review their construction first, in the light of subsequent
developments in CFT. Then we consider the case of irregular 
singular points. Developing further some work of Miwa
\ref\miwai{T. Miwa, ``Clifford operators and Riemann's monodromy problem,''
Publ. Res. Inst. Math. Sci. {\bf 17}(1981)665}, we find
that the theory of irregular singular points can be included 
at the expense of the introduction of a new kind of operator.
In a way we are 
making a nontrivial extension of conformal field theory by expanding 
the class of functions admitted in the theory from analytic functions 
with algebraic singularities to analytic functions with essential 
singularities. This is reflected in the need to expand the 
class of operators from twist operators to star operators.

\subsec{Regular Singular Points}

The basic idea of \holoii\ is that the solution to an $m\times m$
matrix differential equation 
\eqn\diffl{
{d\Psi\over dz}=A(z)\Psi(z) }
may be characterized uniquely by its monodromy properties. 
More precisely, suppose $A$ has only simple poles at points 
$a_\nu$ and the residue can be diagonalized to $L_\nu$. Then
the matrix $\Psi$ can be uniquely characterized by 
the requirement that 

(i.) $\Psi(z_0)=1$

(ii.) $\Psi(z)$ is holomorphic in $z\in\IP^1-\{a_1,\dots , a_n\}$

(iii.) $\Psi(z)=\hat \Psi^{(\nu)}(z)
e^{L_\nu log(z-a_\nu)}$ for, $z\cong a_\nu$
where $\hat \Psi^{(\nu)}$ is holomorphic and invertible in a neighborhood
of $a_\nu$. 

Conversely, any such matrix defines a rational matrix
$A=\Psi_z\Psi^{-1}$ with at most simple poles.
Thus, 
if one can construct appropriate ``twist operators'' 
$\varphi_i$
such that the correlation function 
\eqn\corr{\Psi_{\beta\alpha}(z_0;z)=
(z_0-z)
{\langle
\bar \psi_\beta(z_0)\psi_\alpha(z) \varphi_n(a_n)\cdots \varphi_1(a_1)
\rangle\over
\langle \varphi_n(a_n)\cdots\varphi_1(a_1)\rangle } 
}
has the correct monodromy properties, then it must be a solution of 
\diffl . Thus we have reduced the global Riemann-Hilbert
problem to the {\it local} problem
of finding conformal fields $\varphi_{L}(a)$ with the operator 
product expansion 
\eqn\twis{
\psi_\alpha(z)\varphi_L(a)\sim \bigl[\CO^0_{L,\alpha}(a)+
(a-z)\CO^1_{L,\alpha}(a)+\cdots\bigr](z-a)^{L_\alpha} }

The construction of these operators proceeds by
choosing a basis of curves
$\gamma_\nu$ circling once around $a_\nu$ and generating the 
fundamental group $\pi_1(\IP^1-\{a_\nu\};z_0)$ 
and defining:
\eqn\gentwis{
\varphi_{M}(a)=exp\Biggl[\int_\CC^a tr\biggl\{log(M)J(y)\biggr\}
{dy\over 2\pi}\Biggr] 
}
where $J_{\beta\alpha}=\bar\psi_\beta\psi_\alpha$ 
and $\CC$ is a contour (a branch cut
for $\Psi$) emanating from $a$. Consider now \corr\  with such
operators inserted. As we analytically continue $z$ around
$a$ the simple pole in the ope of $\psi$ with $J$
gives rise to the monodromy $\psi_\alpha\to \psi_\gamma M_{\gamma\alpha}$
in the Fermi field. Therefore \corr\ solves 
conditions ({\it i-iii}). 

Let us now consider isomonodromic deformation of \diffl\ .
It is clear from locality of the ope that changing the 
$a_\nu$ leaves the monodromy data unchanged. 
Necessary and sufficient conditions
for isomonodromic deformation are given by:
\eqn\schles{\eqalign{
{\p \Psi\over \p z_0}&=-\sum {A_\nu\over z_0-a_\nu}\Psi\cr
{\p \Psi\over \p a_\nu}&=\biggl(-{A_\nu\over z-a_\nu}+-{A_\nu\over z_0-a_\nu}
\biggr)\Psi\cr}
}
The
compatibility conditions for these linear equations give the 
nonlinear Schlesinger equations
\foot{For an appropriate choice of matrices, for example, 
these equations reduce to PVI \Jimboii .}.
From \corr\ we see that the linear equations of
isomonodromic deformation theory should 
be thought of as transport equations on moduli space, 
analogous to the Knizhnik-Zamalodchikov equations,
so that the theory of isomonodromic deformation
for regular singular points fits nicely 
into the framework of Friedan-Shenker modular geometry.

According to the general theory 
of isomonodromic deformation \Jimboi\Jimboii\Jimboiii\ there is 
a tau function associated to the deformation parameters $a_\nu$.
In this case it is given by 
\eqn\taufn{
d~log~\tau(a_1,\dots, a_n)=-\sum_\nu Res_{z=a_\nu}tr
\Biggl[ (\hat \Psi^{(\nu)})^{-1}
{\p \hat \Psi^{(\nu)}\over\p z} d \biggl(log(z-a_\nu)L_\nu\biggr)\Biggr]
}
where the $d$ is a differential in the parameters $a_\nu$. 
In fact the $\tau$ function is given by
$\tau=\langle \varphi_1\dots\varphi_n\rangle$\ 
\Jimboi\Jimboii\Jimboiii\ .
We will now rederive this
using general principles of conformal field theory
\foot{Exactly the same observation was made 3 years 
ago in \knizhnik\ , ch. 4. I was unaware of this work when 
I published \geom .}.

We have normalized \corr\ so that it is equal to $\delta_{\alpha\beta}$
at $z=z_0$. Taking the 
operator product expansion as $z\to z_0$ and matching this with 
an expansion of a solution to \diffl\ around $z_0$ we find
\eqn\corrii{\eqalign{
{\langle
J_{\beta\alpha}(z_0) \varphi_n(a_n)\cdots \varphi_1(a_1)
\rangle\over
\langle \varphi_n(a_n)\cdots\varphi_1(a_1)\rangle }
&=-A(z_0)_{\beta\alpha} \cr
{\langle
\CT(z_0) \varphi_n(a_n)\cdots \varphi_1(a_1)
\rangle\over
\langle \varphi_n(a_n)\cdots\varphi_1(a_1)\rangle } 
&=\half tr A^2(z_0)\cr} }
where 
$\CT$ is the stress energy tensor. Since $L_{-1}$ always takes a
derivative with respect to position we have
$${\p\over\p a_\nu}log[\langle\varphi(a_1)\cdots\varphi(a_n)\rangle]
=
\oint_{a_\nu}dz_0
{\langle
\CT(z_0) \varphi_n(a_n)\cdots \varphi_1(a_1)
\rangle\over
\langle \varphi_n(a_n)\cdots\varphi_1(a_1)\rangle } 
=
\oint_{a_\nu}dz_0 \half trA^2(z_0)$$
On the other hand, substituting the local expansion 
$\Psi=\hat\Psi(z-a)^L$ we get
\eqn\lcexp{\hat \Psi^{-1}\hat \Psi_z + L/(z-a)
=\hat \Psi^{-1}A\hat \Psi\qquad .}
Squaring this equation we find
\eqn\sqaring{
{\p\over\p a}\bigl(log~\tau\bigr)=Res~tr~\biggl(\hat\Psi^{-1}\hat\Psi_z
{L\over z-a}\biggr)=\half Res~tr~A^2
\qquad ,}
and hence the tau function is simply the correlation function 
of twist operators. 

\subsec{Irregular Singular Points}

Let us now attempt to repeat the 
previous discussion for the case of 
a differential equation \diffl\ where 
$A$ is rational but can have irregular singularities. 
Our treatment is the same in spirit as the discussion of 
T. Miwa \miwai\ , although there are some
differences of detail.

Recall that
at an irregular singular point we divide up a neighborhood 
of the point into sectorial domains, each containing 
a fundamental solution with asymptotics 
\eqn\stoii{
\Psi\sim\biggl(\sum_{l\geq 0}\hat\Psi^{(l)}(z-a)^l\biggr)(z-a)^L
e^{T(z-a)} }
where $L$ and 
$$T(z-a)=\sum_{i=1}^{r}{T_r\over (z-a)^r}$$ 
are diagonal, and $\hat \Psi^{(0)}$ 
is invertible. In 
particular, 
the analytic continuation of $\Psi_1$ will have the asymptotic 
expansion 
\eqn\stoki{
\Psi_1\sim\biggl[\sum_{l\geq 0}
\hat\Psi^{(l)}(z-a)^l\biggr](z-a)^L e^T(S_1\cdots
S_{k-1})^{-1} }
in the sector $\Omega_k$. 
A solution to the differential equation can be uniquely characterized
by the conditions $(i)-(iii)$ above except that $(iii)$ must
be replaced by the requirement that $\Psi$ have the asymptotic
expansions \stoki\ . 
Assume there is only one
irregular singular point and
define $\tilde \Psi\equiv \Psi_1 e^{-T(z-a)}$. 
We now search for quantum 
field operators $V_{S,T,L}(a)$, which we call ``star'' operators,
such that 
\eqn\corstar{
\tilde\Psi_{\beta\alpha}(z_0;z)= 
(z_0-z)
{\langle
\bar \psi_\beta(z_0)\psi_\alpha(z) V_{S_1,T_1,L_1}(a_1)\cdots \rangle\over
\langle V_{S_1,T_1,L_1}(a_1)\cdots\rangle } 
}
Evidently,
a star operator is characterized by its operator product expansion 
with $\psi$, $\bar\psi$, e.g., for $z\in \Omega_k$ we have
\eqn\starope{
\psi_\alpha(z)V_{S,T,L}(a)\sim \bigl[\CO^0_{\gamma}(a)+
(a-z)\CO^1_\gamma(a)+\cdots\bigr](z-a)^{L_\gamma}\bigl[
e^T(S_1\cdots S_{k-1})^{-1}e^{-T}\bigr]_{\gamma\alpha}
}
From this description it looks very unlikely that star operators
exist 
\foot{For example, it is often claimed that operator product expansions
in CFT are convergent. Note that \starope\ is only asymptotic.
The reason for this is ultimately to be found in the fact that 
the string coupling has become dimensionful
\DS . Note that it is $x$ and the masses $T_j$ 
which multiply the terms giving the essential 
singularity at infinity. }.
We now give at least a formal construction of these operators.

Consider a ray $\CC$ emanating from a point $a$. Consider the product 
of operators 
$$exp\bigl[\int^a_\CC dy\psi_\alpha(y) M_{\alpha\beta}(y)\bar\psi_\beta(y)
\bigr]\psi_\alpha(z)$$
where $M$ is some matrix defined along the line. 
If we analytically continue in $z$ through 
the curve $\CC$ and compare with the other operator ordering 
it is a simple consequence of Cauchy's theorem and the 
operator product expansion that we have the exchange algebra:
\eqn\galg{\eqalign{
exp\bigl[\int^a_\CC dy\psi_\alpha(y) M_{\alpha\beta}(y)\bar\psi_\beta(y)
\bigr]\psi_\alpha(z-\epsilon)&=\cr\cr
\psi_\gamma(z+\epsilon) (e^{M(z)})_{\gamma\alpha}&
exp\bigl[\int^a_\CC dy\psi_\alpha(y) M_{\alpha\beta}(y)\bar\psi_\beta(y)
\bigr]\cr}
}
where $z$ is a point on $\CC$ and $z\pm\epsilon$ are points above and 
below $z$. Thus, defining $\CS_k\equiv e^T S_k e^{-T}$ we may define, 
at least formally, 
\eqn\stardef{
V_{S,T,L}(a)=\varphi_L(a)\prod_k exp\Biggl[\int^a_{\CC_k}
tr\biggl\{(log \CS_l(y))J(y)\biggr\}dy \Biggr]
}
where we choose contours $\CC_k$ in $\Omega_k$ such that the 
matrix $\CS_k$ approaches the identity rapidly. 
In \miwai\ T. Miwa obtained formulae for star operators using 
a slightly different formalism. Comparing his formulae in 
terms of infinite expansions evaluated by Wick's theorem we 
obtain the same result.
As shown in \miwai\ contours of integration can be defined so 
that for small enough Stokes data the integrals make 
sense, thus giving a more precise definition to the star 
operator.

It follows from locality of the operator product 
expansion that the differential equation 
satisfied by \corstar\ has the property that the monodromy
data $S,L$ are preserved if we vary the parameters $a_i,T_i$. 
Just as the tau function for isomonodromic deformation of 
an equation with regular singular points is given by the 
correlation function of twist operators, the tau function for 
isomonodromic deformation in an equation with irregular singular
points
is given by the correlation function
of star operators \miwai\ . One may give a formal 
argument for this following steps analogous to those 
leading from 4.7 to 4.9. The analog of \corrii\ is
\eqn\stress{\eqalign{
{\langle
J_{\beta\alpha}(z_0) V_{S_1,T_1,L_1}(a_1)\cdots 
\rangle\over
\langle V_{S_1,T_1,L_1}(a_1) \cdots\rangle }
&=-A(z_0)+T'(z_0)\cr
{\langle
\CT(z_0) V_{S_1,T_1,L_1}(a_1)\cdots 
\rangle\over
\langle V_{S_1,T_1,L_1}(a_1)\cdots \rangle } 
&=\half tr (A-T')^2(z_0)\cr}
}
where $\CT$ is the stress-energy tensor.
Using formal manipulations with the ope one can show that
\eqn\gobpe{
-Res_{z_0=a} tr\delta T(z_0) A(z_0)=
{\sum_k\int^a_{\CC_k}dy tr\biggl(\delta T(y){\delta\over
\delta T(y)}\biggr)\langle V_{S,T,L}\cdots\rangle\over
\langle V_{S,T,L}(a)\cdots \rangle}
}
Putting together \stress\ and \gobpe\ we then find
\eqn\taustar{
{d\over da}~log~\langle V\cdots V\rangle= \half Res_{z_0=a} 
tr~A^2(z_0)
={d\over da}~log~\tau
}
where the second equality follows from an argument similar to 
the case of regular singularities. Similarly, one can show
that the dependence
of $log~\tau$ and $log\langle V\cdots V\rangle$ on other 
parameters is the same.

\subsec{$\tau$ functions for 2D gravity}

As a special case of the above formalism we can express the solution 
$u$ of the string equations in terms of a fermion correlation function. 
We represent the solutions $W$ to \dforz\ and $\Psi$ to \lasaga\ as 
fermion two-point functions in the presence of star operators. For 
the $\tau$ function of the PII family we may define
$t(y)\equiv \half
\sum_\ell {2m+1\over 2\ell +1}t_\ell \lambda^{2\ell+1}$, so that 
$\tau(t_\ell)=\langle V(\infty;s_k,t_\ell)\rangle$ where
\eqn\srptw{\eqalign{
V(\infty;s_k,t_\ell)=\prod_k &
exp\biggl[s_{2k+1}\int^a_{\CC_{2k+1}}e^{2t(y)}\psi_1\bar\psi_2(y)\biggr]\cr
&\qquad
exp\biggl[s_{2k}\int^a_{\CC_{2k}}e^{-2t(y)}\psi_2\bar\psi_1(y)\biggr]\cr}
}
For the PI family we have a fermion twist operator at the origin 
(from the regular singularity in \dforz\ ). If we bosonize the 
two fermions $e^{i\phi_i}=\psi_i$ then the twist 
operator at the origin is just $e^{i\omega/\sqrt{2}}$
where $-i\sqrt{2}\p\omega=\bar\psi\sigma_3\psi$. The 
$\tau$ function is now 
\eqn\srpon{
\tau(t_\ell)=\langle V(\infty;s_l,t_\ell) e^{i\omega/\sqrt{2}} \rangle}
where the star operator is the same as in the PII case but 
$t(y)=-{1\over 4}\sum t_j\zeta^{2j+1}$.

These expressions are, of course, rather formal. It would be worthwhile
making rigorous sense of them since, at least formally, they make
transparent some interesting properties of the 2D gravity 
partition function. For example, a corollary of the operator 
formalism is that a $\tau$ function satisfies certain 
Virasoro constraints. Applying \stress\ to this case
with $A,T$ obtained from \dforz\lasaga\ we find:
\eqn\vircni{
\eqalign{
L_n\tau&={1\over 4}\delta_{n,0}\tau\qquad n\geq-1\qquad{\rm for~
PI}\cr
L_n\tau&=0\qquad \qquad n\geq-1\qquad{\rm for~
PII}\cr} }

Similarly since commutation with $J=\bar\psi\sigma_3\psi$
rotates the fermions $\psi_{1,2}$ oppositely we may imagine 
that there is an identity like
\eqn\cohi{
e^{\int_\CC\bigl(t^{(2)}(y)-t^{(1)}(y)\bigr)J(y)dy}
V(a;s,t^{(1)}_\ell)
e^{-\int_\CC\bigl(t^{(2)}(y)-t^{(1)}(y)\bigr)J(y)dy}
=V(a;s,t^{(2)}_\ell)}
and hence we would have
\eqn\cohii{
\tau(t_\ell)=\langle t_\ell-\bar t_\ell|\Omega_{\bar t}\rangle}
where $|\Omega_{\bar t}\rangle$ is the state created by 
the star operator at $\bar t$, and $\langle t_\ell-\bar t_\ell|$
is a coherent state for the scalar $\omega$ where only the 
odd oscillator modes are excited. (This follows since 
$t(y)$ involves only odd powers of $y$.) Combining with 
\vircni\ we may obtain, formally, expressions similar to those
in \fukuma\dvv\morozovi\morozovii\ .

Finally, let us compare with the fermion formalism of the 
matrix model described in section 2. There we found that the 
partition function at couplings $t_\ell$ can be expressed in 
terms of a correlation function in the ground state for 
couplings $\bar t_\ell$ according to:
\eqn\mtrxtaus{
\eqalign{
\tau_{PI}&=\biggl\langle e^{\int_\IR\sum_{\ell}\bigl(t_\ell-\bar
t_\ell\bigr)\lambda^{\ell +1/2}\psi^\dagger\psi(\lambda) d\lambda}
\biggr\rangle\cr
\tau_{PII}&=\biggl\langle e^{i\int_\IR\sum_{\ell}\bigl(t_\ell-\bar
t_\ell\bigr)\lambda^{2 \ell +1}\psi^\dagger\psi(\lambda) d\lambda}
\biggr\rangle\cr} }

DERIVE? natural explanation of stokes data etc .

For the $(2l-1,2)$ models we have a star operator at infinity and a 
twist operator at the origin. We may choose the contours of the star 
operator to converge to the branch cut for the twist operator at the 
origin. Then applying \gobpe\ to the case where we vary the parameter
$x$, (one of the parameters in $T(\zeta)$) we see that the 
tau function for the string equations can be computed using the 
star operator,
\eqn\grvstr{\eqalign{
V_{S,x}(a)=e^{i(\phi_1(a)+\phi_2(a))/2}\prod_k &
exp\biggl[s_{2k+1}\int^a_{\CC_{2k+1}}e^{2T(y)}\psi_1\bar\psi_2(y)\biggr]\cr
&\qquad
exp\biggl[s_{2k}\int^a_{\CC_{2k}}e^{-2T(y)}\psi_2\bar\psi_1(y)\biggr]\cr}
}
and a twist operator $\varphi$, as the two-point 
function:
\eqn\taustr{
\tau(x)=\langle V_{S,x}(\infty)\varphi(0)\rangle
}
Very similar considerations hold for the string equations in 
the PII family, the main difference being that 
we do not have to take a squareroot of the spectral parameter, 
hence there is no twist field.

\subsec{Virasoro action, heuristic origin from matrix model}


\newsec{Grassmannians, Krichever's construction, and all that}


When the connection to the KDV hierarchy was discovered in 
2D gravity the theory of the KDV hierarchy, as presented in 
\ref\BMN{Dubrovin, Matveev, and Novikov, ``Non-Linear Equations 
of Korteweg-De Vries Type, Finite-Zone Liner Operators, and 
Abelian Varieties,'' Russian Math Surveys, {\bf 31} (1976)59}
\ref\qpKdV{E. Date, M. Jimbo, M. Kashiwara, and 
T. Miwa, ``Transformations Groups for Soliton Equations,''
I. Proc. Japan Acad. {\bf 57A}(1981)342; II. Ibid., 387;III. J. Phys. Soc.
Japan {\bf 50}(1981)3806;IV. Physica {\bf 4D}(1982)343;V. Publ. RIMS, 
Kyoto Univ. {\bf 18}(1982)1111;VI. J. Phys. Soc. Japan {\bf 50}
(1981)3813;VII. Publ RIMS, Kyoto Univ. {\bf 18}(1982)1077.}
\ref\segal{G. Segal and G. Wilson, ``Loop Groups and Equations of 
KdV Type,'' Publ. I.H.E.S. {\bf 61}(1985)1.} , 
was already familiar to string theorists. 
Indeed, one of the main points of the so-called operator
formalism
\foot{Representative papers include...}
is the equivalence of 
the Krichever theory with the 
theory of free fermions on an algebraic curve.
In the previous section we related the $\tau$
function of 2D gravity to a correlation function of 
free fermions.
In this section we will see that the two theories 
are closely connected, but
because of stokes phenomenon we require an 
extension of the the theory in \BMN\qpKdV\segal\ .

\subsec{Quasiperiodic kdv flow and isomonodromy}

We begin by explaining a remark of M. Jimbo and T. Miwa
that the tau function of the quasiperiodic
solutions to the kdv equations is a special case of 
the isomonodromic tau function \Jimboii\ .

Recall that quasiperiodic kdv flow is straightline 
motion along $Pic_{g-1}(X)$ for a riemann surface $X$
of genus $g$, and, fixing an origin $\CL_0$ for
$Pic_{g-1}$ the tau function is just 
\eqn\qptau{tau={Det~\bar\p_{\CL}\over Det~\bar\p_{\CL_0}} \qquad .}
If $\CL\otimes\CL_0^{-1}$ has divisor $P_1+\cdots P_g-Q_1-\cdots-Q_g$
then by the insertion theorem
\ref\bost{Alvarez-Gaum\'e, Bost, Moore, Nelson, Vafa}
we have
\eqn\qpti{\tau=\langle\psi(P_1)\cdots\psi(P_g)\bar\psi(Q_1)\cdots
\bar\psi(Q_g)\rangle_{(X,\CL_0)} \qquad .}
Choosing a point $P_\infty$ ``at infinity'' and a local coordinate
$1/z$ near $P_\infty$ the Baker function is essentially
\eqn\qpbki{\langle\bar\psi(P_\infty)
\psi(z)\psi(P_1)\cdots\psi(P_g)\bar\psi(Q_1)\cdots
\bar\psi(Q_g)\rangle_{(X,\CL_0)} \qquad .}
Now suppose $\pi:X\to \IP^1$ is an $m$-fold branched covering.
As is well-known from the theory of orbifolds 
\ref\zam{Al. B. Zamalodchikov, ``Conformal scalar field on the 
hyperelliptic curve and critical Ashkin-Teller multipoint 
correlation functions,'' Nucl. Phys. {\bf B285}(1987)481.}
\ref\berrad{M. Bershadsky and A. Radul, ``Conformal field theories
with additional $Z_N$ symmetry,'' Int. Jour. of Mod. Phys. 
{\bf A2}(1987)165.}
\ref\lance{L. Dixon, D. Friedan, E. Martinec and S. Shenker, 
``The conformal field 
theory of Orbifolds,'' Nucl. Phys. {\bf B282}(1987)13.}
\ref\vafa{ S. Hamidi and C. Vafa, ``Interactions on Orbifolds,'' 
Nucl. Phys. {\bf B279}(1987)465}
\knizhnik\ 
we can represent one weyl fermion on $X$ by $m$ weyl fermions 
$\psi_\alpha,\bar\psi_\alpha$ on $\IP^1$, where 
$\alpha=1,\dots, m$ labels the sheets, in the presence of 
twist operators at the branch points. For example,
denoting the branch points on $\IP^1$ by $b_i$ we have
the twist field correlator
$Det~\bar\p_{\CL_0}=\langle \prod\varphi_i(b_i)\rangle\qquad .$
Similarly, if $P_i$, $Q_i$ lie on branches $\alpha_i,\beta_i$, 
respectively then the Baker function descends to the 
``Baker framing''
\eqn\qpbkfr{\tilde Y_{\alpha\beta}(\lambda)=
\langle\bar\psi_\alpha(\infty)
\psi_\beta(\lambda)\psi_{\alpha_1}(\pi(P_1))\cdots
\psi_{\alpha_g}(\pi(P_g))\bar\psi_{\beta_1}(\pi(Q_1))\cdots
\bar\psi_{\beta_g}(\pi(Q_g))\prod_i\varphi_i(b_i)
\rangle_{\IP^1} \qquad ,}
where $\lambda=\pi(z)$. The actual Baker framining $Y$ differs
from $\tilde Y$ by an invertible 
diagonal matrix with an essential singularity
at infinity of the form $\sim exp\bigl(\sum t_j z^j\bigr)$.
Regarding the fermion insertions as special cases of twist operators
and following the reasoning of the previous section we see that 
$Y$ satisfies a liner ODE in $\lambda$ with regular singularities
at $\pi(P_i)$. Clearly we have isomonodromic deformation in 
these parameters, and, as we have seen, the $\tau$ function is 
just
\eqn\qptii{
\langle
\psi_{\alpha_1}(\pi(P_1))\cdots
\psi_{\alpha_g}(\pi(P_g))\bar\psi_{\beta_1}(\pi(Q_1))\cdots
\bar\psi_{\beta_g}(\pi(Q_g))\prod_i\varphi_i(b_i)
\rangle_{\IP^1} \qquad ,}
but this is the same as \qpti\ which is the tau function of 
the krichever theory.

From this discussion we conclude that the required generalization of 
the krichever theory is the generalization from twist operators 
to star operators. In the next section we will arrive at the 
same conclusion from a different point of view.

\subsec{Noncommutative Burchnall-Chaundy-Krichever theory}

BCK THEORY: based on [P,Q]=0. here [P,Q]=1. noncommutative rs?
put hbar, take limit to relate the two.
impossible to diagonalize simultaneously. must look to another 
formulation. go to lax pair. 

Compare us, novikov, is his conjecture right? how do we
see his transcendental equation? other papers.


{\subsec Grassmannians}

general discussion of tau functions-- is it the most
general solution to kdv hierarchy?

also--compare with segal-wilson tau function, argue that it 
is NOT in their class because of the baker function. 
note that already you can see this in pure gravity, 
where ``baker function'' is just an airy function and 
exhibits stokes phenomenon. nevertheless, free fermion 
interpretation in terms of ``star'' operators etc. shows
that there must be a sense in which this tau lies on 
some kind of completion of the orbit.



It is well known that in the case of almost periodic solutions to 
the KdV hierarchy the tau function is properly thought of as a 
section of a determinant line bundle \segal\ 
and that this determinant line bundle is just the vacuum 
bundle for free fermions defined on the algebraic curve associated to 
the solution $u(x)$ via the BCK theory.
We can argue that, with some modifications, 
this picture continues to hold
for the tau functions associated to the isomonodromy problems 
discussed above.

Recall that
in the operator formalism we choose a disc surrounding some point 
$P$ and define a local Hilbert space on a circle surrounding that 
point. The partition function of fermions on the surface may 
be regarded as a function of the line bundle $\CL\to\Sigma-\{P\}$
of which the fermion wavefunctions are holomorphic 
sections. By considering the restriction of these sections to 
a circle surrounding $P$ we obtain
a subspace $W\subset L^2(S^1)$ defining an element of the Grassmannian.
Denoting by $\Omega$ the fermion vacuum created by the geometry
$\CL\to\Sigma-\{P\}$, that is, $\Lambda^{max}W$ we obtain:
\eqn\taudet{
\tau_W=\langle 0|\Omega_W\rangle=\int_W d\psi d\bar \psi
e^{\int \bar\psi\bar\p\psi}=Det\bar\p_W
}
As explained in 
\ref\wittgr{E. Witten, ``Conformal field theory, Grassmanians, 
and algebraic curves,'' Commun. Math. Phys. {\bf 113}(1988)189}\
we may easily incorporate correlation functions into this 
picture through use of the multiplicative Ward identities. 

Recall that
if we want a correlator of 
$\langle \prod\psi(P_i)\prod\bar\psi(Q_i)\rangle$
when the fermion field lives in the 
line bundle $\CL\to\Sigma$ associated to 
$W\in Gr$ we proceed as follows. Note that the Lagrangian is 
invariant under $\psi\to f\psi$, $\bar\psi\to f^{-1}\bar\psi$. 
This does not imply that the expression \taudet\ 
is invariant, since
the allowed space of sections is drastically changed.
First of all,
the boundary conditions are modified by $W\to f\cdot W$. Secondly,
at zeroes and poles of $f$ the $L^2$ condition on the 
wavefunctions becomes an unusual normalizability constraint.
Using a local analysis of the Hilbert space on a circle surrounding
such a pole one can show 
\wittgr\ that this unusual constraint can be replaced by 
the insertion of an appropriate operator in the presence
of ordinary fermions. As an example, let us take $f$ to 
have a single zero and pole.
Taking into account
the conformal weight one half 
of fermions we get
\eqn\multwrd{[(df^{-1}/dz)_P(df/dz)_Q]^{-1/2}
\langle\bar\psi(P)\psi(Q)\rangle_{f\cdot W}={Det\bar\p_{W}\over
Det\bar\p_{f\cdot W}}
}
As pointed out in \wittgr\ this is the physical explanation for 
the relation between the Baker function and the tau function.

Suppose now we have an $m$-tuplet of fermions $\psi_\alpha$
defined on a riemann surface $\Sigma$. Removing 
a neighborhood of a point $P\in\Sigma$,
the boundary conditions on $\vec\psi$ define a subspace of 
$W\subset Gr(H^m)$ where $H=L^2(S^1)$ as usual
\foot{The relation between the quantum field theory 
of a fermion on a
surface $\Sigma$ which is an $m$-fold cover of the plane, 
and the quantum field theory of an $m$-tuplet of fermions 
on the plane in the presence of branch points gives a 
nice explanation of the construction on pp. 34-35 of
\segal\ . Moreover, it shows directly why their association of a 
space $W$ in $Gr^{(n)}$ to an $n^{th}$ order operator
coincides with the space 
constructed via the BCK theory in the case that the operator
satisfies the stationary generalized KdV equations.}.
The partition function is again $Det\bar\p_W$ as usual, and 
the bottom row of the Baker-Akhiezer framing describes a 
holomorphic section of a holomorphic $m$-plane bundle 
over $\Sigma-P$. 

The theory of multiplicative 
Ward identities again holds, but with some interesting 
generalizations. 
the  transformations:
\eqn\genmt{
\eqalign{
\vec\psi(\zeta)&\longrightarrow Y(\zeta)\vec\psi(\zeta) \cr
\vec{\bar\psi}(\zeta)&\longrightarrow 
(Y^{-1})^{tr}(\zeta)\vec{\bar\psi}(\zeta) \cr}
}
preserve the action.
The multiplication by $Y$ 
will again change the boundary conditions (by the obvious 
multiplicative action of $Y(\zeta)$ on $H^m$) and if $Y$ 
has singularities then the $L^2$ condition on the fermi 
wavefunctions is modified in the neighborhood of these
singularities. Again,
we can replace these conditions by operator 
insertions. If $Y$ has poles or zeroes then the operators 
are simply products of $\psi_\zeta$, $\bar\psi_\alpha$
and their derivatives. If $Y$ has branch cuts then we 
have inserted twist operators $\varphi_L$. 
A rigorous discussion of construction of the associated
$\bar\p$ operator and its determinant for the insertion of 
twist operators on $\IP^1$ has been given in 
\ref\Palmer{J. Palmer, ``Determinants of Cauchy-Riemann operators
as $\tau$-functions,'' Univ. of Arizona preprint; ``The tau 
function for Cauchy-Riemann operators on $S^2$,'' unpublished letter
to C. Tracey.}.
Finally, 
if $Y$ has essential singularities associated to an
irregular singular point of a differential equation 
then we have inserted a star operator. In all cases we have
$$\langle V\dots\rangle_W\sim {Det\bar\p_{Y^{-1}\cdot W}\over
Det\bar\p_W} $$

Moreover, and quite generally, 
if we consider the multiplicative WI's for a 
correlation function of twist and star operators 
obtained by multiplying the Fermi fields by a function with 
a single zero and pole we obtain a ``discrete'' relation 
between the fermion two point function in the presence of 
these operators and a ratio of tau functions. 
We expect that one can explain
in conformal-field-theoretic terms all 
the relations obtained by use of 
``Schlesinger transformations'' in 
\Jimboii\ . (Schlesinger transformations themselves
can be viewed as replacing a twist operator by its normal
ordered product with $\psi,\bar\psi$.) 

From these remarks we see 
that the KdV flow is again induced by the infinite dimensional 
abelian group $\Gamma_+$ of \segal . In this case the flow has 
the interesting interpretation of being the flow of the renormalization 
group. 

\newsec{Conclusions}

main points. future directions.

I am very grateful to T. Banks, E. Br\'ezin, C. Crnkovic,
M. Douglas, V. Drinfeld, I. Frenkel, H. Garland, A. Its, 
M. Jimbo, V. Korepin, A. Morozov, D. Pickrell, 
D. Sattinger,
N. Seiberg, R. Shankar, 
S. Shatashvili, S. Shenker, and 
G. Zuckerman for helpful 
discussions. 
This work was supported by DOE grant DE-AC02-76ER03075. 
thanks cargese,kyoto,trieste


\listrefs
\bye



\ref\frsh{D. Friedan and S. Shenker, ``The analytic geometry of
two-dimensional conformal field theory,'' Nucl. Phys. {\bf B281}
(1987)509; D. Friedan, ``A new formulation of string theory,'' 
Physica Scripta T {\bf 15}(1987)72.}.
The conformal blocks of a correlation function 
are horizontal sections of
a flat vector bundle over the moduli space of 
curves. A horizontal section satisfies a differential 
equation which essentially follows from the idea that 
the stress energy tensor defines a connection 
on the bundle. If we discuss nontrivial (nonrational) 
conformal field theories, e.g., those associated 
with nonlinear sigma models with Calabi-Yau spaces as
targets, the flatness of the connection is the 
condition that the spacetime equations of motion 
are satisfied, i.e., that the appropriate generalizations
of Einstein's equations are satisfied. 

We propose that a similar picture holds in the case of 
$2D$ gravity. We begin by writing the string equations as
flatness conditions. These conditions are compatibility 
conditions for transport equations in a space parametrized
by $x,T_j$, the cosmological constant and the masses 
associated to the $2D$ gravity model. The parameters 
$x,T_j$, together with the initial conditions for 
the nonlinear differential equations known as the 
``string equations,'' are identified with the moduli 
of a certain class of meromorphic gauge fields on $\IP^1$. 
This moduli space is given a further interpretation in 
section five as a generalization of the moduli space 
of curves. The analogy to conformal field theory is 
developed further in section six where we interpret the transport
equations in $x,T_j$ as Knizhnik-Zamalodchikov-type equations for 
a free fermion construction of current algebra. The 
novel element is that the correlators in question involve
operators (dubbed ``star operators'') which are not normally
considered in conformal field theory. In section seven we 
suggest how one might establish a direct connection between the 
formalism of this paper and the random matrix formulation of 
$2D$ gravity.

It is well-known that the quantum field theory of free fermions
on a curve provides an elegant framework for understanding 
much of the theory of the quasiperiodic solutions of the 
generalized KdV hierarchies. Following some observations 
of Gross and Migdal \GM\ , Douglas emphasized the importance of the 
generalized KdV hierarchies in \newD\ . This led to the suggestion
\newD\bdss\ 
that the partition function of the matrix model might 
be a tau function in the sense of 
\ref\qpKdV{E. Date, M. Jimbo, M. Kashiwara, and 
T. Miwa, ``Transformations Groups for Soliton Equations,''
I. Proc. Japan Acad. {\bf 57A}(1981)342; II. Ibid., 387;III. J. Phys. Soc.
Japan {\bf 50}(1981)3806;IV. Physica {\bf 4D}(1982)343;V. Publ. RIMS, 
Kyoto Univ. {\bf 18}(1982)1111;VI. J. Phys. Soc. Japan {\bf 50}
(1981)3813;VII. Publ RIMS, Kyoto Univ. {\bf 18}(1982)1077.
}
While not strictly true, we show that this conjecture 
is essentially correct: the partition function of 2D gravity 
is given by the tau function for an isomonodromic deformation
problem
closely related to that of the stationary KdV equations.
The tau function in the quasiperiodic case admits an interesting 
interpretation as a function on an orbit of a loop group
\qpKdV
\ref\igor{I. Frenkel, ``Representations of affine Lie algebras,
Hecke modular forms, and Korteweg-de Vries type equations,'' 
Proceedings of the 1981 Rutgers Conference on Lie Algebras
and related topics. Lecture Notes in Mathematics
933, 71. Springer 1982.}\ 
, and it would be very interesting to find an analogous
interpretation in this case. 

In an effort to demonstrate that the above picture is not 
merely useless reinterpretation of known results we have shown in 
appendix A how the present formalism can be used to establish some
properties of the string equations which have recently 
become interesting in connection with the so-called 
``nonperturbative violation of universality'' in matrix models.

It has been repeatedly emphasized by Atiyah, Hitchin, Ward, and 
Witten that low-dimensional integrable differential equations and field
theories should be related to higher dimensional gauge theory. The 
four-dimensional self-dual Yang-Mills equations are expected to 
play a central role in such a formulation. In appendix B we sketch 
some connections between those ideas and the ideas of this paper.

After we completed most of this work we found that some
ideas similar to those of sections 2,3 and 5, in the 
context of the MKdV hierarchy and the associated PII 
equation, have been discussed by Flaschka and Newell
\ref\flasch{H. Flaschka and A. Newell, ``Monodromy and 
Spectrum-Preserving Deformations I,'' Comm. Math. Phys. 
{\bf 76}(1980)65.}. V. Korepin also pointed out to us 
some overlap between the remarks of section seven and those of 
\ref\Korep{A.R. Its, A.G. Izergin, V.E. Korepin, N.A. Slavnov,
``Differential Equations for Quantum Correlation Functions,''
Australian National University preprint; Trieste preprints, 
IC/89/120,107,139}. 
We have been informed by E. Martinec of similar
progress, especially in relating the gravity partition function 
to a tau function
\ref\martinec{E. Martinec, private communication.}.

\newsec{String Equations as Flatness Conditions}

Let us recall how M. Douglas wrote the general $(p,q)$ string 
equations in \newD\ . If $L=D^q+u_{q-2}D^{q-2}+\cdots + u_0$
is the continuum limit of a multiplication operator 
$f(\lambda)\to \lambda f(\lambda)$ on the orthogonal 
polynomials $f$ in a 
matrix chain model then, he argued, the continuum limit
of the conjugate derivative operator
$f(\lambda)\to {d\over d \lambda} f(\lambda)$ must be of the 
form $P=L^{p/q}_+$,
where the subscript indicates we keep only the differential
operator part of a pseudodifferential operator.
The nonlinear differential equations $[P,L]=1$ should define
nonperturbative 2D quantum gravity coupled to the $(p,q)$ 
minimal model of conformal field theory. Similarly, the 
equations for massive models coupled to 2D gravity are of the form
$[P,L]=1$ where $P=\sum_p t_p L^{p/q}_+$ and the $t_p$ 
are the ``masses'' in the theory. Our first task will be to 
rewrite these equations in first order matrix form.

{\it The $(2l-1,2)$ equations:}


The fact that a solution to $\sum(j+\half)T_jR_j=\hbar x$
satisfies the KdV flow in $T_j$ 
\bdss\  
is extremely surprising to those
familiar with the almost periodic solutions of KdV, where 
analogous parameters play the role of moduli of an associated 
Riemann surface, while the KdV flows are (straightline)
flows along the Jacobian 
of that surface. We will comment on this relation further below. 
For now we content ourselves with the following consistency 
check \bdss\  on \linsys\ , using the notation of Gelfand-Dickii
\Gelf .
Taking derivatives with respect to 
$x,T_k$ and assuming the KdV flow in $T_k$ we have
\eqn\consis{
\eqalign{
\hbar&=\sum (j+\half)T_jR_j'\cr
0&=(k+\half)R_k'-{1\over \hbar}\sum_j(j+\half)T_j\xi_jR_{k+1}'\cr}
}
where $\xi_j$ are the vector fields generating KdV flow \Gelf\ 
and we have used commutativity of the flows. The first equation 
implies $\hbar \delta/\delta u=-\sum_j(j+\half)T_j\xi_j$, and 
substitution into the second equation gives $0=(k+\half)R_k'+
(\delta/\delta u)R_{k+1}'$, a true identity.
This verifies consistency. One argument for the KdV flow 
was given in \bdss\ , we will give another argument below.
Note especially that the argument fails for $\hbar=0$. 

\bigskip
{\it The $(p,q)$ equations}

In this case we will be somewhat less detailed. Let 
$L=D^q+u_{q-2}D^{q-2}+\cdots$ and consider the generalized
KdV flow $[L^{p/q}_+,L]=dL/dt$. We may rewrite this, following 
Drinfeld-Sokolov, using the operator 
$$\CL=D+\Lambda+\pmatrix{0&\dots&-u_0\cr
                        .&\dots&.\cr
                        .&\dots&.\cr
                        .&\dots&.\cr
                       0&\dots&-u_{q-2}\cr
                     0&\dots & 0\cr}
$$
where $\Lambda$ has entries $1$ along the lower diagonal and 
$\lambda$ in the $1,q$ matrix element. Using the methods of 
\DrS\ one may construct
$\CA_{q,p}(\lambda)=\Lambda^p+\cdots$ such that the generalized
KdV flows are equivalent to 
$${\p\CL\over \p t}=[\CA(\lambda),\CL]$$
in particular, there are potentials $R_{q,p;i}$, $i=2,3,\dots q$
generalizing the $R_l$'s used above such that 
$$[\CA_{q,p},\CL]=\pmatrix{0&\cdots&-R_{q,p;q}'\cr
                     0&\cdots&-R_{q,p;q-1}'\cr
                     .&\dots&.\cr
                     .&\dots&.\cr 
                     0&\dots&-R_{q,p;2}'\cr
                     0&\dots&0\cr}
$$
Thus, as before, we may write the string equations as the 
flatness conditions $[\IB_{q,p},\CL]=0$ where 
$$\IB_{q,p}=\hbar {d\over d\lambda}+\CA_{q,p}+
\pmatrix{0&\cdots&\hbar x -R_{q,p;q}\cr
                     0&\cdots&\CH_{q-1}-R_{q,p;q-1}\cr
                     .&\dots&.\cr
                     .&\dots&.\cr 
                     0&\dots&\CH_2-R_{q,p;2}\cr
                     0&\dots&0\cr}
$$
and the $\CH_i$ are constants, analogous to the magnetic field of the
Ising model. We may formulate the equations for massive models 
in the obvious way by taking linear combinations of the 
$\IB_{q,p}$. 

\bigskip
{\it Unitary matrix models}



{\it General Semisimple Lie algebras}

In \DrS\ Drinfeld and Sokolov wrote analogues of the KdV
equations associated to any Lie algebra $g$. These are 
again of the form $\dot\CL=[\CP,\CL]$ where 
$\CL={d\over dx} + q + \Lambda$, $q$ is a function taking 
values in the Lie algebra and $\Lambda$ is a standard
element in the affine Kac-Moody algebra $\hat g$. When 
this algebra is realized as a loop algebra we may again 
modify $\CP\to \tilde\CP$ to obtain some equations of the
form $0=[{d\over d\lambda} + \tilde \CP , \CL]$ and these
should be the string equations for some matrix model
\foot{In \newD\ Douglas suggested they would be associated 
to the nondiagonal minimal models. This idea has
been studied in detail in 
\ref\kutas{D. Kutasov and Ph. Di Francesco, ``Unitary minimal models
coupled to 2D quantum gravity,'' Princeton preprint, PUPT-1173.}.
}. 
%MORE DETAIL.
%ARE THE MODIFIED EQS AGAIN SELF-SIMILAR?
%
%HAMILTONIAN STRUCTURE, W-ALGEBRAS

Flatness conditions arise very often in physics. The above 
interpretations suggest, e.g.,  that possibly one should think about 
a pure (holomorphic) Chern-Simons theory along the lines of 
\ref\witjones{E. Witten, ``Quantum Field Theory and the Jones Polynomial,''
Commun. Math. Phys. {\bf 121}(1989)351} 
with a suitable restriction on the fields. 
Such an interpretation yields a nice interpretation, e.g., of 
the $(p,q)$ actions of
\ref\paulact{P. Ginsparg, M. Goulian, M.R. Plesser,
and J. Zinn-Justin, ``$(p,q)$ String Actions,'' Harvard preprint 
HUTP-90/A015;SPhT/90-049}
in terms of Wilson loops. However it is difficult to see 
why the gauge symmetry should be broken. We will comment
again on this below.

\newsec{String Equations and Isomonodromic Deformation}

The compatibility conditions of the previous section arise naturally in 
a very interesting problem known as the isomonodromy problem.
The theory of isomonodromic deformation has been adequately 
reviewed in 
So we confine ourselves here to a 
very brief description of the method.

Consider a linear homogeneous differential equation 
\eqn\irreg{
{d\Psi\over d z}=A(z)\Psi
}
At an irregular singular point $a$ of order $r$ we can write
$$A(z)=\sum_{n=-r}^\infty A_n (z-a)^{n-1}\qquad .$$
Assuming $A_{-r}$ is diagonalizable one can show that 
there is a formal solution to \irreg\ of the form

\foot{We state this more carefully in the following 
section.}.
The analytic meaning of the formal solution \stoii\ is 
that we can divide up a neighborhood of $a$ into sectorial 
domains $\Omega_k=\{d_k<arg(z-a)<e_k\}$ for some constants 
$d_k,e_k$ such that in each domain there is a unique 
true solution $\Psi_k$ to \irreg\ which is asymptotic to 
\stoii\ . On $\Omega_{k+1}\cap
\Omega_k$ we have $\Psi_{k+1}=\Psi_k S_k$ for some Stokes matrices.
If the differential equation \irreg\ depends on parameters we 
can ask how we may change the parameters so that the ``monodromy 
data'' $S_k,L$ remains unchanged. As shown in the above 
references, such questions lead to interesting nonlinear differential 
equations. We now apply the general formalism of these works to 
the string equations.

\bigskip
{\it Asymptotic analysis}

Consider the massless $(2l-1,2)$ equation.
In order to perform the 
asymptotic analysis we follow \Jimboii\ 
so that $W$ satisfies the differential equation:
\eqn\dforz{\eqalign{
\hbar {d W\over d\zeta}&=\Biggl[(B_l+\zeta^2 C_l+\Delta_l)
\sigma_3-(B_l-\zeta^2
C_l+\Delta_l)i\sigma_2+(2\zeta A_l
-{\hbar \over 2\zeta})\sigma_1\Biggr]W\cr\cr
&=\biggl[(\zeta^{2l+2}+{u^2\over 8}\zeta^{2l-2}+\cdots)\sigma_3 +
(-{u'\over 4}\zeta^{2l+1}+\cdots)\sigma_1 
+ (-{u\over 2}\zeta^{2l}+\cdots)(-i\sigma_2)\biggr]W
\cr}  }
where $\Delta_l=\hbar x+2R_{l+1}$. Equation \dforz\ 
has an irregular singularity of order $2l+3$ at infinity
and a regular singularity at the origin. For the massive 
string equations we replace
$$\eqalign{
C_l\rightarrow C\equiv -\half\sum_j (j+\half)T_{j}C_{j-1}&
\qquad \qquad \Delta\rightarrow \Delta=\hbar x -\sum (j+\half)T_jR_j\cr
A_l\rightarrow \half C' &\qquad\qquad B_l\rightarrow(\lambda+u)C-A'\cr}
$$
In particular the PI equation is associated with 
the differential equation:
\eqn\dforzz{
\hbar {d W\over d\zeta}=\Biggl[(\zeta^4 + {u^2\over 8}+\hbar x)
\sigma_3-({u\over 2}\zeta^2+{u^2\over 8} + \hbar x )i\sigma_2
-(\zeta {u'\over 4}
+{\hbar \over 2\zeta})\sigma_1\Biggr]W }

Returning to the general case, 
the asymptotic expansion at infinity of a solution $W$ 
to \dforz\ is given by 
$W\sim \hat W e^{T/\hbar}$ where 
\eqn\asymp{
\eqalign{
T&=
\biggl(-{1\over 4}
\sum_{j=1} T_j \zeta^{2j+1} +\zeta \hbar x\biggr)
\sigma_3\cr
\hat W&=1+ {H_1\over \zeta}\sigma_3 -
{u\over 4\zeta^2}(i\sigma_2)+\CO(1/\zeta^3)\cr} }
%
%where 
%\eqn\asympii{\eqalign{
%\hat W_1&={u\over 4\zeta^2}+\cdots\cr
%\hat W_2&= \CO(1/\zeta^3)\cr }
%} 
%
To prove this we observe that we can rewrite $C$ 
in terms of the resolvent $R(x,\lambda)$
of the Schr\"odinger operator, 
$(-D^2 + u +\lambda)^{-1}$, used extensively
in \Gelf\ . In particular,
\eqn\res{C=p(\zeta;T_j)R(x,\zeta^2) + {\hb x- \Delta\over 2\zeta^2}
+\CO(1/\zeta^4) }
where $p=-\half \sum (j+\half)T_j\zeta^{2j-1}$. Defining
$\alpha=(2\zeta^2+u)R-\half R''$, $\beta=u R-\half R''$ and 
$\gamma=\zeta R'$ we find that 
\eqn\what{\hat W=\pmatrix{\zeta+\alpha&-\zeta+\alpha\cr
              \beta+\gamma&\beta+\gamma\cr} D}
for any diagonal matrix $D$ diagonalizes equation \dforz\ to 
order $\CO(1/\zeta)$. Equating positive powers of $\zeta$ 
we find 
$${\p T\over \p \zeta}=(\zeta p + x)\sigma_3$$
from which one obtains the first equation in 
\asymp. Using the diagonal freedom in defining $\hat W$ we 
can arrange that the expansion in $1/\zeta$ have the form of 
the second equation
\foot{Strictly speaking the $\CO(1)$ part of the differential 
equation with this substitution requires $\Delta=0$, which 
is a condition we will find later for isomonodromic 
deformation. A more tedious analysis shows that one can 
start without assuming $\Delta=0$ and derive the same 
result.}.

Near the origin we may diagonalize the regular singularity to 
be of the form $-\sigma_3/(2\zeta)$ so that, after a diagonalization 
the matrix near the origin behaves as $(1+\CO(\zeta))e^{-\half log
\zeta\sigma_3}$. True solutions with given asymptotic behavior 
will be linked by a connection matrix.
Actually, the full extent of the machinery for handling 
several singular points is not necessary. 



\bigskip
{\it Isomonodromic deformation}:

We now interpret the compatibility conditions of the previous section 
as the conditions required for isomonodromy under changes of 
$x,T_j$. In the literature \Jimboi\Jimboii\Jimboiii\Its
on isomonodromic deformation the Stokes parameters $s_j$
are referred to as monodromy parameters
\foot{One also considers the formal monodromy, 
i.e., the matrix $T_\infty$ multiplying the term with 
$log \zeta$ in $T$ as a monodromy paramter. In our problem 
we consider this parameter to be set equal to $1/2$ always.}.
The problem of isomonodromic deformation is 
the problem of finding
conditions on the $u,u_x,u_{xx},\dots$ (considered as independent
numbers) such that if we change $x,T_j$ the 
Stokes matrices remain fixed. We will give both a geometric 
and a physical description of the isomonodromy problem in 
subsequent sections. For the moment let us simply consider
the solution. Under 
a deformation of $x,T_j$ the solution $W$ of \dforz\ will become a 
smooth function 
of the $x,T_j$. 
Since $\Psi$ is regular throughout the $\lambda$ 
plane we know that the differential $d\Psi \Psi^{-1}$ is a 
matrix of holomorphic one-forms whose only (rational) singularity 
lies at infinity, and is thus uniquely determined by that 
singularity. Since the Stokes data are assumed to be 
$x$-independent we can
substitute the asymptotic expansion and
compute
\eqn\defii{
\eqalign{
{\p \Psi\over \p x}\Psi^{-1}&=\pmatrix{1&1\cr 1/\zeta&-1/\zeta\cr}
\biggl[\bigl(\zeta+H_1'/\zeta\bigr)\sigma_3+{u\over 2\zeta}
\pmatrix{0&-1\cr 1&0\cr}\biggr]
\half\pmatrix{1&\zeta\cr
1&-\zeta\cr}\ {\rm mod}(1/\zeta)\cr\cr
&=\pmatrix{0&\lambda+H_1'+(u/2)\cr 1&0\cr} \cr} 
}
Consistency with the $\lambda$ equation then forces $H_1'=u/2$. 
In terms of $W$ we have:
\eqn\defx{
{\p W\over \p x}
W^{-1}=\zeta \sigma_3 + {u\over 2\zeta}\pmatrix{1&-1\cr
1&-1\cr} } 
(If we had worked directly with 
$W$ then in computing $W_xW^{-1}$ we would have had to drop 
only terms of order $\CO(1/\zeta^2)$.)

Similarly, requiring that the Stokes parameters and 
the formal monodromy be independent of the 
$T_j$ leads to the equations:
\eqn\kflo{
{\p W \over\p T_j}W^{-1}=\Biggl[{1\over \zeta}{\p H\over\p T_j}
\sigma_3 -{1\over 4}\zeta^{2j+1}\hat W\sigma_3\hat W^{-1}\Biggr]{\rm mod}
1/\zeta^2 
}
Using \what\ , which holds to $\CO(\zeta^{-(2l+1)})$
we establish the third equation in \linsys .
The compatibility of the $x,T_j$ flows now shows that $u$ 
satisfies the KdV hierarchy.

\bigskip
{\it $\tau$ functions}
(As one may easily check directly from the 
differential equation, the coefficient of a second order pole 
must be $2$, implying that our $\tau$ function must be the 
square of a holomorphic function. This is naturally explained 
by the occurence of a pair of Weyl fermions in the free 
fermion formulation below.)

\bigskip
{\it Inverse Monodromy Problem and Initial Conditions}

As we have seen, under isomonodromic deformation 
the monodromy data are in fact independent of the 
parameters $x,T_j$. 
On the other hand, in order to construct 
the differential equation and solve explicitly for the 
monodromy data
we must begin with an actual solution 
$u(x;T_j)$ of the string equations. It turns out that
we may identify
the $2l$ initial conditions needed to define a solution of 
the string equations as the data needed to 
fix completely the differential equation.
Strictly speaking then, we should speak of the 
function $u(x;T_j,s_j)$ since all these data are needed to 
determine $u$. 

The problem of determining explicitly the 
data $s_i$ in terms of a solution $u$ (or in terms of its
asymptotics) is known as the direct monodromy 
problem, while the problem of finding the inverse relation, i.e.,
given the $s_i$ find $u(x)$ (in particular, its initial 
conditions)
is known as the inverse monodromy problem. These issues 
have been well-studied in the literature
\Its .
%
%One important lesson of the philosophy of the inverse monodromy
%problem is that the parameters $x,T_j$ and the initial conditions,
%or, equivalently, the $s_i$ should be considered as being ``on 
%the same footing.'' That is, in the proper formulation of the 
%IMP one asks if, given the $x,T_j,s_j$ there exist numbers
%$u,u_x,u_{xx},\dots$ for which the monodromy of \dforz\ gives 
%back the same numbers $x,T_j,s_j$. 
%\foot{
%This might have implications regarding 
%background independence in string field theory. 
%I would like to thank Tom Banks for pointing out 
%this connection to me.}
%
It is not always true that the inverse monodromy problem 
is solvable. For instance, an unfortunate choice of 
initial conditions in the form of the $s_i$ and the parameter
$x$ could put us on top of a pole for that function $u(x)$
determined by the $s_i$
\foot{The fact that such a pole is the only possible obstruction 
will be obvious, if not rigorous, from the free fermion 
point of view below.}.
The existence of this pole, 
whose physical implications have been widely discussed
\BK\bdss
\ref\newbrez{E. Brezin, E. Marinari, and G. Parisi, ``A Non-Perturbative
Ambiguity Free Solution of a String Model,'' preprint ROM2F-90-09}
\ref\david{F. David, ``Loop Equations and Non-Perturbative 
Effects in Two-Dimensional Quantum Gravity,'' preprint SPhT/90-043}\  
can be given a geometrical interpretation 
in terms of the nontriviality of a certain vector bundle
in the case of the PII equation 
\ref\itsnovok{A.R. Its and V. Yu. Novokshenov, ``Effective
Sufficient Conditions for the Solvability of the Inverse Problem
of Monodromy Theory for Systems of Linear Ordinary Differential
Equations,'' Funct. Anal. and Appl. {\bf 22}(1988)190.}.
We expect a similar statement will hold for the family of 
equations discussed here.


%
%Consider a punctured neighborhood $\Omega$ 
%of $\zeta=\infty$. Let us pass to 
%the double cover defined by $\zeta=\xi^2$. By analytic continuation 
%of one of the fundamental solutions we obtain a single valued
%solution $\Psi$ of the equation $\IP\Psi=0$. Consider a circle $C$ around 
%infinity. Restricting $\Psi$ to this circle defines a transition 
%matrix for a holomorphic vector bundle over $\IP^1$. The function 
%$u(x)$ has a pole when this vector bundle is nontrivial. 
%EXPLAIN IN DETAIL.
%

\bigskip
{\it The $(p,q)$ Equations}

The above discussion may be generalized to the entire set of 
$(p,q)$ string equations. Again we must diagonalize the leading 
singularity. To do this, choose a $q^{th}$ root of $\lambda$,
call it $\xi$ so that the other roots are $\xi_k=\omega^{k-1}\xi$
where $\omega=e^{2\pi i/q}$, $k=1,\dots q$. Then we take
$$\Xi\equiv\pmatrix{\xi_1^{q-1}&\dots&\xi_q^{q-1}\cr
                  \xi_1^{q-2}&\dots&\xi_q^{q-2}\cr
                   \cdot&\cdots&\cdot\cr
                   \cdot&\cdots&\cdot\cr 
                     \xi_1&\cdots&\xi_q\cr
                    1&\cdots&1\cr} 
$$
so that $\Xi^{-1}\Lambda\Xi=\Omega\equiv Diag\{1,\omega,\dots,\omega^{q-1}
\}$.
Therefore, defining $W(\xi)=\Xi^{-1}\Psi$ 
we obtain the equation
\eqn\gen{
-{dW\over d\xi}=\biggl(q\xi^{p+q-1}\Omega^p+\CO(\xi^{p+q-3})\biggr)W 
}
From the asymptotic expansion we find again that
$W=\hat W e^{T/\hbar}$ with 
\eqn\asmpq{\eqalign{
T&={-q\over p+q}\Omega^p\xi^{p+q}-\hbar \xi x \Omega\cr
\hat W&=1+{H_1\over \xi}\Omega^{-1} + {W_2\over \xi^2}+\cdots\cr} 
}
and for isomonodromic deformation we have $u_{q-2}=q H_1'$. It then 
follows that 
\eqn\taupq{\eqalign{
{d\over dx}log~\tau&=Res_{\xi=\infty} tr\biggl[\hat W^{-1}
{\p \hat W\over \p\xi}\bigl({\p T\over \p x}\bigr)\biggr]\cr
&=q H_1\cr}
}
so that 
$$u_{q-2}={\p^2\over \p x^2}log~\tau$$
is always true. Hence the only singularities for a solution to 
any of the string equations are second order poles.

%ADD MASS TERMS, FLOW IN H's? SEE OTHER OPERATORS
%
A similar analysis can be carried out for the unitary matrix models
\cdm\  .
In this case one finds that if $f$ is determined by \unmm\ 
then $f^2={\p^2\over \p x^2}log~\tau$. Since $f^2$ is the 
specific heat of the matrix model we see once again that the 
partition function is a tau function.


\newsec{The Moduli Space of the String Equations}

In this section we will give our first geometrical interpretation 
of the physical parameters determining the connection 
$\IP$, that is, the parameters $T_j$ 
and the initial 
conditions of the equation for $u(x)$. They will be seen to be 
moduli for a certain class of meromorphic gauge potentials.

Consider a meromorphic gauge potential $A_z(z)$ on the complex
plane. $A_z$ defines a connection on a (trivial) bundle over 
$\IC$. Consider the problem of classifying $A$ up to 
gauge equivalence. The resulting classification 
depends crucially on the nature of the singularities of $A$ and on 
the admissible class of gauge transformations. 
For example, in the 
local theory we take the possible (isolated) singularity
of $A$ to be at the origin. If $A$ has a regular singular 
point, i.e., if $A$ has a simple pole then,
up to equivalence under meromorphic gauge transformations,
$A$ is classified by the conjugacy class of the
monodromy $Pexp \int A$ around zero. This follows since
if we try to gauge $A$ to zero by solving the equation 
$d\Psi/dz= A \Psi$ then there is a solution of the form
$\Psi=\hat\Psi(z)z^M$ where $\hat\Psi$ is holomorphic
near zero, and
$e^{2\pi iM}$ is the monodromy matrix. Passing 
to the global theory of a connection on a vector bundle 
over a Riemann surface $\Sigma$, the solution to the famous
Riemann-Hilbert problem states that the moduli space of 
connections with regular singularities at $a_i$
are classified by the conjugacy classes
$Hom(\pi_1(\Sigma-\{a_i\}), GL(m,\IC) )/\sim$. 
Related facts
have been used 
extensively in investigations of Chern-Simons-Witten theory.

The situation is very different, and far more complicated, 
for connections with irregular singularities, i.e., with
higher order poles. 
We briefly summarize the situation, see
\nref\Wasow{W. Wasow, {\it Asymptotic Expansions for Ordinary 
Differential Equations}, Interscience, 1965.}
\nref\sibu{Y. Sibuya, ``Stokes phenomena,'' Bull. Amer. Math. Soc. 
{\bf 83}(1977)1075}
\nref\Mal{B. Malgrange, ``La classification des connexions irr\'eguliers
\'a une variable,'' in {\it Math\'ematique et Physique: S\'em
\'Ecole Norm. Sup.} 1979-1982, Birkh\"auser, 1983.}
\nref\Jur{W. Jurkat, {\it Meromorphe Differentialgleichungen}
Lect. Notes in Math. 637. Springer.}
\nref\VI{D. Babbitt and V. Varadarajan, ``Local Moduli
for Meromorphic Differential Equations,'' Bull. Amer. Math.
Soc. {\bf 72}(1985)95.}
\nref\VII{D. Babitt and V. Varadarajan, 
``Deformations of nilpotent matrices over rings and reduction of analytic
families of meromorphic differential equations,'' 
Mem. Amer. Math. Soc. {\bf 55}(1985)1}
\nref\VIII{V. Varadarajan, ``Recent Progress in differential equations in 
the complex domain,'' preprint}
\nref\Majima{J. Majima, ``Asymptotic Analysis for Integrable 
Connections with Irregular Singular Points,'' LNM 1075}\refs{\Wasow{--}
\Majima}\ 
for the full story. We follow the presentation of \VII\Majima\ .
A major result in the theory of differential equations, the
Hukuhara-Turittin formal reduction theorem,
states that if $A$ has an irregular singular point at $z=0$, 
we may 
solve $d\Psi/dz= A \Psi$ with $\Psi=\hat\Psi e^\Lambda$ 
where $\hat\Psi$ is a formal asymptotic series and 
\eqn\equivl{
{d\Lambda\over dz}=D_{r_1}z^{r_1}+D_{r_2}z^{r_2}+\cdots
D_{r_m}z^{r_m} + z^{-1} C}
where $r_i$ are rational numbers with $r_1<\cdots r_m <-1$,
$D_{r_i}$ are diagonal constant matrices and $C$ commutes 
with the $D_{r_i}$. (See, e.g., \Wasow ,
Theorem 19.1 for a careful statement and proof of this.)
Thus, a gauge transformation by 
$\hat\Psi$ transforms the connection to the above {\it canonical 
form}. Moreover, the numbers $r_i$ and the conjugacy class of 
$\{D_{r_i},exp[2\pi i k C]\}$ are gauge invariant. These are 
therefore the moduli of meromorphic gauge fields under
the orbits of formal gauge transformations. 
Applying the above observations to the class of gauge potentials 
defined by \dforz\ and its generalizations we see that 
the $\{T_j,x\}$ should be thought of as such moduli. 

If we now ask 
for the orbits under meromorphic gauge transformations, i.e.,
transformations with a convergent Laurent expansion near zero 
then, as shown in \sibu\Mal\Jur\VI\VII\ one must take into 
account a more subtle invariant.
This is the collection of Stokes matrices, or more properly, a 
Stokes cocycle, which we now describe
\foot{We skip over many technical details in what follows. 
The interested reader should consult the above references.}.
Suppose $A$ is in the formal 
equivalence class \equivl\ and $A$ gives rise to Stokes matrices 
$S_k$ in sectors around $z=0$. We can then consider the 
matrices $\CS_k(z)=exp(\Lambda(z))S_k exp(-\Lambda(z))$, defined
in each sector $\Omega_k$. These satisfy 
\eqn\stcy{\eqalign{
\CS_k\sim 1&\qquad {\rm in}\quad\Omega_k\cr\cr
{d\CS_k\over dz}&=\biggl[{d\Lambda\over dz} ,\CS_k
\biggr]\cr}
}
Let us now identify a sector $\Omega=\{\theta_1<arg z<\theta_2\}$
with the corresponding region on the unit circle. In this way 
we can define a sheaf 
\foot{of unipotent groups}, $st(\Lambda)$, 
over $S^1$ whose sections consist of 
matrices satisfying the two conditions in \stcy\ . 
The main theorem of Sibuya-Malgrange states that the nonabelian
sheaf cohomology group $H^1(S^1,st(\Lambda))$ is isomorphic to 
the moduli space 
under meromorphic gauge equivalence
of connections formally equivalent to \equivl\ . 

In the examples studied in this paper we may represent the 
Stokes cocycle by the matrices:
$$\pmatrix{1&e^{2T/\hbar}s_{2k+1}\cr 0&1\cr} , \qquad 
\pmatrix{1&0\cr e^{-2T/\hbar}s_{2k}&1\cr}$$
in $\Omega_{2k+2}\cap \Omega_{2k+1}$ 
and $\Omega_{2k+1}\cap \Omega_{2k}$
respectively. The moduli in this case are simply the 
Stokes parameters $s_i$ and hence the initial conditions
for the solution $u(x)$ to the string equations.
Hence
the physical data - the masses $T_j$ and the initial conditions, 
as expressed through the Stokes parameters $s_j$ are coordinates on 
the moduli space of a class of meromorphic gauge fields under 
meromorphic gauge equivalence. We think that it is an interesting 
question to elucidate what distinguishes 
potentials associated with integrable systems
in the space of all connections with irregular singular 
points. This is the same as the problem mentioned at the 
end of section 2. 


\newsec{Quantum Riemann Surfaces}

In this section we will examine more closely the geometry of the 
vector bundles defined in section three.
In particular we will be interested 
in the following question. It is well known that a pair of 
commuting differential operators $[P,L]=0$ leads, via the 
Burchnall-Chaundy-Krichever theory, to the construction of an 
algebraic curve $\Sigma$ with a line bundle
\foot{If $\Sigma$ is singular $\CL$ is not a line bundle
because of identifications 
at the singular points, instead it is a ``coherent sheaf.''
We will assume below that our Riemann surfaces are nonsingular.}
$\CL\to \Sigma$. The KdV flows are then just
straightline motion of $\CL$ along the Picard variety of $\Sigma$.
Since the string equations 
can be formulated as $[P,L]=\hbar$ one is naturally lead to ask 
if there is a sense in which one may associate to solutions of 
these equations a corresponding ``quantum'' geometry.
We will 
propose such a notion below, based on isomonodromic 
deformation
\foot{This is not to be confused with the fact that the stationary
KdV equations themselves can be written as an isomonodromic
deformation problem. We discuss this latter formulation of 
stationary KdV briefly in section six.}

Although the BCK theory has already been beautifully described in 
\ref\BMN{Dubrovin, Matveev, and Novikov, ``Non-Linear Equations 
of Korteweg-De Vries Type, Finite-Zone Liner Operators, and 
Abelian Varieties,'' Russian Math Surveys, {\bf 31} (1976)59}
\segal
\ref\mumf{D. Mumford, ``An algebro-geometric construction of 
commuting operators and of solutions to the Toda lattice equation,
Korteweg-de Vries equation and related non-linear equations,''
{\it Proceedings of Symposium on Algebraic Geometry}
M. Nagata, ed. Kinokuniya, Tokyo, 1978}
and elsewhere, we review it again from a slightly different 
point of view better suited to our purposes.

We work with the KdV hierarchy 
for simplicity, and will indicate the proper generalizations 
to the generalized KdV hierarchies later. The usual presentation 
of the BCK theory begins with the band theory of the 
Schr\"odinger operator $L=D^2-u(x)$. We consider the two-dimensional
eigenspace of $L$:
\eqn\eignsp{
V_\lambda=\{\psi(x)|L\psi(x)=\lambda\psi(x) \} 
}
Eigenfunctions can be shown to be meromorphic functions of the 
``momentum'' $z$ defined by $\lambda=z^2$. More precisely,
one introduces 
the Baker-Akhiezer (BA) eigenfunction 
which is uniquely characterized by 
a normalization condition $\psi(x_0,z)=1$, for some fixed
$x_0$, and the expansion near $z=\infty$
$$\psi(x,z)=e^{(x-x_0)z}\bigl[1 + {a_1(x)\over z} + {a_2(x)\over z^2}
+\cdots \bigr] $$ 
The BA function has a meromorphic extension to an affine hyperelliptic
curve $\Sigma_0$ double covering the affine $\lambda$ line $\IC$, and 
is a section of a (trivialized) line bundle $\CL\to\Sigma_0$. 
For our purposes it is more convenient to work with vector bundles on 
$\IC$ rather than line bundles on
$\Sigma_0$. 
The $V_\lambda$ define a trivial bundle over the affine 
$\lambda$-line. We will be concerned with 
various framings of this bundle, which are in one-one
correspondence with the invertible solutions of
\eqn\frm{\CL \Psi=\Biggl[-{d\over dx} + \pmatrix{0&\lambda+u\cr
1&0\cr}\Biggr]\Psi=0 }
since a solution is given by the Wronskian matrix
\eqn\frmi{
\Psi(x)=\pmatrix{\psi_1'&\psi_2'\cr 
\psi_1&\psi_2\cr} 
}
where $\psi_1,\psi_2$ are two linearly independent elements of $V_\lambda$.
A standard frame for this bundle is provided by the 
two solutions $\phi_{i}(x;x_0,\lambda)$, $i=1,2$, which satisfy
$\phi_{i}^{(j-1)}(x_0;x_0,\lambda)=\delta_{ij}$. The
$\phi_i$ are entire functions of $\lambda$.

Recall that
if $u(x)$ solves the stationary KdV equations:
\eqn\stKdV{
[P,L]=0 \qquad \qquad P=\sum c_l L^{(2l+1)/2}_+ }
then $P$ restricts to $V_\lambda$ and its restriction in 
the $\phi_i$ basis is a matrix which is polynomial in 
$\lambda$ and a differential polynomial in $u(x_0)$. 
In fact, letting $P_l=L^{(2l+1)/2}_+$ one may easily 
show that 
\eqn\act{\CP_l[u(x)]\Psi=\pmatrix{(P_l\psi_1)'&(P_l\psi_2)'\cr
P_l\psi_1&P_l\psi_2\cr} }
from which it follows that 
\eqn\pact{
P_l:\pmatrix{\phi_1\cr \phi_2\cr}\to\pmatrix{-A_l[u(x_0)]&B_l[u(x_0)]\cr
C_l[u(x_0)]&A_l[u(x_0)]\cr}\pmatrix{\phi_1\cr \phi_2\cr} 
}
The eigenvalues of $P_l|_{V_\lambda}$ thus satisfy the 
characteristic equation of this matrix, 
which defines an affine curve
$\Sigma_0$ by the equation 
$\mu^2=A^2+BC$. (Recall that $A,B,C$ are polynomials in 
$\lambda$ and differential polynomials in $x_0$. It is 
a simple consequence of \stKdV\ that the determinant 
$A^2+BC$ is independent of $x_0$.)
Consider a patch $U$ in $\IC$. If $U$ is simply connected and
does not contain any 
of the roots of $\mu(\lambda)$ then we may choose a framing which 
diagonalizes $P_l$ over $U$. The eigenspaces associated to $\pm \mu$
canonically define lines in the pullback bundle $\pi^*V$
where $\pi:(\mu,\lambda)\to \lambda$ is the projection.
We may choose, e.g., $\phi_{\pm\mu,\lambda}=C\phi_1 + (\pm\mu+A)\phi_2$.
(One can impose the normalization condition 
$\phi_{\mu,\lambda}(x_0)=1$.
In this case the framing will be meromorphic.)
At a branch 
point $\lambda_i$, where $\mu(\lambda_i)=0$ the two 
eigenspaces degenerate to a single line, so that we can 
say the $\mu$ eigenspace of $P_l$ at the point $(\mu,\lambda)$
defines a holomorphic line bundle $\CL\to \Sigma_0$. 
The situation is summarized in fig. 1
\foot{
The situation can also be summarized in the sheaf-theoretic 
formula:
$0\rightarrow \pi_*\CL\rightarrow V\rightarrow\oplus_P
S_P\rightarrow 0$
where $\pi_*\CL$ is the direct image and $S_P$ are sheaves supported on
branch points.}.
Finally, note that for large $\lambda$ the eigenfunctions must 
be plane waves,
$\sim e^{\pm xz}$,
and since $P$ is a differential operator of {\it  odd} order we must
have asymptotics $\phi_{\mu,\lambda}(x)\sim e^{\pm xz}$, with the 
choice of sign depending on $\mu$, so that these eigenfunctions 
are proportional to the BA functions $\psi(x,\pm z)$.
%explicitly: $\psi_1=\phi_1 + (C/\mu-A)\phi_2$,
%$\psi_2=\phi_1 - (\mu-A/B)\phi_2$ 

Let us define a {\it Baker-Akhiezer framing}
$\Psi$ to be a simultaneous solution of \frm\ and 
\eqn\bkfrm{
\CP\Psi=\Psi \sigma_3 \mu(\lambda)\qquad ,
}
which has the asymptotics
\eqn\bkasm{\Psi\sim \hat\Psi e^{(x-x_0)z\sigma_3} }
where $\hat\Psi$ has a power series expansion in $1/z$
near $\lambda\to\infty$, and is nonsingular except at the branch points
$\lambda_i$ and the zeroes of $C$.
For example, 
choosing $\psi_{1,2}=\psi(x,\pm z)$ in \frmi\  gives 
a {\it normalized} BA framing.
A BA framing is single-valued and meromorphic
on $\Sigma_0$ but not on the $\lambda$
line. Rather, upon analytic continuation around a branch point 
we have the monodromy 
\eqn\mndrmy{
\Psi\rightarrow \Psi\sigma_1 \qquad ,
}
reflecting the interchange of the two sheets.
Nevertheless, we can construct our riemann surface $\Sigma_0$ 
working completely 
with the framing for a bundle $V\to\IC$ and $\Sigma_0$ can 
be pictured  as ``sitting inside the total space of $V$,'' 
in the sense that if we consider the surface in $\IC\times V$
swept out by the tips of the frame we obtain a 
copy of $\Sigma_0$ as in fig. 1.
Since the data defining a BA framing is essentially the data
defining the famous quintuplets occuring in Krichever's
construction \segal\ we expect that one can define a
``moduli space of BA framings,'' which will be, essentially,
the universal Jacobian over the moduli space of curves.

We have phrased the BCK theory in such a way that we can 
now describe its noncommutative, or quantum, analog associated
to the equation $[P,L]=\hbar$. The main claim is that the
analog of 
the BA framing is a solution $\Psi(x,\lambda)$ to
\eqn\compa{
\eqalign{\CL\Psi&=0\cr
        \IP_l\Psi&=0\cr} }
and that in the $\hbar\to 0$ limit this framing reproduces 
the BA framing in the following sense. 

The gauge connection $(\CL,\IP_l)$ is flat, so an entire
solution of \compa\ exists for finite values of $x,\lambda$. 
Morally speaking, $\Psi$ defines a global framing of $``\pi^*V$''
where $\pi$ should be a projection from a ``quantum riemann surface.''
We can try to get a picture
of this object by studying the framing over the $\lambda$ line, in 
the semiclassical $\hbar\to 0$ limit. To this end let us 
consider a family of solutions
$u_\hb(x)$ to the string equations $\sum(j+\half)T_jR_j=\hb x$ 
which have a smooth $\hb\to 0$ limit. For example, in the case
of PI the 
asymptotic analysis of Boutroux shows that $u_\hb$ approaches
either a constant or the Weierstrass $\wp$ function.
In general we can write $u=u^{(0)}+\hb u^{(1)}+\cdots$. 
It will also be of interest to let the equation itself
vary with $\hb$, so we also let 
$T_i=T_i^{(0)} + \hb T_i^{(1)} + \cdots$. 
(Effectively, we 
are working over the ring $\IC[[\hb]]$.)
For such a family 
we can study the equation $\IP\Psi=0$, i.e.,
\eqn\wkay{
\hb{\p\Psi\over\p\lambda}=\biggl(-\half\sum_j(j+\half)T_j
\CP_{j-1}[u]\biggr)\Psi} 
via the WKB approximation using an analysis similar to that 
of appendix A. 
We first summarize the main conclusion of the analysis:
In the $\hbar\to 0$ limit the WKB turning 
points become the branch points $\lambda_i$
of the curve $\Sigma_0$ and a choice of Stokes ray emanating from 
each of these points becomes a branch cut for the affine hyperelliptic
curve $\Sigma_0$ defined by $\{(\mu_0,\lambda)\}$
where $\pm\mu_0$ 
are the eigenvalues of $\sum(j+\half)T_jP_{j-1}[u^{(0)}]$
acting on $V_\lambda$. 
In the various WKB sectors, and outside a neighborhood
(vanishing with $\hb$) of the branch cuts we may find 
solutions $\Psi_i$ of \compa\ so that if we define:
\eqn\classbak{
\Psi_i(\lambda)\equiv \tilde\Psi_\hb^{(i)}(\lambda)exp\biggl[
{\sigma_3\over\hb}\int^\lambda_{\lambda_i}\mu_0
(\lambda')d\lambda'\biggl]
\qquad ,
}
then $\tilde\Psi^{(i)}_\hb$
has a smooth classical limit, and is, in fact, a BA framing 
in the appropriate sector. 
Note that it immediately follows from the the smoothness of
$\tilde\Psi$ that if we 
substitute \classbak\ into \wkay\ then in 
the limit $\tilde\Psi_\hb\to\tilde\Psi$
the differential equation \wkay\ becomes the eigenvalue condition \bkfrm :
\eqn\eigag{
-\half\biggl(\sum(j+\half)T_j\CP_{j-1}[u^{(0)}]\biggr)
\tilde\Psi=
\tilde\Psi\sigma_3\mu_0(\lambda)
\qquad .
}
One surprise we find is that the BA framings defined in the 
different WKB sectors differ by multiplication on the right 
by nontrivial (i.e., nondiagonal) matrices, a reflection
of the Stokes phenomenon. 
%
%
%Moreover, while $\Psi_\hb$ is well-defined as a function of 
%$\lambda$,
%$\tilde\Psi_\hb$ is not. Upon analytic continuation of 
%$\lambda$ around a branch point the rapidly varying part of the 
%wavefunction has a {\it smooth} monodromy even in the 
%semiclassical limit:
%\eqn\monoi{
%exp\bigl[{\sigma_3\over\hb}\int^\lambda\mu(\lambda')d\lambda'\bigr]
%\longrightarrow
%\sigma_1
%exp\bigl[{\sigma_3\over\hb}\int^\lambda\mu(\lambda')d\lambda'\bigr]
%\sigma_1
%}
%and correspondingly $\tilde\Psi_\hb\to\tilde\Psi_\hb\sigma_1$, 
%which is preserved in the $\hb\to 0$ limit. 
%
Thus, heuristically speaking, we can 
picture a quantum riemann surface as 
a wildly fluctuating trivialization of a vector bundle 
which becomes smooth, and looks like a classical riemann surface,
in the neighborhood of the branch points as in fig. 2. 
The fluctuations are always directed along the line of the 
Krichever line bundle. 
Because of the Stokes phenomenon, even when we factor out 
the wild oscillations along the Krichever line bundle, we find 
that in different sectors of $\IC\times V$ the Riemann surface
as been ``rotated'' and ``sheared.''
\foot{The resulting object is thus somewhat
reminiscent of certain cubist paintings of Picasso and Braque.}
We now provide some justification
of this picture.
%
%
%Attempting to remove the 
%oscillating phase to obtain a smooth classical limit
%effectively introduces the monodromy which describes the 
%nontrivial topology of the covering surface. 
%We have attempted 
%to draw a picture of this in fig. (qrs).
%We now provide some justification of these statements.

\bigskip
{\it WKB Analysis}

We write \wkay\ as
\eqn\rwr{\hb{\p\Psi\over\p \lambda}=\bigl(\alpha\sigma_3+\beta\sigma_2+
\gamma\sigma_1\bigr)\Psi \equiv \CA\Psi\qquad .
}
where $\alpha=A$, $\gamma+i\beta=C$, $\gamma-i\beta=B$. 
Note that $\CA$ is $\CO(1)$ as $\hb\to 0$. The eigenvalues of 
$\CA$ are given by $\mu^2=\alpha^2+\beta^2+\gamma^2=A^2+BC[u_\hb]$
and approach $\mu_0^2$ smoothly in the classical limit.
Therefore, writing $\mu_0^2=\prod(\lambda-\lambda_i)$ we must define 
$3(2l+1)$ conjugate Stokes lines by the vanishing real part:
\eqn\sks{
\Re\int_{\lambda_i}^{\lambda}\mu_0(\lambda')d\lambda'=0
}
These are similar to the Stokes lines in appendix A. 
In this case the integrand vanishes as 
$\CO(\lambda-\lambda_i)^{1/2}$
at the points $\lambda_i$. Hence there are three lines 
emerging from each of these points at angles $2\pi/3$. 
We can get a picture
of the Stokes lines, at least, for some region of 
moduli space, by first considering the case in which 
$\mu^2=\prod_{i=1}^{2l+2}(\lambda-\lambda_i)$ with all the 
$\lambda_i$ real, with $\lambda_i<\lambda_{i+1}$. 
Then one can easily show that the Stokes lines 
have the form given in fig. 3 . The proof 
proceeds as follows. We know that as $\lambda\to \infty$ 
we must reproduce the full set of $2l+3$ Stokes rays 
going to infinity. Moreover, whenever two branch points
coalesce the triple joining points for Stokes lines 
must turn into quadruple joining points. Finally, 
the intervals on the real axis are certainly among 
the Stokes lines. The above picture is uniquely determined
by these criteria.
We may now imagine moving the branch points off the real 
axis. The picture will deform smoothly unless two branch 
points collide or unless a branch point hits a Stokes 
line. In that case we may expect fairly intricate 
phenomena which is outside the scope of this paper.
In general the lines will divide the plane into several
regions $\Omega_k$. 

Our first goal is to establish the following result.
In the interior of each region
$\Omega_k$, for $|\lambda-\lambda_i|\geq\CO(\hb^{2/3})$ 
there are solutions $\Psi_i(x,\lambda)$ to \compa\ 
such that 
$$\lim_{\hb\to 0}\Psi_i(x,\lambda)
exp\biggl[-
{\sigma_3\over\hb}\int^\lambda_{\lambda_i}\mu_0
(\lambda')d\lambda'\biggl]
$$
exists and is a BA framing. 

To prove this, recall that the usual theorem in asymptotic 
analysis guarantees that there will be a solution $\Psi_i$ in 
$\Omega_i$ with the asymptotic expansion
$$\Psi_i\sim L_i\bigl(1 + \hb N^{(1)}_i+\cdots\bigr)
exp\biggl[
{\sigma_3\over\hb}\int^\lambda_{\lambda_i}\mu_\hb
(\lambda')d\lambda'\biggl]$$
where 
\eqn\diag{\CA L_i=L_i\sigma_3\mu_\hb \qquad ,}
and, if we expand
$L_i=L_i^{(0)}+\hb L_i^{(1)}+\cdots$ then we require:
$$Diag\bigl\{(L_i^{(0)})^{-1}{\p L_i^{(0)}\over\p 
\lambda}\bigr\}=0 \qquad .$$
Therefore, we study the difference of the WKB phase factors:
\eqn\difaze{
F_i(\lambda,x;T_j)\equiv \lim_{\hb\to 0}\Biggl[
{1\over\hb}\int^\lambda_{\lambda_i}\mu_\hb
(\lambda')d\lambda' -
{1\over\hb}\int^\lambda_{\lambda_i}\mu_0
(\lambda')d\lambda'\Biggr]\qquad .
}
If $\mu^2$ has a simple zero then it is easy to see the limit
exists. 

Since the most general BA framing differs from 
the normalized BA framing by right-multiplication by a 
diagonal matrix depending only on $\lambda$ and 
having power series asymptotics at infinity we can establish 
the result by finding a suitable function $f_i(\lambda)$ such that
$$L_i^{(0)}=\pmatrix{\phi_{\mu,\lambda}'&\phi_{-\mu,\lambda}'\cr
\phi_{\mu,\lambda}&\phi_{-\mu,\lambda}\cr}
\pmatrix{f_ie^{-F_i}&\cr
&f_i^{-1} e^{F_i}\cr}
$$
where we normalize the first matrix to have unit Wronskian.
We determine $f_i$ by 
\eqn\corff{
{\p\over\p \lambda}log f_i={\p F_i\over\p \lambda}
-\bigl(\phi_{-\mu,\lambda}{\p\over\p\lambda}\phi_{\mu,\lambda}'
-\phi_{-\mu,\lambda}'{\p\over\p\lambda}\phi_{\mu,\lambda}
\bigr) }
Differentiating the above with respect to $x$ we see that 
since
$${\p\over\p x}F_i=\int_{\lambda_i}^\lambda {C(x,\lambda')\over
2\mu_0(\lambda')}d\lambda'$$
and $\phi_{-\mu,\lambda}\phi_{\mu,\lambda}=C/2\mu_0$ the rhs 
of \corff\ is 
$x$-independent, as it must be. Thus we can solve \corff\ for 
a function of $\lambda$ only.

We may establish the asymptotics of $f_i$ by using the observation of 
section three that $C=p(\lambda,T_j)R(x,\lambda)+{\hb x/2\lambda}
+\CO(1/\lambda^2)$, where $R(x,\lambda)$ is the resolvent of 
the Schr\"odinger operator, to find 
$\mu_\hb=\half[p+\hb x/z +\CO(1/z^3)]$
so that
\foot{From \asymp\ we see that we are simply verifying 
the fact that the asymptotics in $\hb$ and $\zeta$ is 
double.} 
\eqn\asympfi{F_i(x,\lambda,T_j)=-{1\over 4}\sum_j T_j^{(1)}z^{2j+1}
+ x z+\CO(1/z)\qquad .}
Note the occurence of the parameters $T_j^{(1)}$. 
Since $u$ satisfies KdV flow in $T_j$ we know from the standard
theory that the BA functions have asymptotics
$$\phi_{\mu,\lambda}={1\over \sqrt{2z}}e^{xz-{1\over 4}
\sum T_j^{(1)}z^{2j+1}}
\biggl(1+{a_1(x,T_j)\over z}+\cdots \biggr)$$
from which it follows that $\p/\p z(log f_i(z))=\CO(1/z^2)$,
which completes the proof.

The different solutions $\Psi_i$ with WKB asymptotics in 
the various sectors $\Omega_i$ will not agree on the overlaps 
of sectors because of the Stokes phenomenon at the irregular 
singular point at infinity. There is further disagreement from 
the conditions for matching to the well-defined solutions at 
the turning points $\lambda_i$, as we now describe.

For $|\delta\lambda_i|
=|\lambda-\lambda_i|\leq \CO(\hb^{2/3-\epsilon})$
we can expand
\eqn\turni{
\hb{\p \Psi\over\p\lambda}=\biggl(\CA_i+\delta\lambda_i\dot \CA_i
+\CO(\hb,(\delta\lambda_i)^2)\biggr)\Psi
}
and the determinant $\mu^2=\dot\mu_i^2\delta\lambda_i$. 
Defining a variable $\eta=\hb^{-2/3}\dot\mu_i^{2/3}\delta\lambda_i$
and $\tau=\hb^{1/3}\dot\mu_i^{2/3}$, 
one easily shows that the WKB solution has asymptotics
\eqn\turnii{
({C_i\over 2\tau})^{1/2}\eta^{-1/4}\pmatrix{1/C_i&A_i/C_i\cr
0&1\cr}\pmatrix{\tau\eta^{1/2}&-\tau\eta^{1/2}\cr
1&1\cr}exp\bigl[{2\sigma_3\over 3}\eta^{3/2}\bigr]
}
up to right-multiplication by a diagonal constant matrix.
On the other hand, after transforming \turni\ by the first 
matrix in \turnii\ we can obtain an exact solution in 
terms of Airy functions:
\eqn\airy{\Psi=\pmatrix{\tau Bi'(\eta)&\tau Ai'(\eta)\cr
Bi(\eta)&Ai(\eta)\cr} }
We can now use the standard asymptotic expansion of these 
functions 
\ref\abram{M. Abramowitz and I. Stegun, {\it Handbook of 
Mathematical Functions}, Dover}:
\eqn\asympariy{
\eqalign{
Ai(\eta) \sim & {1\over 2\sqrt{\pi}}\eta^{-1/4}e^{-2 \eta^{3/2}/3}
(1+\CO(\eta^{-2/3}))\qquad\qquad -\pi< arg \eta< \pi\cr
%Ai'(\eta)\sim {-1\over 2\sqrt{\pi}}\eta^{1/4}e^{-2 \eta^{3/2}/3}&
%(1+\CO(\eta^{-2/3}))\qquad\qquad -\pi< arg \eta< \pi\cr
Bi(\eta)\sim &{1\over \sqrt{\pi}}\eta^{-1/4}e^{2 \eta^{3/2}/3}
(1+\CO(\eta^{-2/3}))\qquad\qquad -\pi/3< arg \eta< \pi/3\cr
%Bi'(\eta)\sim {1\over \sqrt{\pi}}\eta^{1/4}e^{2 \eta^{3/2}/3}&
%(1+\CO(\eta^{-2/3}))\qquad\qquad -\pi/3< arg \eta< \pi/3\cr
Bi(\eta)\sim &{1\over \sqrt{\pi}}\eta^{-1/4}
\bigl[e^{2 \eta^{3/2}/3}+{i\over 2}e^{-2 \eta^{3/2}/3}\bigr]
\qquad\qquad -\pi/3< arg \eta< \pi\cr
%Bi'(\eta)\sim {1\over \sqrt{\pi}}\eta^{1/4}
%\bigl[e^{2 \eta^{3/2}/3}-{i\over 2}e^{-2 \eta^{3/2}/3}\bigr]&
%\qquad\qquad -\pi/3< arg \eta< \pi\cr\cr
Bi(\eta)\sim &{1\over \sqrt{\pi}}\eta^{-1/4}
\bigl[e^{2 \eta^{3/2}/3}-{i\over 2}e^{-2 \eta^{3/2}/3}\bigr]
\qquad\qquad -\pi< arg \eta< \pi/3\cr
%Bi'(\eta)\sim {1\over \sqrt{\pi}}\eta^{1/4}
%\bigl[e^{2 \eta^{3/2}/3}+{i\over 2}e^{-2 \eta^{3/2}/3}\bigr]&
%\qquad\qquad -\pi< arg \eta< \pi/3\cr\cr
} }
We see that to match onto a well-defined framing near
$\lambda_i$ the standard solutions $\Psi_i$ must
be multiplied by the nontrivial lower-diagonal matrices
$$\pmatrix{1&0\cr
\pm i/2 & 1\cr}\qquad . $$
Thus, well defined framings lead to fragmented bits of 
Riemann surface, and well-defined Riemann surfaces lead to 
discontinuous jumps in framings. 
This completes the justification of the picture mentioned 
above.

We expect that the above discussion can be generalized easily
to Riemann surfaces which are $n$-fold coverings of the plane.
In this case the Schr\"odinger operator is replaced by 
the operator $L=D^n+\cdots$. One can still define the notion of a
BA framing of an $n$-plane bundle on the $\lambda$ line and so on. 
We also remark that in the above analysis we have occasionally 
made use of the fact that the point at infinity is a Weierstrass 
point, but this is probably not essential.


\bigskip
{\it Quantum Moduli Space}
 
As $\hb\to 0$, the parameters $T^{(0)}_i$ 
define the moduli of the classical surface $\Sigma_0$. 
On the other hand, if we allow the $T_i$ to vary: 
$T_i=T_i^{(0)}+\hb T_i^{(1)} +\cdots$ then the $T_i^{(1)}$ 
become coordinates on the Jacobian of $\Sigma_0$,
as we can see from \linsys\ as well as from \asympfi\ .
Thus, we propose that one should regard the space of 
$(x,T_j)$ as a generalization of the universal Jacobian 
over moduli space. This suggests several interesting questions. 
For example, is there an analog of the modular group?
Can we study limits like $T\to \infty$ using known 
results about the boundary of the moduli space of curves?
And so forth.

Finally, note that another approach to our problem would be
to use noncommutative geometry in the sense of Connes. One
might begin with the Heisenberg algebra $[P,L]=\hb$ and 
look for interesting ideals which would allow us to define 
the ``ring of functions on the noncommutative Riemann surface,''
and so on. It would be interesting to see if such ideas are
related to the above pictures.


\newsec{Further Speculations}

Since all known matrix models are described by nonlinear
equations associated to isomonodromic deformation one might 
wonder if this is not a general feature. Possibly 2D 
gravity should be formulated as a conformal field theory 
(on auxiliary riemann surfaces!) in the presence of a new
class of operators. This raises many 
questions. The Riemann surfaces discussed in this 
paper have no obvious connection to the world sheet of 
the 2D gravity model. The latter emerges from the 
Feynman diagram expansion of a theory of a Hermitian 
matix $\phi_{ij}$. The surfaces in this 
paper are coordinatized by the spectral parameter $\lambda$,
interpreted in terms of the eigenvalues of $\phi_{ij}$. 
Nevertheless,
it is difficult to believe that the surfaces are unrelated.
Indeed, it was suggested by A. Morozov and S. Shatashvili 
that 2D gravity should be thought of in terms of a 
nonconformal field theory at genus zero, the effects of 
the sum over topologies effectively spoiling the conformal 
invariance 
\ref\samson{A. Morozov and S. Shatashvili, private communications,
Nov. 1989, and Dec. 1989}. We have arrived at a picture
strikingly similar to their guess.
Possibly the auxiliary surface is, in some sense, a
picture of an effective world-sheet, with the star
operator reproducing the effect of a ``condensation of 
handles.'' (This would raise the spectre of whether 
we should consider the role of
nontrivial topologies of the auxiliary Riemann surface itself.)
Continuing this line of thought, 
it is conceivable that extending 
the above picture to conformal field theories other 
that those of free fermions might correspond to 
exotic types of gravity, or perhaps to $c>1$ theories.
For example we can consider star operators in WZW theories. 
As we saw, the isomonodromic deformation equations can 
be interpreted as KZ equations, but these equations were
at level $1$. It would be interesting to write down 
and study the higher level equations. The
free field realizations of WZW might facilitate the 
construction of star-like operators in these models.

It would also be very interesting to understand if 
there is a connection between the above ideas and 
the topological field theory approach of 
\witt\DiWit\distler\newvsq .
In the latter approach the topological origin of 
KdV flows has proven elusive. In the present 
approach they are interpreted as flows defined by 
Knizhnik-Zamalodchikov-type equations on a certain 
moduli space. Perhaps some kind of asymptotic expansion 
of a star operator in negative powers of $x$ will
provide the missing link between these formalisms.

Be all this as it may, our work has implications 
beyond the framework of 2D gravity.
We have been led to a generalization of modular geometry.
Firstly, one should include the notion of star operators
in general conformal field theories. Thus, the basic 
monodromy data of such a theory will include not only 
matrices representing the braid group and operator 
product expansions (``$B$ and $F$ matrices'')  but also 
Stokes matrices occuring in more exotic exchange algebras 
than have been hitherto examined. We might therefore 
expect that these monodromy data will define some generalization 
of a modular tensor category
\ref\ms{G. Moore and N. Seiberg, ``Lectures on RCFT,'' preprint
RU-89-32; YCTP-P13-89}.
Secondly, as we have described, the introduction of star 
operators leads to a notion of quantum Riemann surface and 
quantum moduli space. Just as there are analytical and 
topological aspects to modular geometry 
(the latter corresponding to the the modular functor point of view), 
one might expect that the analytic aspects associated to 
this generalized MTC would be related to some geometry of 
quantum moduli space. The geometric category representing 
this extended notion of a MTC might well be useful 
in understanding the analog of modular
geometry for integrable but nonconformal models.
From what we have managed to glean of the geometrical meaning 
of this extension we may expect to find a rich but
peculiar combination of algebraic and analytic geometry.


\bigskip
\centerline{\bf Acknowledgements}


\appendix{A}{An Application: The BMP Solutions}


\appendix{B}{Twistor Correspondence}

Mason and Sparling 
\ref\mas{Mason and Sparling, ``Nonlinear Schrodinger and 
Korteweg-De Vries are reductions of self-dual Yang-Mills,'' 
Phys. Lett. {\bf 137A}(1989)29.}
have shown that solutions to the KdV 
equations are in one-one correspondence with holomorphic 
vector bundles over minitwistor space ($=T\IP^1$) which 
posess an extra symmetry. Since solutions to the string
equations define, in particular, solutions to the KdV equations 
we may apply the observation of 
\mas\ to associate corresponding holomorphic vector bundles over 
$T\IP^1$. 
%The question is: what distinguishes these solutions of 
%the KdV geometrically?    
%
%As shown in \mas\ the holomorphic vector bundle can 
%be characterized by defining its space of sections as follows.

Consider the equations:
\eqn\linsysi{\eqalign{
%\IP \Psi(\lambda,x,T_j)&=0\cr
\CL \Psi(\lambda,x,T_1)&=0\cr
(-2{\p\over \p T_1}+\CP_1)\Psi(\lambda,x,T_j)&=0\cr} }
which are compatible when $u(x,T_1)$ satisfies the KdV equation.
By taking a linear combination of these conditions and 
changing the framing by 
$$\Psi\to\pmatrix{1&0\cr
                  H&1\cr}\Psi$$
where $H'=u$ we obtain the linear system which can be 
regarded as the twice dimensionally-reduced system 
equivalent to the SDYM's equations in a space of signature
$(2,2)$ \mas .  This is similar to the relation to inverse 
scattering theory pointed out in 
\ref\Belav{A. Belavin and V. Zakharov, ``Yang-Mills Equations as 
Inverse Scattering Problem,'' 
Phys. Lett. {\bf 73B}(1978)53}.
Using standard twistor methods these can be regarded as 
defining a holomorphic structure on a two-dimensional 
complex vector bundle over $T\IP^1$. 
(In brief, pulling back the connection to 
$\IP^3(\IC)-\IP(\IC)$ via the standard twistor fibration,
the self-duality condition becomes an integrability 
condition for a holomorphic 2-plane bundle, and dividing out
by the lift of the symmetry $\IR^4\to\IR^4/\IR=\IR^3$ brings
us down to minitwistor space $T\IP^1$. See 
\ref\atiyah{M.F. Atiyah, {\it Collected Works}, vol. 5, Clarendon 
Press, Oxford, 1988}
for details.) In this paper a crucial role was played by the 
``equation in $\lambda$,'' $\IP\Psi=0$. Such equations arise 
naturally in the twistor construction - essentially they embody
the statement that the bundle on twistor space is the pullback of 
a bundle on spacetime.

%A very direct construction of solutions to the bogomolnyi 
%equations is by solving the scattering problem along the 
%line. describe mason-sparling's extra symmetry, fixes sphere 
%but acts nontrivially in fibers. 
%
%
%Thus the Baker framing $\Psi$ can also be 
%regarded as a framing for a holomorphic vector bundle on 
%($T\IP^1$ ?) The key point is that $\Psi$ also satisfies 
%a differential equation in $\lambda$. SHOW: This equation 
%effectively expresses the fact that a unitary bundle has been 
%pulled back from spacetime up to twistor space.
%
%%is x on the tangent space?-no, x is not a coordinate in minitwistor
%space. complexified x is coordinate on the space of sections of 
%%tangent bundle
%
%what distinguishes geometrically the stationary solutions of KdV?
%(these can also be written as isomonodromy!)
%role of tau function in twistor theory?
\vfill\eject
\bigskip
\centerline{\bf Figure Captions}

\noindent
Fig. 1. A schematic drawing of the Riemann surface and the Krichever
line bundle defined by 
the BA framing, in the neighborhood of a branch point.
\bigskip
\noindent
Fig. 2. A schematic drawing of a framing defined by the equation in 
$\hb$ in the neighborhood of a branch point. The oscillations are along 
the directions in $V_\lambda$ defined by the framing in fig. 1. 
\bigskip
\noindent
Fig. 3. An example of conjugate Stokes lines in the case where 
the branch points are all real. In this case the surface has genus
two. 
\bigskip
\noindent
Fig. 4. A schematic view of a choice of contours for the star operator.
\bigskip
\noindent
Fig. 5. A simple Fermi sea. 
\bigskip
\noindent
Fig. 6. Three large regions abutting infinity and a turning 
point $\xi_i$. 
\bigskip
\noindent
Fig. 7. Stokes lines for the case $l=1$, $u\sim +(-x)^{1/2}$
for $x\to-\infty$. We have also indicated the Stokes parameter 
associated with each region.
\bigskip
\noindent
Fig. 8. Stokes lines for $l=1$, $u\sim -(-x)^{1/2}$, $x\to-\infty$.
\bigskip
\noindent
Fig. 9. Stokes lines for $l=2$, $u\sim x^{1/3}$, $x\to+\infty$.
\bigskip
\noindent
Fig. 10. Stokes lines for $l=2$, $u\sim x^{1/3}$, $x\to-\infty$.
\bigskip
\noindent
Fig. 11. Stokes lines for $l=4$, $u\sim x^{1/5}$, $x\to+\infty$.
\bigskip
\noindent
Fig. 12. Stokes lines for $l=4$, $u\sim x^{1/5}$, $x\to-\infty$.



\listrefs
\bye




