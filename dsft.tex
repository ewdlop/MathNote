

 
\input harvmac.tex
\def\CD {{\cal D}}
\def\CF {{\cal F}}
\def\CP {{\cal P }}
\def\CL {{\cal L}}
\def\CV {{\cal V}}
\def\CO {{\cal O}}
\def\p {\partial}
\def\CS {{\cal S}}
\def\hb {\hbar}
\def\inbar{\,\vrule height1.5ex width.4pt depth0pt}
\def\IB{\relax{\rm I\kern-.18em B}}
\def\IC{\relax\hbox{$\inbar\kern-.3em{\rm C}$}}
\def\IP{\relax{\rm I\kern-.18em P}}
\def\IR{\relax{\rm I\kern-.18em R}}
 
 
\Title{\vbox{\baselineskip12pt\hbox{YCTP-P8-91}\hbox{RU-91-12}}}
{\vbox{\centerline{Double-Scaled Field Theory at $c=1$}}}
 
\centerline{Gregory Moore}
\centerline{Department of Physics and Astronomy}
\centerline{Rutgers University}
\centerline{Piscataway, NJ 08855-0849}
\centerline{and}
\centerline{Department of Physics\footnote{*}{Permanent address}}
\centerline{Yale University}
\centerline{New Haven, CT 06511-8167}
\bigskip
\noindent

We investigate the double-scaled free fermion theory of the 
$c=1$ matrix model. We compute correlation functions of the 
eigenvalue density field and compare with the predictions of 
a relativistic boson theory. The $c=1$ theory behaves as a 
relativistic theory at long distances, but has softer behavior
at short distances. The soft short distance behavior is closely
related to the breakdown of the topological expansion
at high energies. We also compute macroscopic loop amplitudes 
at $c=1$, finding an integral representation for $n$-
loop amplitudes to all orders of perturbation theory. We evaluate
the integrals explicitly for two, three, and four macroscopic 
loops. The small loop length asymptotic expansion then 
gives correlation functions of local operators in the theory.
The two-macroscopic-loop formula gives information on the spectrum
and wavefunctions in the theory. The three and four loop 
amplitudes give scattering amplitudes for tachyon operators 
to all orders of 
perturbation theory. Again, the topological expansion 
breaks down at high energies. We compare our amplitudes with 
predictions from the liouville theory.

\Date{Feb. 20, 1991}

\newsec{Introduction}

The discovery of the double scaling limit of matrix models 
has opened the way for a study of nonperturbative effects in 
subcritical string theory
\ref\brkz{E.Br\'ezin and V. Kazakov, Phys. Lett. {\bf 236B}(1990)144.}
\ref\dgsh{M.R. Douglas and S. Shenker, Nucl. Phys. {\bf B335}(1990)
635.}
\ref\grmgdl{D. Gross and A. Migdal, Phys. Rev. Lett. {\bf 64}(1990)127}\ .
The exact solvability of the 
the $c<1$ models is closely connected to the equivalence of
the relevant matrix models to theories of free fermions
\ref\bpiz{E. Br\'ezin, C. Itzkyson, G. Parisi, and J.-B. Zuber,
Comm. Math. Phys. {\bf 59}(1978)35.}
\ref\bdss{T. Banks, M.R. Douglas, N. Seiberg, and S. Shenker, 
Phys. Lett. {\bf 238B}(1990)279} .
In the $c<1$ models the 
free fermions provide an efficient calculational tool
but their relevance goes much
deeper. First, the connection of the $c<1$ models to integrable systems 
established in \brkz\dgsh\grmgdl\bdss\
\ref\mike{M.R. Douglas, Phys. Lett. {\bf 238B}(1990)176}
may be understood very naturally in terms of the theory of
the double-scaled free fermion field
\ref\geom{G. Moore, Comm. Math. Phys. {\bf 133}(1990)261;
``Matrix Models of 2D 
Gravity and Isomonodromic Deformation,'' to appear in the proceedings
of the 1990 Yukawa International Seminar, and in the proceedings
of the Carg\'ese Workshop on Random Surfaces}\ . Second, 
there are many indications that the double-scaled 
free fermion field theory will play a role in formulating a natural 
spacetime interpretation of the $c<1$ models.
The 
fermionic field theory for the $c=1$ matrix model has been explored to some 
extent in 
\ref\senwad{A.M. Sengupta and S.R. Wadia, ``Excitations and interactions
in d=1 string theory,'' Tata preprint 90-33, July 1990. To 
appear Int. Jour. of Mod. Phys. A.}
\ref\gk{D.J. Gross and I.R. Klebanov, ``Fermionic String Field 
Theory of c=1 Two-Dimensional Quantum Gravity,'' PUPT-1198}\ ,
but these treatments have made important approximations
limiting their applicability to genus zero physics. 
Consequently, the full potential 
of the fermionic formulation has not yet been exploited in 
the study of $c=1$ string physics. 

In this paper we continue the study of the $c=1$ 
free fermions in the double-scaling limit.
In section two we derive the 
theory and comment on the relation of the fermi field 
to a relativistic fermi field. In section three we compute
correlation functions of the eigenvalue density field, and 
compare with the results of the Das-Jevicki formulation
\ref\dsjv{S.R. Das and A. Jevicki, ``String field theory and physical 
interpretation of D=1 strings,'' Brown preprint BROWN-HET-750}\ 
of $c=1$. In section four we compute macroscopic loop amplitudes 
and use these to define scaling operators. From explicit 
expressions we obtain information about
 the spectrum of the theory, the  
wavefunctions for operators in the theory, and some 
scattering amplitudes.
For other approaches to $c=1$ scattering amplitudes see
\nref\kostov{I. Kostov, Phys. Lett. {\bf B215}(1988)499}
\nref\bmh{S. Ben-Menahem, ``Two and three-point functions in 
the $d=1$ matrix model,'' SLAC-PUB-5262}
\nref\grkltwpt{D.J. Gross, I.R. Klebanov, and M.J. Newman,
``The two-point correlation function of the one-dimensional
matrix model,'' PUPT-1192}
\nref\JOE{J. Polchinski, ``Critical Behavior of Random Surfaces in 
One Dimension,'' UTTG-15-90}\ 
\nref\kutasov{D. Kutasov and P. Di Francesco, ``Correlation Functions
in 2D String Theory,'' PUPT-1237, to appear. We thank D. Kutasov for 
communicating the results of this paper prior to publication.}
\refs{\kostov{--}\kutasov}\ .
The conclusion summarizes some of the 
physical lessons which may be learned from our computations.


\newsec{The double-scaled field theory at $c=1$.}
 
\subsec{Derivation}

We begin with the standard matrix model integral at finite
$N$
\ref\kazrv{For a recent review see V. Kazakov, ``Bosonic 
strings and string field theories in one dimensional target
space,'' to appear in the proceedings of the Carg\'ese workshop
on random surfaces and quantum gravity.}\ .
For definiteness we
choose a potential $V(\lambda)=\half(A^2-\lambda^2)$ with 
infinite walls at $\lambda=\pm A$. Dependence on $A$ 
reflects dependence on a nonuniversal cutoff. We will see
below that all dependence on $A$ disappears in the 
double scaling limit. The schr\"odinger equation
is
\eqn\scrho{
\biggl({1\over 2\beta^2}{d^2\over d\lambda^2}-V(\lambda)+
\varepsilon\biggr)\psi=0}
with boundary conditions $\psi(\varepsilon,\pm A)=0$,
leading to a discrete spectrum $\varepsilon_i(\beta,A)$,
$i=1,\dots$.
It follows that the wavefunctions are expressed in 
terms of parabolic cylinder functions (see appendix A for 
our conventions) as
\eqn\qmwvf{\eqalign{
\psi^+_i&=N^+\biggl(W(\nu_i,\tilde\lambda)+W(\nu_i,-\tilde\lambda)\biggr)\cr
\psi^-_i&=N^-\biggl(W(\nu_i,\tilde\lambda)-W(\nu_i,-\tilde\lambda)\biggr)\cr}
}
where $\nu_i=\beta(\half A^2-\varepsilon_i)$ 
and $\tilde\lambda=\sqrt{2\beta}
\lambda$. In the quasiclassical limit $\beta\to\infty$ the 
normalization factors may be evaluated using the asymptotic
expansions of $W$ to give:
\eqn\normfc{
(N^+)^2=(N^-)^2={1\over 2\sqrt{2(1+e^{2\pi \nu})}\beta^{-1/2}
log\sqrt{2\beta} A+\CO(\beta^{-1/2})}
}

From WKB estimates we find that the fermi level for $N$ fermions is 
\eqn\fermlevel{
\pi{N\over \beta}=\half A\sqrt{2\varepsilon_N}
-\half(A^2-2\varepsilon_N)
log\biggl((A+\sqrt{2\varepsilon_N})\sqrt{A^2-2\varepsilon_N}\biggr)}
while the density of states at the fermi level is 
$\rho(\varepsilon)={1\over \pi}\beta log\sqrt{2\beta}A$. 
It follows that the $c=1$ double-scaling limit of
\ref\brkzzm{E. Br\'ezin, V. A. Kazakov, Al. B. Zamalodchikov, 
Nucl. Phys. {\bf B333}(1990)673}
\ref\ginsparg{P. Ginsparg and J. Zinn-Justin, Phys. Lett. {\bf 240B}
(1990)333}
\ref\grmil{D.J. Gross and N. Miljkovi\'c, Phys. Lett. {\bf B238}(1990)217}
\ref\parisi{G. Parisi, Phys. Lett. {\bf B238}(1990)209}
\ref\grkli{D.J. Gross and I.R. Klebanov, Nucl. Phys. {\bf B344}(1990)475}\ 
is obtained by taking $N,\beta\to \infty$, 
with $N/\beta\to A^2/(2\pi)$ such that the cosmological constant
\eqn\dscl{
\mu\equiv\beta\bigl(\half A^2-\varepsilon_N(\beta,A)\bigr)
}
is held fixed.

Correlation functions in the model at finite $N$ may be computed
in a free fermion formalism much as in 
\bdss .
The fermion field is simply
\eqn\finfer{
\hat \Psi(\lambda,x)=\sum_{i=1}^\infty a_\epsilon(\varepsilon_i)
\psi^\epsilon(\varepsilon_i,\lambda)e^{-i\varepsilon_i x}
} 
Here $\epsilon=\pm$ refers to the parity of the wavefunction;
we always sum over parity states when the $\epsilon$ index
is repeated. The $a$'s anticommute and satisfy
$\{a_\epsilon(\varepsilon_i),a_{\epsilon'}^\dagger(\varepsilon_j)\}
=\delta_{ij}\delta_{\epsilon,\epsilon'}$. 
We may take the double scaling limit by defining
\eqn\sclfer{
\hat\Psi_\beta(\lambda,x)\equiv {1\over (2\beta)^{1/4}}
\hat\Psi
\bigl({\lambda\over\sqrt{2\beta}},\beta x\bigr)e^{\half i A^2 \beta x}
}
In the double-scaling limit 
%
%$\hat\Psi_\beta(\lambda,x)\to
%\hat\psi(\lambda,x)$, 
%
the discrete energy levels become continuous so the 
sum in \finfer\ becomes an integral $\int d\nu \rho(\nu) \cdots$ 
with $\rho(\nu)\sim {1\over \pi}log\sqrt{2\beta}A$ near the 
fermi level.
The oscillators $a_\epsilon(\varepsilon_i)$ are related to 
continuum oscillators satisfying 
$\{ a_{\epsilon}^\dagger(\nu),a_{\epsilon'}(\nu')\}=\delta_{\epsilon,
\epsilon'}\delta(\nu-\nu')$ by the rescaling:
\eqn\contosc{
a_\epsilon(\varepsilon_i)\rightarrow {a_\epsilon(\nu)\over
\bigl({1\over \pi}log\sqrt{2\beta}A\bigr)^{1/2} }
\qquad .}
Thus, taking into account \normfc\ the {\it operator}
$\hat\Psi_\beta(\lambda,x)$ has a limit as $\beta\to\infty$
for fixed $\lambda,x$
\eqn\field{
\hat\Psi_\beta(\lambda,x)\to \hat\psi(\lambda,x)=\int d\nu e^{i\nu x} 
a_{\epsilon}(\nu)\psi^\epsilon(\nu,\lambda) }
where $\psi^\epsilon(\nu,x)$ are normalized as in appendix A.

Thus we may compute quantities in the double scaling limit
directly from the fermionic field theory defined by the 
action
\eqn\action{S=\int_{-\infty}^\infty dx d\lambda
 \hat\psi^\dagger\biggl( i{d\over dx}+{d^2\over d\lambda^2}
+{\lambda^2\over 4}\biggr)\hat\psi}
The infinitely negative energies of the upside-down oscillator are
filled by the Fermi sea, i.e., the vacuum is defined by
\eqn\vac{\eqalign{
a_{\epsilon}(\nu)|\mu\rangle&=0\qquad \nu<\mu\cr
a_{\epsilon}^\dagger(\nu)|\mu\rangle&=0\qquad \nu>\mu\cr}
}
and illustrated in \fig\fermsea{The fermi sea for 
the upside down oscillator}\ ,
where $\mu$ is the cosmological constant.
When we compute physical quantities below we will often 
be interested in their topological expansion to obtain 
the results which should be reproduced by string perturbation
theory. We can re-introduce the loop counting parameter $\kappa^2$
by the substitutions $\lambda\to\kappa^{-1/2}\lambda$ 
and $\mu\to\kappa^{-1}\mu$. It is worth emphasizing that the 
above theory provides a nonperturbative definition of the 
$c=1$ model.
\foot{This has also recently been emphasized in \kazrv\ .}
Indeed, we will discuss some nonperturbative effects 
in sections three and four.

We will see below that this field theory reproduces many of the 
known results for $c=1$. For example, one thing which follows immediately 
from the identity (A.12) of appendix A is that we can compute
the derivative of the density of states as
\eqn\dos{\eqalign{
-Tr\delta'(H-\mu)&\equiv Im{1\over\pi}\int d\lambda {\p
\over \p\mu} R(\mu-i\epsilon;\lambda,\lambda)\cr
&={1\over 2\pi}Im\int_0^{\infty}ds e^{-i\mu s}{s\over sh~s/2}\cr
&={\p\over\p\mu}\biggl[{1\over 2\pi }Re~\Psi(\half+i\mu)\biggr]\cr}
}
reproducing the well-known answer for the partition function. 

\subsec{Propagators}

Correlation functions in the matrix model are given by the 
analytic continuation of correlators in the fermi field theory 
from minkowski space to euclidean space (in $x$). It is useful to 
have an operator formalism for computation directly in euclidean 
space. Thus after computing a correlation function from wick's 
theorem, using the minkowski space propagator, which is a 
sum of ``hole'' and ``particle'' propagators:
\eqn\propmink{\eqalign{
\langle\mu|
T\bigl(\hat\psi^\dagger(x_1,\lambda_1)\hat\psi(x_2,\lambda_2)
\bigr)|\mu\rangle
&=\theta(\Delta x)S_h(1,2)-\theta(-\Delta x)S_p(2,1)\cr
=\int d\nu [\theta(\nu-\mu)\theta(\Delta x)&
-\theta(\mu-\nu)\theta(-\Delta x)] e^{-i\nu \Delta x}
\psi^\epsilon(\nu,\lambda_1)\psi^\epsilon(\nu,\lambda_2)\cr}
}
where $\Delta x\equiv x_1-x_2$, 
we analytically continue $\Delta x\to -i \Delta x$, so we can 
equivalently work with  ``Euclidean time ordered'' correlation 
functions with propagator:
\eqn\propeuc{\eqalign{
S^E(x_1,\lambda_1;x_2,\lambda_2)=&
\int d\nu [\theta(\nu-\mu)\theta(\Delta x)
-\theta(\mu-\nu)\theta(-\Delta x)]e^{-\nu(\Delta x)}
\psi^\epsilon(\nu,\lambda_1)\psi^\epsilon(\nu,\lambda_2)\cr
&=e^{-\mu\Delta x}\int d\nu\int {d~p\over 2\pi}
e^{- i p\Delta x}{i\over p+i(\nu-\mu)}
\psi^\epsilon(\nu,\lambda_1)\psi^\epsilon(\nu,\lambda_2)\cr
=i e^{-\mu\Delta x} \int {d~p\over 2\pi}
e^{- i p\Delta x}&
\int_0^{sgn(p)\cdot\infty} ds 
{e^{-sp+i\mu s}\over (-4\pi i sh s)^{1/2}}
exp\bigl(-{i\over 4}({\lambda_1^2+\lambda_2^2\over
th~s}-2{\lambda_1\lambda_2\over sh~s})\bigr)
\cr}
}
where we have again used eq. A.12.

\subsec{Simple variant theories}

The above theory can be modified simply by generalizing to 
the case in which the infinite walls are at $\lambda=B,A$.
Unless $B$ or $A$ is in the scaling region near $\lambda=0$
choosing $B\not=-A$ has no effect on the universal physics. 
On the other hand, if $B=\CO(\beta^{-1/2})$, the physics 
can be modified. The simplest example of this is to take
$B=0$. In this case we have wavefunctions $\psi_i^-$ and 
the double-scaled propagator is modified to 
\eqn\newprop{
S(\lambda_1,\lambda_2;\Delta x)\to\half\biggl[
S(\lambda_1,\lambda_2;\Delta x)-
S(\lambda_1,-\lambda_2;\Delta x)\biggr]
}
Most of the calculations below can be redone for this theory.
  
\subsec{Relation to relativistic fermions}

One of our goals in this paper is to investigate 
correlation functions in this theory, and the extent to which they 
are related to correlators in a relativistic field theory. 
Let us begin by considering the anticommutator in minkowski space.
We have
\eqn\anti{
\{\hat\psi(x_1,\lambda_1),\hat\psi^\dagger(x_2,\lambda_2)\}=
{1\over (4\pi~ i sh(\Delta x))^{1/2}}exp\biggl[{i\over
4}\bigl({\lambda_1^2+\lambda_2^2\over th(\Delta
x)}-2{\lambda_1\lambda_2\over sh(\Delta x)}\bigr)\biggr]
}
where $\Delta x\equiv x_1-x_2$. For $\Delta x\to 0$ this 
approaches $\delta(\lambda_1-\lambda_2)$ as it should. Note, however
that for $\lambda_1=\lambda_2$ the short distance singularity is 
$\sim 1/(\Delta x)^{1/2}$ and differs from relativistic field theory.

Similarly the minkowski propagator has the following 
asymptotics:
\eqn\asympprop{\eqalign{
S^M&\sim \CO(1)\qquad\qquad\qquad \qquad\qquad\qquad\qquad \Delta x\to 0^+\cr
&\sim -{1\over 4\pi |\Delta x|^{1/2}}\qquad\qquad\qquad\qquad\qquad\quad 
\lambda_1=\lambda_2,\Delta x\to 0^-\cr
&\sim \CO(1)\qquad\qquad\qquad\qquad\qquad\qquad\qquad 
\lambda_1\not=\lambda_2,\Delta x\to 0^-\cr
&\sim \pm i {1\over \Delta x} e^{-i\mu \Delta x}\psi(\mu,\lambda_1)
\psi(\mu,\lambda_2)\qquad\qquad \Delta x\to \pm\infty\cr}
}


The different singularity structure from relativistic field 
theory might come as a surprise to some readers. Naively one 
might expect the following. 
Important contributions to correlation functions come from 
the neighborhood of the fermi level $\nu\cong \mu$. 
Using the ``plane-wave'' linear combinations of appendix A, which 
have the property that
\eqn\appxi{
\chi^\pm(\nu,\lambda)\cong \chi^\pm(\mu,\lambda)e^{\pm i (\nu-\mu)
\tau(\lambda,\mu)}}
where 
\eqn\timflt{
\tau(\lambda,\mu)\equiv \int_{2\sqrt{\mu}}^\lambda {dy\over
\sqrt{y^2-4\mu}}=
log\biggl({\lambda+\sqrt{\lambda^2-4\mu}\over
2\sqrt{\mu} }\biggr)
}
is the ``time-of-flight'' coordinate we may be tempted to rewrite
the fermi field \field\ as
\eqn\nwfield{
\hat\psi(\lambda,x)=e^{i\mu x}\chi^+(\mu,\lambda)\int d\nu e^{i(\nu-\mu)
(\tau+x)}\alpha^+(\nu)
+
e^{i\mu x}\chi^-(\mu,\lambda)\int d\nu e^{-i(\nu-\mu)
(\tau-x)}\alpha^-(\nu)}
where $\alpha$'s are linear combinations of the $a's$. 
It thus appears that we should have a relativistic field theory 
perhaps with small corrections.
It is important to realize 
that \appxi\ is only a valid approximation for $\lambda^2\gg \mu$,
$\lambda^2\gg\nu$, and $\mu\gg |\nu-\mu|$.
On the other hand, short distance 
singularities always come from an integration in the region 
$-\nu\gg |\lambda|$, that is, from energies well above
the top of the parabola in fig. 1,  where the parabolic cylinder functions
again behave like plane-waves, but of a different sort:
\eqn\appxii{
\psi^+\pm i\psi^-
\sim {1\over (4\pi)^{1/2}|\nu|^{1/4}} e^{\pm i\sqrt{-\nu}\lambda}
} 
Since we have taken the double scaling limit,  all nonuniversal 
quantities have dropped out so this $\nu\to -\infty$ 
behavior is physical. Equivalently, energies with $\nu$ large and negative are
still infinitesimally close to the fermi level in unscaled variables 
and thus can affect the universal physics.

It is, of course, possible to find an expansion around the relativistic
theory which allows us to calculate corrections to low-momentum/
long-distance physics. We may do this by using the WKB expansion of
parabolic cylinder functions to calculate the corrections to \appxi\
as a power series in $(\mu-\nu)^n/\lambda^m$. These correction
terms in the wavefunction may be traded in for correction 
terms to a relativistic lagrangian. The region of 
validity of this expansion defines the region of validity
of the effective theories of 
\senwad\gk\dsjv\ . 

\newsec{Correlation Functions of the Eigenvalue Density}

\subsec{$n$-Point function}

We derive 
\foot{I would like to thank T. Banks for some useful 
conversations relevant to these calculations.}
an integral representation for 
the $n$-point function of $\hat\psi^\dagger\hat\psi$.
Let 
\eqn\gfns{\eqalign{
G(x_1,\lambda_1,\dots , x_n,\lambda_n)&\equiv
\langle \mu| \hat\psi^\dagger\hat\psi(x_1,\lambda_1)\cdots
\hat\psi^\dagger\hat\psi(x_n,\lambda_n)|\mu\rangle_c\cr
G(q_1,\lambda_1,\dots q_n,\lambda_n)&\equiv \int \prod_i dx_i
e^{i q_i x_i}
G(x_1,\lambda_1,\dots x_n,\lambda_n)\cr}
}
Applying wick's theorem \gfns\ can be expressed in terms of 
a sum of ring diagrams. 
These are easily evaluated using 
\propeuc\ and after fourier transforming and taking a 
derivative with respect to $\mu$ we have 
\eqn\nptev{\eqalign{
{\p\over\p\mu}G(q_1,\lambda_1,\dots,q_n,\lambda_n)= i^{n+1}&
\delta(\sum q_i)
\sum_{\sigma\in\Sigma_n}\int_{-\infty}^{\infty} d~\xi e^{i\mu\xi} 
\int_0^{\epsilon_1\infty} ds_1\cdots\int_0^{\epsilon_{n-1} \infty} 
ds_{n-1}\cr
e^{-s_1 Q^\sigma_1-\cdots-s_{n-1}Q^\sigma_{n-1}}&
\langle\lambda_{\sigma(1)}|e^{2 i s_1 H}|\lambda_{\sigma(2)}\rangle
\cdots
\langle\lambda_{\sigma(n)}|e^{2 i \bigl(\xi-\sum_1^{n-1}s_i\bigr)H}
|\lambda_{\sigma(1)}\rangle\cr}
}
where $Q_k^\sigma\equiv q_{\sigma(1)}+\cdots q_{\sigma(k)}$,
$\epsilon_k=sgn[Q^\sigma_k]$ and $H=\half p^2
-{1\over 4}\lambda^2$ so that
\eqn\ker{
\langle\lambda_1|e^{2 i t H}|\lambda_2\rangle
={1\over (-4\pi i~sh~t)^{1/2}}
e^{-{i\over 4}({\lambda_1^2+\lambda_2^2\over
th~t}-2{\lambda_1\lambda_2\over sh~t})}
}

We can recover the result eq. 6.10 of 
\gk\ 
by taking the limit of \nptev\ 
as $q_i\to 0$ with $q_i>0$ for $i=1,\dots, n-1$. All terms for which 
$Q_k^\sigma$ change sign as $k$ runs from $1$ to $n-1$
may be seen to cancel, and the remaining terms give:
\eqn\zeromo{
-{2\over\ n}Im~i^n
\sum_{\sigma\in\Sigma_n}
\int_0^{\infty} ds_1\cdots\int_0^{\infty} 
ds_{n} e^{i\mu\bigl(\sum s_i\bigr)} 
\langle\lambda_{\sigma(1)}|e^{2 i s_1 H}|\lambda_{\sigma(2)}\rangle
\cdots
\langle\lambda_{\sigma(n)}|e^{2 i s_n H}
|\lambda_{\sigma(1)}\rangle }
for the zero-momentum eigenvalue-density correlator.

\subsec{One point function}
 
The one-point function is simply the eigenvalue density:
\eqn\evdens{
\langle \rho(\lambda)\rangle=
\langle\hat\psi^\dagger\hat\psi(\lambda,x)\rangle=
\int_\mu^\infty d\nu ~ \psi^\epsilon(\nu,\lambda)^2}
In the region $\lambda^2\gg 4\mu\gg 1$ we have the 
asymptotics:
\eqn\derevi{
-{\p\over\p\mu}\langle
\rho(\lambda)\rangle=\psi^\epsilon(\mu,\lambda)^2
\sim {2\over
\pi\sqrt{\lambda^2-4\mu}}\biggl[1-sin\bigl(
\lambda\sqrt{\lambda^2-4\mu}-2\mu\tau(\lambda,\mu)\bigr)\biggr]
}
We may use a similar wkb approximation in the integrand of \evdens\ 
for an energy range $1\ll C\ll\lambda^2-4\nu\le\lambda^2-4\mu$ for any
$C$ to conclude that the eigenvalue distribution has a smooth 
envelope 
\eqn\envelope{\rho^{cl}={1\over 2\pi}\sqrt{\lambda^2-4\mu} }
(away from $\lambda^2=4\mu$)
on which are superposed rapid oscillations of small amplitude. 
Higher order
corrections to \envelope\ are easily obtained from the asymptotic
expansions of parabolic cylinder functions. 
We thus see that in the planar limit the ``Wigner
distribution'' (in the sense of the rate of vanishing of the 
distribution) holds also for $c=1$, where there are two 
cuts with support $(-\infty,-2\mu^{1/2}]$ and
$[2\mu^{1/2},\infty)$.
\foot{It also follows that $c=1$ is a good place to look for 
a physical interpretation of so-called 2-cut models. Indeed 
the present model bears many fascinating similarities with a 
recently discovered ``topological phase'' of the 2-cut 
models
\ref\crdglmr{\v C. Crnkovic, M. Douglas, and G. Moore, ``Loop 
equations and the topological phase of multi-cut matrix models,''
to appear.}}
 The region $\mu\gg\lambda^2$ is therefore 
far outside the cut. Here we can estimate the integral \evdens\
to be (for $\lambda>0$):
\eqn\outcut{\langle\rho\rangle\sim {e^{-\pi\mu+2\sqrt{\mu}\lambda}
\over 2\pi(\pi\mu)^{1/4}}
}
Across the turning point $\lambda=\pm 2\mu^{1/2}$ the 
eigenvalue density smoothly matches the behavior 
\outcut\ to \envelope\ as in 
\fig\evdens{A sketch of the eigenvalue density at $c=1$.}.

\subsec{Two-point function}
 
Since the eigenvalue density $\hat\psi^\dagger\hat\psi$ plays the 
role of the bosonic field $\rho$ in \dsjv
we would like to investigate the singularity structure of 
the two point function and compare with the relativistic
field theory described in 
\dsjv\senwad
\gk\ . The singularity structure is 
most directly understood working in position space:
\eqn\tptsp{\eqalign{
\langle\rho(1)\rho(2)\rangle_c=
&\biggl(\int_\mu^\infty e^{-\nu |\Delta x|}\psi^\epsilon
(\nu,\lambda_1)\psi^\epsilon(\nu,\lambda_2)\biggr)
\biggl(\int_{-\infty}^\mu e^{\nu |\Delta x|}\psi^\epsilon
(\nu,\lambda_1)\psi^\epsilon(\nu,\lambda_2)\biggr)\cr
&={e^{-{1\over 4}({\lambda_1^2+\lambda_2^2\over
th~|\Delta x|}-2{\lambda_1\lambda_2\over sh~|\Delta x|})}
\over (4\pi sin|\Delta x|)^{1/2}}
\int_\mu^\infty e^{-\nu |\Delta x|}\psi^\epsilon(\nu,\lambda_1)
\psi^\epsilon(\nu,\lambda_2)\cr
&-\biggl(\int_\mu^\infty e^{-\nu |\Delta x|}\psi^\epsilon
(\nu,\lambda_1)\psi^\epsilon(\nu,\lambda_2)\biggr)
\biggl(\int_\mu^\infty e^{\nu |\Delta x|}\psi^\epsilon
(\nu,\lambda_1)\psi^\epsilon(\nu,\lambda_2)\biggr)\cr}
}
The integrals in the second equality make sense for
$|\Delta x|<\pi$. 
As $|\Delta x|\to 0$ we see that 
\eqn\shrtpt{\eqalign{
\langle \rho\rho\rangle_c&\sim -(S_h(\lambda_1,\lambda_2))^2\qquad
\lambda_1\not=\lambda_2\cr
\langle \rho\rho\rangle_c
&\sim {1\over |\Delta x|^{1/2}} \langle\rho(\lambda)\rangle \qquad
\lambda_1=\lambda_2\cr}
}

For the long-distance behavior we estimate the integrals by 
expanding the parabolic cylinder functions in powers of $(\nu-\mu)$
to find
\eqn\esti{\eqalign{
\int_\mu^\infty 
e^{-\nu |\Delta x|}
\psi^\epsilon(\nu,\lambda_1)\psi^\epsilon(\nu,\lambda_2)
&={e^{-\mu|\Delta x|}\over |\Delta x|}\biggl[
\psi^\epsilon(\mu,\lambda_1)\psi^\epsilon(\mu,\lambda_2)+\CO(1/|\Delta
x|)\biggr]\cr
\int_{-\infty}^\mu
e^{\nu |\Delta x|}
\psi^\epsilon(\nu,\lambda_1)\psi^\epsilon(\nu,\lambda_2)
&={e^{\mu|\Delta x|}\over |\Delta x|}\biggl[
\psi^\epsilon(\mu,\lambda_1)\psi^\epsilon(\mu,\lambda_2)+\CO(1/|\Delta
x|)\biggr]\cr}
}
so that we have the asymptotics for $|\Delta x|\to \infty$:
\eqn\lgtpt{
\langle \rho\rho\rangle_c\sim{1\over |\Delta x|^2}\biggl(
\psi^\epsilon(\mu,\lambda_1)\psi^\epsilon(\mu,\lambda_2)
\biggr)^2+\CO(1/|\Delta x|^4)
}

In \dsjv\senwad\gk\ it was proposed that the genus zero correlators
of the eigenvalue density would be identicle to that of a free boson.
Taking the continuum limit of the expressions in \senwad\gk\ for 
example we obtain the two-point function:
\eqn\relbs{
\langle \rho\rho\rangle_c=-{\p \tau_1\over\p\lambda}
{\p \tau_2\over\p\lambda}
{1\over 4\pi}\p_{\tau_1}\p_{\tau_2}
log\biggl({(\Delta x)^2 + (\tau_1-\tau_2)^2\over (\Delta x)^2+
(\tau_1+\tau_2)^2}\biggr)
}
which behaves like $1/(\Delta x)^2$ at large and small $\Delta x$. 

Taking into account the wkb estimate of the parabolic cylinder 
functions we see from \lgtpt\ that \relbs\ agrees with the exact formula 
in the region $\lambda^2\gg\mu,|\Delta x|^2\to \infty$. 
The deviation of \relbs\ from the exact result in the short
distance region is due, technically, to the origin of the 
short distance singularities in the region $\nu\to-\infty$
as explained in section 2.4.

The deviation from relativistic behavior has an 
important consequence. 
Physically, there is a crossover
in the behavior due to a breakdown in the genus expansion
\foot{Conversations with T. Banks and S. Shenker have been very helpful
in clarifying these points.}. 
This is seen most clearly from \nptev\ for the case
$n=2$, which may be expressed in terms of the resolvent
$R(\zeta; \lambda_1,\lambda_2)$ of the Schr\"odinger 
operator with potential $u=-\lambda^2/4$ :
\eqn\twopt{
{\partial\over\partial\mu} G= \biggl(\psi^\epsilon(\mu,\lambda_1)
\psi^\epsilon(\mu,\lambda_2)\biggr)\biggl[
R(\zeta=\mu+i|p|;\lambda_1,\lambda_2)
+R(\zeta=\mu-i|p|;\lambda_1,\lambda_2)\biggr]}
On the diagonal we may use the asymptotic expansion of 
\ref\gelfdick{I.M. Gelfand and L.A. Dickii, Russian Math Surveys. 
{\bf 30}(1975)77.}\ 
(with potential $u=-\lambda^2/4$) to obtain 
\eqn\expres{\psi^\epsilon(\mu,\lambda)^2\sum_{n=0}^\infty
\biggl[{R_n[u]\over(-\mu+i|p|)^{n+1/2}}+
{R_n[u]\over(-\mu-i|p|)^{n+1/2}}\biggr]
}
so that the asymptotic expansion in $1/\mu$, i.e., the 
genus expansion, breaks down for 
large $p$.
\foot{After the this work was completed we learned that
G. Mandal, A. Sengupta and S. Wadia have obtained related results
\ref\wadsem{S. Wadia, Seminar at Rutgers University.}\ .}
The perturbative regime 
has a fermi level well below the tip of the potential 
$u=-\lambda^2/4$. As we 
increase $|p|$ we probe states at higher energies above the 
fermi sea. At $|p|\sim \mu$ there is a qualitative change in the 
nature of the wavefunctions, and a qualitative change in the 
physics.

\subsec{Comparison with Collective Field Theory}

In some very interesting papers 
\dsjv\ 
\ref\jvrv{A. Jevicki, ``Matrix models and field theory,'' BROWN-HET-771}
\ref\jvii{A. Jevicki, ``Collective field theory and schwinger-dyson 
equations in matrix models,'' BROWN-HET-777}\ 
an attempt has been made to use the matrix model to derive a 
field theory for the eigenvalue-distribution field $\rho(\lambda,t)$.
In particular, the classical solution of the lagrangian of \dsjv\ is
\envelope\ . Comparing with the exact eigenvalue distribution we 
see that the field theory of \dsjv\ omits the oscillatory terms 
in the eigenvalue distribution. One might be tempted to call such 
terms ``nonperturbative'' since they involve an exponential 
$\sim e^{i \lambda^2/\kappa}$ but one must be careful about 
such terminology, since these terms are also $\CO(1)$. Indeed 
in ordinary quantum mechanics such terms in WKB wavefunctions 
can contribute to the {\it perturbative} expansion in $\hbar$ in 
the computations of physical expectation values.
Similarly, as we
have seen, the two-point function only agrees with 
the genus zero 
predictions of the Das-Jevicki lagrangian at large distances.
It thus appears that the exact field theory of the eigenvalue 
distribution - if it can be written as a field theory at all -
is only approximately relativistic at long distances and has
a different (and much {\it softer}) behavior at short 
distances. Moreover,
Polchinski's spacetime effective field theory 
interpretation of $c=1$ \JOE\ 
was interpreted \dsjv\ as being 
equivalent to the Das-Jevicki lagrangian after identification 
of the tachyon field with the eigenvalue field.
Combining these observations we are lead to ask
if it is a generic feature of nonperturbative
string field theory that the theory only resembles
a relativistic field theory on long distance scales and is a 
much softer theory on short distance scales.
 
\newsec{Macroscopic loop amplitudes at $c=1$}
 
\subsec{$n$-point function}

Formula \nptev\ for the correlation functions of
the eigenvalue densities allows us 
to calculate the $n$-point function $M(z_i,x_i)$
of the ``macroscopic loop operators'': 
\eqn\macropt{\eqalign{
\hat\psi^\dagger e^{iz\hat \lambda}\hat\psi&\equiv
\int_{-\infty}^{\infty}d\lambda \hat\psi^\dagger(\lambda,x)
e^{iz\lambda}\hat\psi(\lambda,x)\cr
M(z_i,x_i)&\equiv \langle
\hat\psi^\dagger e^{iz_1\hat\lambda}\hat\psi
\cdots\hat\psi^\dagger e^{iz_n\hat\lambda}\hat\psi\rangle\cr}
}
From $M(z_i,x_i)$ we can obtain the double-scaled correlation
functions of the resolvent $tr\bigl({1\over \zeta-\phi}\bigr)$
via a laplace transform. 
Note that these integrals 
converge and get their main contribution from the region 
near the edge of the classical eigenvalue distribution,
since, for large $\lambda$ the oscillatory function
$e^{i z\lambda}$ gives a small contribution to the 
integral. 
By contrast, correlation functions of the operators 
$\hat\lambda^n$, which correspond to correlators of
$tr~\phi^n$ in the matrix model, are infinite. We believe
this is the source of some of the wavefunction-renormalization 
ambiguities which have plagued previous $c=1$ calculations.

By shrinking the macroscopic loop amplitudes we may extract 
the local scaling operators of the theory and their correlation
functions, as in the one-matrix model
\bdss\ . Namely, if
$M(z_i,q_i)$ is the fourier transform of \macropt\ 
we expect on physical grounds that the 
small $z_i$ asymptotics of $M$ will have the form
\eqn\expasymps{
M(z_i,q_i)\sim \sum_{\Delta_i}\prod z_i^{\Delta_i}
\langle \CO_{\Delta_1}\cdots\CO_{\Delta_n}\rangle
}
where $\CO_{\Delta}$ are the scaling operators of 
dimension $\Delta>0$. We will find below that there are
also divergent terms in the $z_i\to 0$ limit. A 
nontrivial prediction of the liouville theory is that
these terms will be analytic in $\mu$ since they arise
from surfaces of zero area
\ref\natiliouv{N. Seiberg, ``Notes on Quantum liouville Theory and 
Quantum Gravity,'' Rutgers preprint RU-90-29, to appear in the 
proceedings of the 1990 Yukawa International Seminar, Common 
Trends in Mathematics and Quantum Field Theories}
\ref\wdwppr{G. Moore, N. Seiberg, M. Staudacher, Rutgers preprint RU-91-11}\ .
We will verify below that this is indeed the case in all our
explicit formulae.

The calculation of \macropt\ reduces to the 
evaluation of a gaussian integral after we use
\nptev\ .
One finds the result
\foot{Some details of the calculation are 
provided in appendix C.}
\eqn\macptii{\eqalign{
{\p\over\p\mu}M(z_i,q_i)=\half i^{n+1}\delta(\sum
q_i)\sum_{\sigma\in\Sigma_n}&
\int_{-\infty}^{\infty} d\xi {e^{i\mu \xi}\over |sh~\xi/2|}
\int_0^{\epsilon_1\infty} ds_1\cdots\int_0^{\epsilon_{n-1}\infty}
ds_{n-1}\cr
exp\bigl(-\sum_{k=1}^{n-1}s_k Q^\sigma_k\bigr)
exp\bigl({i\over 2}cth(\xi/2)\sum z_i^2 \bigr)&
exp\bigl(i\sum_{1\leq i<j\leq n}
{ch\bigl(s_i+\cdots s_{j-1}-\xi/2\bigr)\over
sh~(\xi/2)}z_{\sigma(i)}z_{\sigma(j)}\bigr)\cr}
}
It is easy to show that $M$ is real, totally symmetric in the 
$(z_i,q_i)$ and invariant under parity: $q_i\to -q_i$. 

Several remarks are in order regarding the physical interpretation
of \macptii\ . In \macropt\ we have introduced macroscopic 
loop lengths which are imaginary in order for the integrals 
over $\lambda$ to converge. We cannot continue back to 
physically sensible real loop lengths $\ell$ by $z\to \pm i \ell$ 
because the eigenvalue distribution is concentrated on both 
sides of the origin. Note however that in \macptii\ the integral 
naturally splits into two pieces corresponding to integrating over
or $\xi\in [0,\infty)$ or $\xi\in (-\infty,0]$.  In the first 
integral we may continue $z\to i \ell$ in the upper half plane
to obtain a convergent 
answer. This analytic continuation makes no sense in the second 
integral, but there we can analytically continue
$z\to -i\ell$ in the lower half-plane.
We interpret the two pieces as the 
contributions of the two ``worlds'' defined by the two 
eigenvalue cuts. Focusing on either contribution we 
can define macroscopic loop amplitudes for real loop lengths.
We expect that up to an
overall $q_i$-dependent wavefunction renormalization, there
is no difference in the perturbative expansion
between keeping $z_i$ real and continuing 
as above. One can check this explicitly below.
Moreover, when we consider only one contribution, say, 
$\int_0^\infty d \xi$ we might 
try to interpret
\foot{This is a suggestion of N. Seiberg.}
$-i\xi$ as the area and the integrand of the 
$\xi$ integral as the fixed-area partition function in liouville
coupled to $c=1$ conformal field theory. 

\subsec{Two-point function}

We first examine the two-point function. 
From this we can determine the wavefunctions, propagators, and 
some facts about the spectrum of the theory. Factoring out
the momentum-conserving delta function the rhs of \macptii\ becomes
\eqn\twpti{Im~\int_0^\infty d~\xi {e^{i\mu\xi + i/2 cth(\xi/2)(z_1^2+z_2^2)}
\over sh~(\xi/2)}
\int_0^\infty ds e^{-|q|s }
\biggl(e^{i{ch(s-\xi/2)\over sh\xi/2}z_1z_2}-
e^{i{ch(s+\xi/2)\over sh\xi/2}z_1z_2}\biggr)
}
This formula holds for $z_i$ real. If we wish to have 
real loop lengths we replace $Im\to -\half i$ and continue 
$z_i\to i \ell_i$, as discussed above.

We are interested in the small $z_i$ asymptotics of \twpti\ .
Change variables 
from $s$ to $y=e^s$ and expand in inverse powers of $y$. 
The integral over $s$ may be written as
\eqn\expbessl{
2\pi e^{-i\pi|q|/2}{sh(|q|\xi/2)\over sin~\pi|q|}J_{|q|}(2\alpha)
+\sum_{r=1}^\infty {4 i^r r\over r^2-q^2}J_r(2\alpha) sh(r\xi/2)
}
where $\alpha=z_1z_2/2 sh(\xi/2)$. Since $|J_r(z)|\leq {|(z/2)^r|
\over r!}$ for $z$ real the series converges.
Here we take $|q|\notin\IZ$.  Note that \expbessl\ has a 
smooth limit as $|q|\to r$, $r\in \IZ$, since the pole in the
first term exactly cancels the pole in the $r+1^{st}$ term.
So, for $|q|=r$ we have instead:
\eqn\expbessli{
2(-1)^r{\p\over\p |q|}\biggl[e^{-i\pi|q|/2}sh(|q|\xi/2)J_{|q|}(2\alpha)\biggr]_{|q|=r}
+\sum_{r\not= t}^\infty {4 i^t t\over t^2-r^2}J_t(2\alpha) sh(t\xi/2)
}

We can already begin to deduce some characteristics of the spectrum
at $c=1$ from \expbessl\ .The expansion of the first
bessel function $J|_{|q|}$ provides an infinite series of terms of the form
$z_1^{|q|+2n}z_2^{|q|+2n}$ for $n$ a nonnegative integer.
These contributions will be interpreted as arising from a set of operators
\foot{We follow topological field theory notation.}
$\sigma_{2n}(\CO_q)$ for $q\in \IR$. The $\sigma_{0}(\CO_q)$ 
are ``gravitational primaries'' corresponding to the dressed version
of $e^{iqX}$, and each gravitational primary gives rise to an 
infinite tower of gravitational descendents. 
Fortunately, the contributions of $\sigma_{2n}(\CO_q)$,
for $q\notin \IZ$ can be unambiguously distinguished 
from the more intricate contributions of the integer-moded
bessel functions $J_r$. Clearly, these
terms are related to the degenerate representations of the virasoro
algebra for $c=1, \Delta=n^2/4$. In particular, N. Seiberg
has proposed that the operators $\sigma_{2n}(\CO_{\pm r})$
be identified with the dressed {\it virasoro} primaries in the 
$U(1)$ kac-moody representation of charge $\pm r$. We hope
to return to a more thorough study of these contributions in the 
future.  

We now consider the correlation function at fixed 
cosmological constant. 
 One may expand the 
bessel functions and do the integrals term-by-term
(use GR 3.383.4) in terms of whittaker functions. In 
particular, use the integral
\eqn\anint{\eqalign{
\int_0^\infty {e^{i\mu\xi+i\half cth(\xi/2)z^2}\over
(sh~\xi/2)^{1+|q|+2k} } sh~(|q|\xi/2)=2^{|q|+2k}(-iz^2)^{-(1+|q|+2k)/2}&
\cr\cr
\biggl[
\Gamma(k+\half-i\mu)W_{i\mu+|q|/2,|q|/2+k}(-iz^2)
-\Gamma(k+|q|+\half-i\mu)&W_{i\mu-|q|/2,|q|/2+k}(-iz^2)\biggr]\cr}
}
%may be done using the change of variables $w=cth(\xi/2)$
%and
As long as
$|q|\notin \IZ$ we may 
further express the whittaker functions in terms of the degenerate
hypergeometric function 
\def\oneff{{}_1F_1}
$\oneff(\alpha,\beta;z)$,
\eqn\dfonf{
\oneff(\alpha,\beta;z)\equiv 1+{\alpha\over \beta}{z\over 1}+
{\alpha(\alpha+1)\over \beta(\beta+1)}{z^2\over 2!}+\cdots
}
 The resulting amplitude should be separated into 
two parts defined by the holomorphic and nonholomorphic terms
in $\mu$. We find the nonholomorphic terms are:
\eqn\nnhltw{
\eqalign{
2Im\Biggl[e^{-i\pi|q|/2}\sum_{k=0}{\Gamma(-|q|-2k)\Gamma(-|q|-k)\over
k!}(z_1z_2)^{|q|+2k}&\qquad\cr
\biggl({\Gamma(k+|q|+\half-i\mu)\over\Gamma(-k+\half-i\mu)}e^{w/2}
\oneff(k+\half+i\mu, 1+2k+|q|;-w)&-\cr
{\Gamma(k+\half-i\mu)\over\Gamma(-k-|q|+\half-i\mu)}
e^{-w/2}
\oneff(k+\half-i\mu,& 1+2k+|q|;w)\biggr)\Biggr]\cr}
}
where $w\equiv -i(z_1^2+z_2^2)$.

The holomorphic terms in $\mu$ are similar:
\eqn\hltw{
\eqalign{
2
Im\Biggl[\sum_{k=0}{\Gamma(|q|+2k)\Gamma(-k-|q|)\over
k!}\bigl({z_1z_2\over z_1^2+z_2^2}\bigr)^{|q|+2k}&\qquad\qquad\cr
\biggl(e^{-w/2}
\oneff(-k+\half-i\mu, 1-2k-|q|;w)
-e^{w/2}
\oneff(-k&+\half+i\mu, 1-2k-|q|;-w)
\biggr)\Biggr]\cr}
}
As claimed above, all divergent terms for $z_i\to 0$ 
multiply holomorphic expressions in $\mu$.

One interesting special case of the above formula is the genus
zero limit. The topological 
expansion is defined by substituting 
$\mu\to \kappa^{-1}\mu$ and $z\to \kappa^{1/2} z$
and expanding in $\kappa\to 0$. 
Using an expansion of the whittaker function as an 
infinite sum of bessel functions
\ref\buch{H. Buchholz, {\it Die Konfluente Hypergeometrische Funktion}, 
Springer 1953}\ we obtain the genus zero two-macroscopic
loop amplitude:
\eqn\gnzrtwlp{\langle w(\ell_1)w(\ell_2)\rangle=
4|q|\sum_{k=0}^\infty{(-1)^k\over k!}\Gamma(-|q|-k)\biggl(
{\mu\ell_1\ell_2\over\sqrt{\mu(\ell_1^2+\ell_2^2)}}\biggr)^{|q|+2k}
K_{|q|+2k}(2\sqrt{\mu(\ell_1^2+\ell_2^2)})
}   

Turning now to local operators, from \nnhltw\ 
we can find the two-point functions of $\sigma_{2k}(\CO_q)$
which are defined 
\foot{In fact, this is rather naive and one should define an appropriate
inner product on a space of wavefunctions. See \wdwppr\ .
Related to this is the issue of the ``best'' normalization 
of the operators $\sigma_{2n}(\CO_q)$, which is not that 
given above.}
by 
\eqn\dfntwpt{
{\p M\over \p\mu}\equiv 2\sum\ell_1^{|q|+2k_1}\ell_2^{|q|+2k_2}
{\p\over \p\mu}\langle\sigma_{2k_1}(\CO_q)\sigma_{2k_2}(\CO_{-q})\rangle
}
In particular, from \nnhltw\ we may extract
\eqn\twptii{\eqalign{
{\p\over\p\mu}\langle\sigma_{0}(\CO_q)\sigma_{0}(\CO_{-q})\rangle
=\qquad\qquad &\qquad\qquad\qquad\cr
(\Gamma(-|q|))^2
Im~\biggl[e^{i\pi |q|/2}&
\biggl({\Gamma(|q|+\half-i\mu)\over\Gamma(\half-i\mu)}
-
{\Gamma(\half-i\mu)\over\Gamma(-|q|+\half-i\mu)}\biggr)
\biggr]\cr}
}
for $q\notin \IZ$. As always, the situation for 
$|q|\to r$ is more subtle. We may attempt to extract the 
correlation function by taking a limit of \twptii\ . 
The second order pole multiplies holomorphic terms in $\mu$ and 
can be dropped. The first order pole multiplies nonholomorphic
terms. Thus we may imagine that we can absorb the pole
into a wavefunction renormalization of $\sigma_0(\CO_r)$
in which case we obtain the result 
\eqn\twdg{
\eqalign{
{\p\over\p\mu}
\langle \sigma_0(\CO_r)\sigma_0(\CO_{-r})\rangle=\qquad\qquad\qquad\qquad&
\cr
Im\Biggl[i^r
\biggl((\half-i\mu)_r&+(-1)^{r+1}(\half+i\mu)_r\biggr)
\Psi(\half-i\mu)
\Biggr]\cr}
}
where $(x)_n\equiv x(x+1)\cdots(x+n-1)$ is the pochhammer symbol.


We now check \twptii\  physically
in two ways. First, taking the $q\to 0$ limit we find
\eqn\quezer{
\lim_{q\to 0}{\p\over\p
\mu}\langle\sigma_{0}(\CO_q)\sigma_{0}(\CO_{-q})\rangle
={\p\over\p \mu}Re\Psi(\half-i\mu)
}
As expected, $\sigma_0(\CO_q)$
becomes the  cosmological constant in the zero momentum limit.
An interesting subtlety arises in comparing
\quezer\ with the liouville theory. 
For nonzero momentum the operator $\sigma_0(\CO_q)$ corresponds to the 
liouville operator
\eqn\liouv{
\sigma_0(\CO_q)\leftrightarrow f(q)
\int_\Sigma e^{\gamma(1-|q|/2)\varphi_L}
e^{iq\cdot X} 
}
where $\Sigma$ is the worldsheet, $\gamma=\sqrt{2}$, $\varphi_L$
is the liouville field and $X$ is the embedding coordinate. 
On the other hand, the cosmological constant at $c=1$ is 
given by $\int \varphi e^{\gamma\varphi}$
\JOE\natiliouv\ . Therefore, the operator $\int e^{\gamma\varphi_L}$
must decouple from amplitudes
\foot{This was pointed out by N. Seiberg.}. 
Moreover, it follows from the smooth 
limit in \quezer\ that the ``wavefunction renormalization''
$f(q)$ must have a pole at $q=0$.

As a second check we show that
at fixed $q$, \twptii\ has a topological expansion. 
After analytic continuation to real loop lengths 
$z_i\to i \ell_i$ we may use the expansion of 
gamma functions quoted in appendix B to obtain the 
result at any order of perturbation theory. 
In particular, the first four orders of perturbation 
theory are:
\eqn\twptiii{\eqalign{
\langle \sigma_0(\CO_q)\sigma_0(\CO_{-q})\rangle =
(q\Gamma(-|q|))^2\mu^{|q|}&\biggl[{1\over |q|} \qquad\qquad\qquad\cr
&-(|q|-1){(q^2-|q|-1)\over 24}\mu^{-2}\cr
+\prod_{r=1}^3(|q|-r)&
{(3q^4-10|q|^3-5q^2+12|q|+7)\over 5760}\mu^{-4}\cr
-\prod_{r=1}^5(|q|-r)&
{(9q^6-63|q|^5+42q^4+217|q|^3-205|q|-93)\over 2903040}
\mu^{-6}\cr
&\qquad\qquad\qquad+\cdots \biggr]\cr}
}
The first term is 
in agreement with the famous result of \kostov\ .
The above expansion should prove very useful in 
formulating a Das-Jevicki-like theory to all orders
of perturbation theory. 

We may again observe the phenomena of the breakdown 
of the topological expansion for large momenta.
The asymptotic expansion of the correlation function may 
be obtained from the expansion of 
\eqn\explg{log{\Gamma(a+\half-i\mu)\over \Gamma(\half-i\mu)}
}
The $(2h+1)^{th}$ term in the series has the form
\eqn\proptw{
{a\over\mu^{2h+1}}(-1)^h B_{2h}(1-2^{-2h+1})+\cdots
+i {(-1)^h\over (2h+2)(2h+1)}{a^{2h+2}\over\mu^{2h+1}}}
The first term reproduces the familiar $(2h)!$ in the asymptotic
expansion in genus. The last term, however, is a 
convergent expansion for $|a|<\mu$, and diverges for $|a|>\mu$. 
In this way we see that for large enough momenta, the 
topological expansion ceases to be even asymptotic. As we 
have discussed this is due to the qualitatively 
different nature of the 
wavefunctions for energies above the top of the parabolic 
maximum. From another point of view, 
this phenomenon is probably related to nonrenormalizability
of the effective theories of \JOE\dsjv\senwad\gk\ .
More physically, the string approximation is seen 
to be an effective description only for low momentum processes, but at
high momenta we begin to see the fermi particle itself. 

Another application of the formulae of this section
\foot{Suggested by S. Shenker}
is to the study of high energy processes.
From \twptiii\ we see that the scattering amplitudes 
at fixed genus are (after dividing by a genus-independent
wavefunction renormalization) polynomials in the 
momenta. As we will see, 
similar results hold for other correlation functions derived below. 
This behavior is rather different from the high energy
behavior of perturbative critical string theory
\ref\grmend{D.J. Gross and P.F. Mende, Phys. Lett. {\bf 197B}(1987)
129; Nucl. Phys. {\bf B303}(1988)407}.
If, instead, we study the high energy behavior of the 
nonperturbative result \twptii\ we obtain the large $|q|$ asymptotics: 
\eqn\lrgqtw{
%{\sqrt{\pi}\over sin~\pi|q|}e^{\mu\pi/2}\Gamma(-|q|)\biggl[
%sin\bigl(\pi|q|/2 +\mu log(|q|/\mu)+\mu\bigr)
%-
%sin\bigl(3\pi|q|/2 -\mu log(|q|/\mu)-\mu\bigr)\biggr]
\eqalign{
{-\pi^{3/2}\over |q|sin^2\pi|q|}e^{-|q|log~|q|+|q|}\sqrt{cosh(\pi\mu)}
\biggl[{2+e^{2\pi\mu}\over 1+e^{2\pi\mu}}&
cos\bigl(\pi|q|/2 -\mu log|q|+2\Phi(\mu)\bigr)\cr
+{e^{2\pi\mu}\over 1+e^{2\pi\mu}}
cos\bigl(3\pi|q|/2 &+\mu log|q|-2\Phi(\mu)\bigr)\biggr]\cr}
}
This also differs from proposed summations of the 
results of \grmend\ given in 
\ref\menoog{P.F. Mende and H. Ooguri, Nucl. Phys. {\bf B339}(1990)641} .

\subsec{Wavefunctions and the Wheeler-de-Witt Equation}

In a recent paper \wdwppr\ the role of the wheeler-de-witt
equation in 2D gravity and 
matrix models has been clarified. In this 
section we sketch how many of the results in \wdwppr\ , which 
were derived for $c<1$ may be trivially carried over to the case 
of $c=1$. 

We may use \expbessl\ to obtain a formula for a macroscopic 
loop with one tachyon insertion, by letting $\ell_1\to 0$ holding 
$\ell_2$ finite. 
The genus zero contribution is 
particularly interesting because, following the reasoning of 
\wdwppr\ 
we expect it to solve the wheeler-de-witt equation in minisuperspace. 
The wavefunction of $\sigma_0(\CO_q)$ is easily extracted from 
\gnzrtwlp\ with the result:
\eqn\onbglp{
\psi^{h=0}_q(\ell)\equiv\langle\sigma_0(\CO_q)w(\ell)\rangle=
 \bigl(2|q|\Gamma(-|q|)\bigr)
\mu^{|q|/2}K_{|q|}\biggl(2\sqrt{\mu \ell^2}\biggr)
}
in beautiful confirmation of the expectations of the liouville theory.
In particular, 
$\psi^{h=0}_q$ satisfies the wheeler-de-witt equation 
in minisuperspace:
\eqn\wdwi{
\biggl(-(\ell{\p\over\p\ell})^2+4\mu\ell^2+q^2\biggr)\psi^{h=0}_q=0
}

Inspired by the result at genus zero, and the relatively simple 
expression \anint\ summarizing the wavefunction $\psi_q$ to 
all orders of the topological expansion we ask if some 
simple equation summarizes the corrections to the WdW equation 
from summing over topologies. The answer is especially simple
at $q=0$ where we find the wavefunction
\foot{This wavefunction has fascinating large $\ell$ behavior:
$$\psi_{q=0}(\ell)\to-\sqrt{8\pi\over 1-e^{-\pi\mu}}e^{-\pi\mu}
\ell^{-1}cos\bigl[\half\ell^2-\mu log\ell^2+2\Phi(\mu)-\pi/4\bigr]$$
To all orders of perturbation theory the wavefunction decays
exponentially at $\ell\to\infty$ but nonperturbatively the 
the wavefunction  oscillates rapidly with a slowly-decaying envelope.}
\eqn\alwvfn{
\psi_{q=0}=Im \biggl[e^{-3\pi i/4}\Gamma(\half-i\mu)
\ell^{-1}W_{i\mu,0}(i\ell^2)\biggr]
}
using the whittaker differential equation we find that 
$\psi_q$ satisfies the modified wheeler-de-witt equation:
\eqn\wdwmdf{
\biggl(-(\ell{\p\over\p\ell})^2+4\mu\ell^2-\kappa^2
\ell^4\biggr)\psi_{q=0}=0
}
where we have explicitly introduced the topological coupling
$\kappa$. The extra term is reminiscent of wormhole effects 
arising from nonlocal terms in the liouville action 
$\sim (\int_{\Sigma} \varphi_L e^{\gamma\varphi_L})^2$. 
Similar considerations apply to $\psi_q$. The modified
wheeler-de-witt equation is most simply written by 
taking an inverse laplace transform in $\mu$ to work 
at fixed area.

\subsec{Three-point function}

We now examine the three-point function. This allows us to 
determine the nature of the couplings and fusion rules in 
the theory. Moreover, since there is an ambiguity in 
the relative normalization of matrix-model and liouville 
operators we cannot obtain unambiguous physical amplitudes
until we compute the three-point function.
Choosing $q_1,q_2>0$ the rhs of \macptii\ becomes
\eqn\thrpti{\eqalign{
2\delta(\sum q_i)Re\int_0^\infty d\xi{e^{i\mu\xi}\over sh~(\xi/2)}
e^{{i\over 2}cth(\xi/2)\sum z_i^2}\Biggl\{
&\cr
\int_0^\infty ds_1\int_0^\infty ds_2 e^{-q_1 s_1+q_3 s_2}
&\biggl[ e^{i(f(s_1)z_1 z_2+f(s_1+s_2)z_1z_3+f(s_2)z_2z_3}
\cr
&+e^{i(f(-s_1)z_1 z_2+f(-s_1-s_2)z_1z_3+f(-s_2)z_2z_3}\biggr]\cr
-\int_0^\infty ds_1\int_0^\infty ds_2 e^{-q_1 s_1-q_2s_2}&
\biggl[ e^{i(f(s_1)z_1 z_3+f(s_1-s_2)z_1z_2+f(-s_2)z_2z_3}
\cr
&+e^{i(f(-s_1)z_1 z_3+f(-s_1+s_2)z_1z_2+f(s_2)z_2z_3}\biggr]\cr
+\int_0^\infty ds_1\int_0^\infty ds_2 e^{-q_2 s_1+q_3 s_2}&
\biggl[ e^{i(f(s_1)z_1 z_2+f(s_1+s_2)z_2z_3+f(s_2)z_1z_3} \cr
&+e^{i(f(-s_1)z_1 z_2+f(-s_1-s_2)z_2z_3+f(-s_2)z_1z_3}\biggr]
\Biggr\} \cr}
}
where $f(s)=ch(s-\xi/2)/sh(\xi/2)$. 


The small $z_i$ behavior of the above integrals is subtle. 
For simplicity assume that $|q_i|<1$. 
%
%Denote
%$$\varepsilon_{jk}^{\pm}\equiv {-i z_j z_k e^{\pm \xi/2}\over
%2 sh~(\xi/2)}$$
%
Denote the six integrals by $I_j$, $j=1,6$. 
These integrals contribute both analytic and nonanalytic terms in 
$\mu$. Keeping only the nonanalytic terms in $z_i$ and dropping
terms of order $\CO(z^{2|q|+n})$ for $n\geq 1$ we find that 
$I_1+I_2$ contributes
\eqn\cntrii{\eqalign{
2(-i z_2z_3)^{|q_3|}{\Gamma(-|q_3|)^2\over q_1}F(q_3,-q_1,1-q_1;-z_1/z_2)
&\CF^+(|q_3|;\mu)\cr
+2(-i z_1z_3)^{|q_3|}{\Gamma(-q_1)^2\over q_2}F(-q_1,q_2,1+q_2;-z_2/z_3)
&\CF^+(q_1;\mu)\cr
+2(-i z_1z_3)^{q_1}(-iz_2z_3)^{q_2}\Gamma(-q_1)\Gamma(-q_2)\Gamma(-|q_3|)
&\CF^+(|q_3|;\mu)\cr}
%
%\Biggl[{\Gamma(\half-i\mu+|q_3|)\over \Gamma(\half-i\mu)}
%+{\Gamma(\half-i\mu)\over \Gamma(\half-i\mu-|q_3|)}\Biggr]\cr
%
}
where $F(\alpha,\beta,\gamma;z)$ is the hypergeometric
function and 
we will frequently use the notation
\eqn\dffn{\CF^\pm(a,b;\mu)\equiv
{\Gamma(a+\half-i\mu)\over\Gamma(b+\half-i\mu)}
\pm
{\Gamma(-b+\half-i\mu)\over\Gamma(-a+\half-i\mu)}
}
We also use the notation $\CF^\pm(a;\mu)=\CF^\pm(a,0;\mu)$. 


The contribution of $I_5+I_6$ is obtained by changing
$z_1\leftrightarrow z_2$ and $q_1\leftrightarrow q_2$. 
The contribution of $I_3+I_4$ is similar:
\eqn\cntriii{\eqalign{
-2(-i z_2z_3)^{q_2}{\Gamma(-q_2)^2\over q_1}F(-q_2,q_1,1+q_1;-z_1/z_3)
&\CF^+(q_2;\mu)\cr
%
%\Biggl[{\Gamma(\half-i\mu+q_2)\over \Gamma(\half-i\mu)}
%+{\Gamma(\half-i\mu)\over \Gamma(\half-i\mu-q_2)}\Biggr]\cr
%
-2(-i z_1z_3)^{q_1}{\Gamma(-q_1)^2\over q_2}F(-q_1,q_2,1+q_2;-z_2/z_3)
&\CF^+(q_1;\mu)\cr
%
%\Biggl[{\Gamma(\half-i\mu+q_1)\over \Gamma(\half-i\mu)}
%+{\Gamma(\half-i\mu)\over \Gamma(\half-i\mu-q_1)}\Biggr]\cr
%
-2(-i z_1z_3)^{q_1}(-iz_2z_3)^{q_2}\Gamma(-q_1)\Gamma(-q_2)\Gamma(-|q_3|)
&\CF^+(q_1,-q_2;\mu)\cr}
%
%\Biggl[{\Gamma(\half-i\mu+q_1)\over \Gamma(\half-i\mu-q_2)}
%+{\Gamma(\half-i\mu+q_2)\over \Gamma(\half-i\mu-q_1)}\Biggr]\cr}
%
}
Adding these together we find:
\eqn\cntriv{\eqalign{
2(-i z_2z_3)^{|q_3|}{(\Gamma(-|q_3|)^2\over q_1}F(q_3,-q_1,1-q_1;-z_1/z_2)
&\CF^+(|q_3|;\mu)\cr
%
%\Biggl[{\Gamma(\half-i\mu+|q_3|)\over \Gamma(\half-i\mu)}
%+{\Gamma(\half-i\mu)\over \Gamma(\half-i\mu-|q_3|)}\Biggr]\cr
%
+
2(-i z_1z_3)^{|q_3|}{(\Gamma(-|q_3|)^2\over q_2}F(q_3,-q_2,1-q_2;-z_2/z_1)
&\CF^+(|q_3|;\mu)\cr
%
%\Biggl[{\Gamma(\half-i\mu+|q_3|)\over \Gamma(\half-i\mu)}
%+{\Gamma(\half-i\mu)\over \Gamma(\half-i\mu-|q_3|)}\Biggr]\cr
%
+
2(-i z_1z_3)^{q_1}(-iz_2z_3)^{q_2}\Gamma(-q_1)\Gamma(-q_2)\Gamma(-|q_3|)
&\CF^+(|q_3|;\mu)\cr
%
%\Biggl[{\Gamma(\half-i\mu+|q_3|)\over \Gamma(\half-i\mu)}
%+{\Gamma(\half-i\mu)\over \Gamma(\half-i\mu-|q_3|)}\Biggr]\cr
+
2(-i z_2z_3)^{q_2}(-iz_1z_3)^{q_1}\Gamma(-q_1)\Gamma(-q_2)\Gamma(-|q_3|)
&\CF^+(|q_3|;\mu)\cr
%
%\Biggl[{\Gamma(\half-i\mu+|q_3|)\over \Gamma(\half-i\mu)}
%+{\Gamma(\half-i\mu)\over \Gamma(\half-i\mu-|q_3|)}\Biggr]\cr
-2(-i z_1z_3)^{q_1}(-iz_2z_3)^{q_2}\Gamma(-q_1)\Gamma(-q_2)\Gamma(-|q_3|)
&\CF^+(q_1,-q_2;\mu)\cr}
%
%\Biggl[{\Gamma(\half-i\mu+q_1)\over \Gamma(\half-i\mu-q_2)}
%+{\Gamma(\half-i\mu+q_2)\over \Gamma(\half-i\mu-q_1)}\Biggr]\cr}
}
In isolating nonanalytic powers of $z$ we have to be careful that we can 
expand the hypergeometric functions as power series. Thus their 
arguments must have absolute value smaller than 1. It follows that 
we must apply the inversion formula for $F$ to one of the first 
two terms in \cntriv\ . Taking this into account we finally find
\eqn\threptii{\eqalign{
{\p\over\p\mu}\langle\sigma_0(\CO_{q_1})\sigma_0(\CO_{q_2})
\sigma_0(\CO_{q_3})\rangle&=\cr
\Gamma(-q_1)\Gamma(-q_2)\Gamma(-|q_3|)
Re\biggl[e^{i\pi |q_3|/2}&
\biggl(\CF^+(|q_3|;\mu)-\CF^+(q_2,-q_1;\mu)\biggr)\biggr]\cr}
}
As a check, one may take the limit $q_1\to 0$ in which case we
obtain the derivative with respect to $\mu$ of the two-point
function found above. Moreover, we can compute the topological 
expansion as before. In particular, the first two terms in
the topological expansion are
\eqn\genzrthr{\eqalign{
\langle\sigma_0(\CO_{q_1})\sigma_0(\CO_{q_2})
\sigma_0(\CO_{q_3})\rangle_{h=0}&=
\prod_i q_i\Gamma(-|q_i|)\mu^{|q_3|-1}\qquad\qquad\qquad\qquad
\cr
\biggl[1-&{(|q_3|-1)(|q_3|-2)(q_1^2+q_2^2-|q_3|-1)\over 24}\mu^{-2}
+\cdots\biggr] \cr}   
}
Accounting for wavefunction renormalization the leading term 
agrees with
the recent calculations of Kutasov and Di Francesco using 
the liouville theory \kutasov\ .

\subsec{Four-point function}

Next we turn to the four-point function. This allows us
to consider issues of factorization and the nature of 
intermediate states, a subject fraught with ticklish
issues of principle \natiliouv\ . 

The scattering amplitude depends on the kinematic configuration we 
are studying. There are, up to inversion and permutation, only
 two kinds of configurations:
either two or three momenta have the same sign.
We first consider the kinematic configuration:
\eqn\kincnfi{
q_1>0\qquad q_2>0\qquad  q_3>0,\qquad q_4<0
}
i.e., $0<q_1,q_2,q_3<-q_4$. 
Doing the $s,\xi$ integrals and keeping the 
contribution proportional to 
$\prod z_i^{|q_i|}$ as $z_i\to 0$
we find
\eqn\frscnfg{\eqalign{
2Im\Biggl[\prod_i \bigl(z_i^{|q_i|}\Gamma(-|q_i|)\bigr)e^{-i\pi|q_4|/2}
\biggl(&
\CF^-(q_1+q_2,-q_3,\mu)+\CF^-(q_3+q_1,-q_2,\mu)+\cr
&\CF^-(q_2+q_3,-q_1,\mu)-\CF^-(|q_4|;\mu)\biggr)\Biggr]\cr}
}
Expanding the polygamma functions we obtain the genus zero result
\eqn\frptzeriii{
\langle\sigma_0(\CO_{q_1})\sigma_0(\CO_{q_2})
\sigma_0(\CO_{q_3})\sigma_0(\CO_{q_4})\rangle_{h=0}
=\prod_i \bigl(|q_i|\Gamma(-|q_i|)\bigr)
\biggl((|q_4|-1)\mu^{|q_4|-2}\biggr)
}
reproducing the result of Kutasov and Di Francesco \kutasov\ . 

A rather different result emerges if
we choose the kinematic configuration:
\eqn\kinconf{\eqalign{
q_1>0\qquad q_2>0\qquad & q_3<0\qquad q_4<0\cr
q_1+q_3&=-(q_2+q_4)<0\cr
q_1+q_4&=-(q_2+q_3)<0\cr}
}
i.e., $0<q_1<-q_3,-q_4<q_2$. 
The calculation of the $s$-integrals proceeds by dividing up the 24 permutations
according to the configuration of signs $\epsilon_i$ they 
produce. Furthermore the result of any permutation is easily
related to its inverse. Thus one need only study the integrals 
corresponding to the permutations yielding $+++$, $+-+$, and $++-$. 
As before the small $z_i$ terms may be expressed in terms of 
hypergeometric functions. 
% 
%Of the 24 terms in \macptii\ only four can give a term of the form 
%$\prod_i z_i^{|q_i|}$ in the small $z_i$ expansion. These four 
%terms may be written as:
%\eqn\fourpti{\eqalign{
%\delta(\sum q_i)Im\int_0^\infty d\xi{e^{i\mu\xi}\over sh~(\xi/2)}
%e^{{i\over 2}cth(\xi/2)\sum z_i^2}
%&
%\int_0^\infty ds_1\int_0^\infty ds_2\int_0^\infty ds_3
%e^{-q_1 s_1+(q_1+q_3)s_2+q_4 s_3}\cr
%\biggl[e^{H(s_i,z_i)}
%&-[s_i\to - s_i]\biggr]\cr
%&+[q_3\leftrightarrow q_4,z_3\leftrightarrow z_4]\cr}
%}
%where
%\eqn\ehdef{\eqalign{
%H(s_i,z_i)\equiv &
% exp~ i\biggl(f(s_1)z_1 z_3+f(s_1-s_2)z_1z_2+f(s_2)z_2z_3\cr
%&+f(s_1-s_2+s_3)z_1 z_4+f(s_3-s_2)z_3z_4+f(s_3)z_2z_4\biggr)\cr}
%}
%
%Changing variables to $y_i=e^{s_i}$, and dropping terms from the 
%exponential leading to higher order terms in $z_i$ we find, after 
%some algebra, MORE DETAILS 
After some algebra we reduce \macptii\ to 
\eqn\fourptii{\eqalign{
\delta\bigl(\sum q_i\bigr){\prod_i\Gamma(-|q_i|)z_i^{|q_i|}
\over 2^{S}\Gamma(-S)}\qquad\qquad & \cr
Im\int_0^\infty d\xi {e^{i\mu\xi}\over \bigl(sh~(\xi/2)\bigr)^{1+S}}
\Biggl\{\biggl[e^{-\xi S/2}&F\bigl(q_3,-q_2;-S;1-e^\xi\bigr)\cr
-e^{\xi S/2}&F\bigl(q_3,-q_2;-S;1-e^{-\xi}\bigr)\biggr]
+\biggl[3\leftrightarrow 4\biggr]\cr
+&\biggl[e^{-\xi(q_2-q_1)/2}-e^{\xi(q_2-q_1)/2}\biggr]
-\biggl[e^{-\xi S/2}-e^{\xi S/2}\biggr]\Biggr\}\cr}
}
where $S\equiv q_1+q_2$. 


We can rewrite \fourptii\ in a way which makes its connection 
to the liouville theory more evident using a mellin-barnes 
representation of the hypergeometric function.
\foot{A similar representation for liouville amplitudes has been proposed
in\ref\sasha{A. Zamolodchikov, unpublished.}\natiliouv\wdwppr\ .}
Let $\CC$ be a curve running from $-i\infty$
 to 
$+i\infty$ as in 
\fig\mellbrnes{The curve $\CC$ for the case that $|q_i|<1$.}\ 
such that the poles of $\Gamma(-t)$, i.e., $t=0,1,2,\dots$ lie to 
the right of $\CC$ and the poles of $\Gamma(t-q_2)\Gamma(t+q_3)$ and 
$\Gamma(t-q_2)\Gamma(t+q_4)$, i.e., $t=q_2-n$, $t=-q_3-n$, $n=0,1,2,\dots$
 lie to the left of $\CC$.
Then we may write the four-point function:
\eqn\frmellin{\eqalign{
2\delta\bigl(\sum q_i\bigr) Im\Biggl[e^{-i\pi S/2}\prod_i\biggl(\Gamma(-|q_i|)z_i^{|q_i|}\biggr)
\Biggl[\CF^-(q_2,-q_1;\mu)&-\CF^-(S;\mu)+\cr\cr
\int_\CC {dt\over 2\pi i} {\Gamma(-q_2+t)\over \Gamma(-q_2)}
\biggl({\Gamma(q_3+t)\over \Gamma(q_3)}+{\Gamma(q_4+t)\over \Gamma(q_4)}\biggr)
\Gamma(-t)&\cr
\biggl(
{\Gamma(+S-t+\half-i\mu)\over \Gamma(\half-i\mu)}&-
{\Gamma(q_2-t+\half-i\mu)\over\Gamma(-q_1+\half-i\mu)}\biggr)
\Biggr] \cr }
}
We may obtain the topological expansion by deforming the contour 
so that it wraps around the positive real axis. As we do so we 
pick up a pole at $t=\half+S-i\mu, t=\half+q_2-i\mu$ which gives
an exponentially small correction to the four-point function.
 We can then evaluate
the remaining integral as an infinite sum of residues. Thus, the
perturbative expansion is obtained from 
\eqn\frmellni{\eqalign{
2\delta\bigl(\sum q_i\bigr) Im\Biggl[e^{i\pi S/2}\prod_i\biggl(\Gamma(-|q_i|)\biggr)
\Biggl[\CF^+(S;\mu)&-\CF^+(q_2,-q_1;\mu)+\cr\cr
\sum_{n=1}^\infty {(-1)^n\over n!}{\Gamma(-q_2+n)\over \Gamma(-q_2)}
\biggl({\Gamma(q_3+n)\over \Gamma(q_3)}+{\Gamma(q_4+n)\over \Gamma(q_4)}\biggr)
&\cr
\biggl(
{\Gamma(+S-n+\half-i\mu)\over \Gamma(\half-i\mu)}-&
{\Gamma(q_2-n+\half-i\mu)\over\Gamma(-q_1+\half-i\mu)}\biggr)
\Biggr] \cr }
}

In particular, 
expanding the above results to obtain the genus zero answer we find
\eqn\frptzr{
\langle\sigma_0(\CO_{q_1})\sigma_0(\CO_{q_2})
\sigma_0(\CO_{q_3})\sigma_0(\CO_{q_4})\rangle_{h=0}
=\prod_i \bigl(|q_i|\Gamma(-|q_i|)\bigr)
\biggl((q_2-1)\mu^{S-2}\biggr)
}
Given the result in these two 
kinematic configurations we can easily obtain the genus zero result in all configurations:
\foot{After this work was completed we learned that J. Polchinski,
and G. Mandal, A. Sengupta, and S. Wadia
have derived the same formula from another point of view
\wadsem
\ref\jotlk{J. Polchinski, Talk at Rutgers University.}.}
\eqn\frptzri{ \langle\sigma_0(\CO_{q_1})\sigma_0(\CO_{q_2})
\sigma_0(\CO_{q_3})\sigma_0(\CO_{q_4})\rangle_{h=0}
=\prod_i \bigl(|q_i|\Gamma(-|q_i|)\bigr)
\biggl(({\rm max}\{|q_i|\}-1)\mu^{\half(\sum |q_i|)-2}\biggr)
}
Physically, the singularities found in going from one kinematic 
configuration to another arise from the presence of on-shell
zero-momentum states in the intermediate channels.
\foot{Another remark of N. Seiberg 's.}
This is seen very clearly from \frmellin\ ; continuing from one 
configuration to another a pole of the integrand is forced to 
cross the contour $\CC$.

We expect that the above formulae can be cast into a much more 
beautiful form which illustrates the nature of duality and 
factorization in nonperturbative string theory
\foot{We hope to return to this point in the future.} .
Already from the above we can 
interpret the integral over $\CC$ as an integral over the 
spectrum of the liouville theory. The fact that $\CC$ runs 
along the imaginary axis has a physical interpretation. 
In the language of \natiliouv\ the external states are the 
``hartle-hawking'' or microscopic states.
In accordance with 
general reasoning, we see that the intermediate states which 
propagate are the ``curtright-thorn,'' or 
macroscopic states \natiliouv\ .

\newsec{Conclusion} 

In conclusion, let us address briefly four lessons we may 
learn from the above rather technical calculations. 
We have seen that macroscopic loop amplitudes are
naturally written in terms of hypergeometric functions.
This might have some bearing on the phenomenon of integrability 
of nonperturbative string theory. A general 
feature of matrix models solved thus far is that correlation 
functions turn out to be related to correlators of other
integrable quantum field theories at genus zero
\ref\alsms{The idea that this might be true in general 
was first proposed to me by A. Morozov and S. Shatashvili, Oct. 1989}\ .
For example, painlev\'e functions appear in the study 
of correlation functions of the massive ising model 
and in the nonlinear schr\"odinger theory
\ref\its{A.R. Its, A.G. Izergin, V.E. Korepin, and N.A. Slavnov, 
``Differential equations for quantum correlation functions,'' preprint, 
and references therein.}. 
This is probably related to the fact that the matrix models can be 
reformulated in terms of a field theory on the spectral 
plane (or line) as emphasized in 
\bdss\geom
\ref\alexy{A. Gerasimov, A. Marshakov, A. Mironov, and A. Orlov, 
``Matrix Models of 2D Gravity and Toda Theory,''
Lebedev Inst. Preprints, July, August, 1990; A. Mironov and A. Morozov,
Lebedev Inst. Preprint July 1990}
\ref\tomcar{T. Banks, ``Matrix Models, String Field Theory and Topology,''
RU-90-52, to appear in the proceedings of the Carg\'ese workshop on 
random surfaces.}
\ref\emil{E. Martinec,``On the origin of integrability in matrix
models,'' Chicago preprint EFI-90-67}. 
One might therefore ask if 
there is a reformulation of the free fermi theory
at $c=1$ which makes clearer its relation to 
standard integrable field theories.

Given the observed phenomenon of integrability of nonperturbative
string physics  one must 
ask how generic we expect it to be. If it is indeed merely
a manifestation of the equivalence to a theory of free fermions
then we would guess that the phenomenon is far from generic.
On the other hand, the idea that such a phenomenon might
take place was proposed some time ago by Knizhnik
\ref\knizh{V.G. Knizhnik, Sov. Phys. Usp. {\bf 32}(1989)945}\ 
in a slightly different context.
The essential point of \knizh\ is that 
we could try to sum all orders of perturbation 
theory in {\it critical} string theory by adding a kind of ``handle-gluing 
operator'' to a conformal field theory action. One imagines
that inclusion of this operator converts the theory to some
kind of integrable theory
\foot{We thank A. Morozov for explanations of this idea.}.
 If this is what lies behind the 
phenomenon of nonperturbative integrability then we might guess that it is 
in fact generic. Even if the phenomenon is not generic it is 
worth understanding thoroughly, for such an understanding
might yield a wealth of new examples.

A second lesson is that the equivalent string field 
theory is likely to be more complicated than the lagrangian 
written in \dsjv\ . In particular, the short distance behavior
is softer than that encountered in relativistic field theory.

A third lesson from the above calculations has been 
discussed throughout the paper, namely, that for high 
energies, the string perturbation series ceases to be 
even a good asymptotic expansion. Surfaces do not provide
a good description of the physics at high energies.
Instead, one must use
other, ``parton,'' degrees of freedom. In our case, the 
parton is the double-scaled free fermion.
Does this provide a good model for what happens in the effective
string theory of QCD?

A fourth lesson we may draw from our calculations is that topology- 
changing effects in gravity can induce rather mild modifications
of the wheeler-de-witt equation. If that is true in four dimensional
gravity it will probably be important.
\bigskip
{\it Note added (2/26/91):} We would like to bring the 
reader's attention to two other recent 
papers on  $c=1$ matrix model correlation functions
\ref\wadii{G. Mandal, A. Sengupta, and S. Wadia, ``Interactions 
and scattering in d=1 string theory,'' IAS preprint, IASSNS-HEP/91/8}
\ref\grkliii{D. Gross and I. Klebanov, ``S=1 For c=1,'' 
Princeton preprint, PUPT-1241}\ .
\bigskip
\bigskip
\centerline{\bf Acknowledgements}

As is evident from the many footnotes above, this work was 
heavily influenced by many remarks and insights of T. Banks,
N. Seiberg, and S. Shenker. We thank them for their generous
help and interest, and for comments on the manuscript.
We also thank A. Dabholkar, D. Kutasov,
E. Martinec, K. Rabe, R. Shankar, M. Staudacher, and A. Zamolodchikov 
for useful conversations.
We thank the Rutgers Department of Physics for hospitality.
This work was supported by DOE grants DE-AC02-76ER03075
and DE-FG05-90ER40559 and by a Presidential Young Investigator
Award.

\appendix{A}{Parabolic Cylinder Functions}

\subsec{Definitions}
 
Unfortunately, there are four notations commonly used for 
parabolic cylinder functions 
\ref\gradsh{I.S. Gradshteyn and I.M. Ryzhik, {\it Tables of Integrals,
Series, and Products}, Academic Press, 1980}
\ref\abram{M. Abramowitz and I. Stegun, {\it Handbook of Mathematical
Functions} Dover, 1968}\ .
Our wavefunctions $\psi^\pm(a,x)$ are the delta-function 
normalized even and odd solutions of 
$({d^2\over dx^2}+{x^2\over 4})\psi=a\psi$. 
In terms of degenerate hypergeometric $_1 F_1(\alpha,\beta;x)$ and
whittaker functions $M_{\mu,\nu}(x), D_a(x)$ we have even and odd parity
wavefunctions:
\eqn\wvfnspl{\eqalign{
\psi^+(a,x)&={1\over \sqrt{4 \pi (1+e^{2\pi a})^{1/2}}}(W(a,x)+W(a,-x))\cr
&={1\over \sqrt{4 \pi (1+e^{2\pi a})^{1/2}}}2^{1/4}\biggl|{\Gamma(1/4+i a/2)
\over \Gamma(3/4+i a/2)}\biggr|^{1/2}e^{-i x^2/4} 
{}_1F_1(1/4-ia/2;1/2;i x^2/2)\cr
&={e^{-i \pi/8}\over 2\pi} e^{-a \pi/4}|\Gamma(1/4+i a/2)|{1\over \sqrt{|x|}}
M_{ia/2,-1/4}(i x^2/2)\cr}
}
\eqn\wvfnsmi{\eqalign{
\psi^-(a,x)&={1\over \sqrt{4 \pi (1+e^{2\pi a})^{1/2}}}(W(a,x)-W(a,-x))\cr
&={1\over \sqrt{4 \pi (1+e^{2\pi a})^{1/2}}}2^{3/4}\biggl|{\Gamma(3/4+i a/2)
\over \Gamma(1/4+i a/2)}\biggr|^{1/2} x e^{-i x^2/4} 
{}_1F_1(3/4-ia/2;3/2;i x^2/2)\cr
&={e^{-3 i \pi/8}\over \pi} e^{-a \pi/4}|\Gamma(3/4+i a/2)|
{x\over |x|^{3/2}}
M_{ia/2,1/4}(i x^2/2)\cr}
}
 
It is useful to have a good picture of what these states look like.
In 
\fig\wvfns{The even parity wavefunction for $a=2,0,-2$.}\ 
we have plotted three representative wavefunctions.
Note that, as opposed to the correct sign harmonic oscillator, 
{\it both} solutions of the differential equation decay at infinity, 
and there is a continuous spectrum.

\subsec{Asymptotics}

Define
\eqn\asydefs{\eqalign{
\Phi(\mu)&\equiv{\pi\over 4}+\half arg\Gamma(\half+i\mu)\cr
k(\mu)&=\sqrt{1+e^{2\pi\mu}}-e^{\pi\mu}=\CO(e^{-\pi\mu})\cr
k(\mu)^{-1}&=\sqrt{1+e^{2\pi\mu}}+e^{\pi\mu}=2 e^{\pi\mu}+\CO(e^{-\pi\mu})
\cr}
}
The asymptotic properties of the wavefunctions
\abram\ 
are:
 
\noindent
1. $\mu\gg\lambda^2$.
\eqn\pcfi{\eqalign{
\psi^+(\mu,\lambda)&\sim {e^{-\pi\mu/2}\over (2\pi)^{1/2}\mu^{1/4}}
ch\bigl(\sqrt{\mu}\lambda\bigr)\cr
\psi^-(\mu,\lambda)&\sim {e^{-\pi\mu/2}\over (2\pi)^{1/2}\mu^{1/4}}
sh\bigl(\sqrt{\mu}\lambda\bigr)\cr}
}

\noindent
2. $-\mu\gg\lambda^2$.
\eqn\pcfi{\eqalign{
\psi^+(\mu,\lambda)&\sim {1\over (4\pi)^{1/2}|\mu|^{1/4}}
cos(\sqrt{-\mu}\lambda)\cr
\psi^-(\mu,\lambda)&\sim {1\over (4\pi)^{1/2}|\mu|^{1/4}}
sin(\sqrt{-\mu}\lambda)\cr}
}

\noindent
3. $\lambda\gg |\mu|$.
\eqn\asyi{\eqalign{
\psi^{\pm}(\mu,\lambda)\sim & {1\over 
(2\pi\lambda\sqrt{1+e^{2\pi\mu}})^{1/2}}
\biggl[\sqrt{k(\mu)}cos\bigl(\lambda^2/4-\mu log\lambda+\Phi(\mu)
\bigr)\cr
\pm &
1/\sqrt{k(\mu)}sin\bigl(\lambda^2/4-\mu log\lambda+\Phi(\mu)\bigr)
\biggr]\cr}
}
 
\noindent
4. $X\equiv\sqrt{\lambda^2-4\mu}\gg 1$.
\eqn\asyii{\eqalign{
\psi^{\pm}(\mu,\lambda)\sim &{1\over 
(2\pi X \sqrt{1+e^{2\pi\mu}})^{1/2}}
\biggl[\sqrt{k(\mu)}cos\bigl({1\over 4}\lambda X-\mu \tau(\lambda,\mu)
+{\pi\over 4}
\bigr)\cr
\pm &
1/\sqrt{k(\mu)}sin\bigl({1\over 4}\lambda X-\mu \tau(\lambda,\mu)
+{\pi\over 4}
\bigr)\biggr]\cr}
}

This last estimate suggests that we define ``plane-wave'' combinations
\eqn\planewvii{\eqalign{
\chi^+&\equiv\half(k^{-1/2}+ik^{1/2})\psi^++\half(k^{-1/2}-ik^{1/2})\psi^-\cr
\chi^-&\equiv\half(k^{-1/2}-ik^{1/2})\psi^++\half(k^{-1/2}+ik^{1/2})\psi^-\cr}
}
which have the property that for the asymptotic configuration 4
we may write:
\eqn\planewv{
\chi^{\pm}(\nu,\lambda)\sim 
\chi^{\pm}(\mu,\lambda)e^{\mp i(\nu-\mu)
\bigl(\tau(\lambda,\mu)+\CO(\nu/\lambda^2)\bigr)}
}

\subsec{Integrals}

The wavefunctions \wvfnspl\wvfnsmi\ 
are delta-function normalized states:
\eqn\delone{
\int d~\lambda [\psi^+(a_1,\lambda) \psi^+(a_2,\lambda)+
\psi^-(a_1,\lambda) \psi^-(a_2,\lambda)]=\delta(a_1-a_2)}
and
\eqn\deltwo{
\int d~a [\psi^+(a,\lambda_1) \psi^+(a,\lambda_2)+
\psi^-(a,\lambda_1) \psi^-(a,\lambda_2)]=\delta(\lambda_1-\lambda_2)}
 
In fact, we will need a stronger result, namely,
\eqn\lapi{\eqalign{
\int d\nu e^{i \nu s}\sum_{\epsilon}\psi^\epsilon(\nu,\lambda_1)
\psi^{\epsilon}(\nu,\lambda_2)&=
{1\over\sqrt{4\pi i~sh~s}}
exp{i\over 4}\biggl[{\lambda_1^2+\lambda_2^2\over
th~s}-{2\lambda_1\lambda_2\over sh~s}\biggr]\cr
&=\langle\lambda_1|e^{-2 i sH}|\lambda_2\rangle\cr}
}
which is valid for $-\pi<Im~s<0$ and for $Im~s=0,Re~s\not=0$.
In the second line we have $H=\half p^2-{1\over 4}\lambda^2$.
One may think that this identity is an obvious consequence of 
the analytic continuation of a similar formula for the 
right-side-up oscillator. However, since the nature of the 
spectrum and eigenfunctions is radically different we give 
here a careful proof of \lapi\ . Substituting the 
third line of \wvfnspl\wvfnsmi\ we see that \lapi\ is a 
consequence
\foot{In fact, the identity in \gradsh\ is incorrect. This 
is a second motivation to give a careful proof.}
of GR 7.694
\eqn\grdshdient{\eqalign{
\int_{-\infty}^\infty e^{-2\rho x i}\Gamma(\half+\nu+ix)
\Gamma(\half+\nu-ix)M_{ix,\nu}(\alpha)&M_{ix,\nu}(\beta) dx=\cr
\pi & \bigl(\Gamma(2\nu+1)\bigr)^2{\sqrt{\alpha\beta}\over ch~\rho}
e^{-\half th\rho(\alpha+\beta)}J_{2\nu}
\bigl({\sqrt{\alpha\beta}\over ch\rho}\bigr)\cr}
}

{\it Proof}: Use GR 6.643.1,2. to express the integrand 
in terms of bessel and 
modified bessel functions. Exchange orders of integration and do the $x$ 
integral to get a delta function. 
Evaluating the delta function we are left with 
an integral with a product of two bessel functions. Now use
GR 6.633.4 to obtain \grdshdient\ .
 
One useful immediate corrolary of \lapi\ 
is the resolvent of the upside-down oscillator:
\eqn\rslvnt{\eqalign{
R(\zeta;\lambda_1,\lambda_2)&\equiv\int d\nu {1\over \nu-\zeta}
\sum_{\epsilon}\psi^\epsilon(\nu,\lambda_1)
\psi^{\epsilon}(\nu,\lambda_2)\cr
&=-i\int_0^{-\epsilon \infty}ds e^{-i s\zeta}
{1\over\sqrt{4\pi i~sh~s}}
exp{i\over 4}\biggl[{\lambda_1^2+\lambda_2^2\over
th~s}-{2\lambda_1\lambda_2\over sh~s}\biggr]\cr}
}
where $\epsilon=sgn[Im\zeta]$. This gives a rigorous justification
for the analytic continuations used extensively in \grkli\ .

\appendix{B}{Expansions of $\CF^\pm$}

Here we record some properties of the functions $\CF^\pm$ 
occuring in the macroscopic loop amplitudes. 
In particular, $\CF^\pm(a,b;\mu)$ 
may be expressed in terms of polygamma functions defined by
\eqn\plygmma{\eqalign{
\Psi(\half-i\mu)&=-i{\pi\over 2}+log~\mu+\sum_{n=1}^\infty
{(-1)^n B_{2n}\over 2n}(1-2^{-2n+1})\mu^{-2n}\cr
\Psi^{(n)}(z)&\equiv {d\over dz}\Psi^{(n-1)}(z)\cr}
}
First we taylor expand:
\eqn\plygmii{\eqalign{\CF^\pm(a,b;\mu)&\equiv
{\Gamma(a+\half-i\mu)\over\Gamma(b+\half-i\mu)}
\pm
{\Gamma(-b+\half-i\mu)\over\Gamma(-a+\half-i\mu)}\cr
&=
exp\biggl(\sum_{n=1,odd}^\infty{a^n-b^n\over n!}\Psi^{(n-1)}(\half-i\mu)\biggr)\cr
\Biggl[exp\biggl(\sum_{n=2,even}^\infty{a^n-b^n\over n!}&\Psi^{(n-1)}(\half-i\mu)\biggr)
\pm exp\biggl(-\sum_{n=2,even}^\infty{a^n-b^n\over n!}\Psi^{(n-1)}(\half-i\mu)\biggr)\Biggr]\cr}
}
Then we use the asymptotics in \plygmma\ . The result may be expressed as
\eqn\expeff{\eqalign{
\CF^+(a,b;\mu)&\equiv
e^{-i\pi(a-b)/2}\mu^{a-b}\sum_{h=0}^\infty{Q_h^+(a,b)\over\mu^{2h}}\cr
\CF^-(a,b;\mu)&\equiv
i e^{-i\pi(a-b)/2}\mu^{a-b}\sum_{h=0}^\infty{Q_h^-(a,b)\over\mu^{2h+1}}\cr}
}
where $Q_h^\pm$ are polynomials in $a,b$ with real coefficients.
The first few are
\eqn\qmiply{\eqalign{
Q^+_0(a,b)&=2\cr
Q^+_1(a,b)&=-d(d-1){(3s^2-d-1)\over 12}\cr
Q^+_2(a,b)&=d(d-1)(d-2)(d-3){(15s^4-30ds^2-30s^2+5d^2+12d+7)\over 2880}\cr
Q^-_0(a,b)&=sd\cr
Q^-_1(a,b)&=-sd(d-1)(d-2){(s^2-d-1)\over 24}\cr
Q^-_2(a,b)&=sd(d-1)(d-2)(d-3)(d-4)
{(3s^4-10ds^2-10s^2+5d^2+12d+7)\over 5760}\cr}
}
where $s=a+b,d=a-b$. 

These polynomials are also useful for discussing high-energy
behavior. Indeed, since amplitudes are functions of 
$q+i\mu$ there is an interesting duality between the 
high and low energy regimes. For example, as $a\to +\infty$
we find
\eqn\hgha{
{\Gamma(a+\half-i\mu)\over\Gamma(\half-i\mu)}\sim
\half {\Gamma(a+\half)\over\Gamma(\half-i\mu)}a^{-i\mu}
\biggl[\sum_{h=0}^\infty{Q_h^+(-i\mu,0)\over a^{2h}}
+i \sum_{h=0}^\infty{Q_h^-(-i\mu,0)\over a^{2h+1}}\biggr]
}

\appendix{C}{Gaussian integrals}

The gaussian integral needed for the macroscopic
loop amplitude is most conveniently done using 
harmonic oscillators. Note that the evaluation of 
the quantity
\eqn\trci{
tr\biggl(e^{\xi_n \hat x}e^{-t_1 H} e^{\xi_1\hat x}e^{-t_2 H}
\cdots e^{\xi_{n-1}\hat x}e^{-t_n H}\biggr)
}
in position space, where $H=\half p^2+\half\omega^2 x^2$,
involves the same gaussian integral as the one we need, after analytic 
continuation of $t_i,\omega$ from real to pure
imaginary values. On the other hand, this trace is
easily evaluated using coherent state formalism
with the result that \trci\ is
\eqn\trcii{{1\over 2 sh(\omega T/2)}
exp\biggl[{cth(\omega T/2)\over 4\omega}\sum_1^n\xi_i^2\biggr]
exp\biggl[\sum_{i>j}
{ch\bigl(\omega(t_i+\cdots t_{j+1}-T/2)\bigr)\over 2\omega sh\omega T/2}
\xi_i\xi_j\biggr]
}
One might object to harmonic oscillators with these 
``unphysical'' values of parameters.
Since gaussian integrals are given in terms of the 
inverse and determinant of a certain quadratic form, 
we may regard the introduction of harmonic
oscillators as a trick to prove a certain {\it algebraic}
identity for the quadratic form arising in \macropt .
 
 
\listrefs
\listfigs
\bye





Moreover, 
we can express the two-point functions:
\eqn\twdsc{
\langle \CO^{(k)}_q\CO^{(k)}_{-q}\rangle
=
2Im\Biggl[e^{-i\pi|q|/2}\CF^-(k+|q|,-k;\mu)
\Biggr]
}
A similar discussion can be carried out for the degenerate
fields $\CD^{(n)}_r$. In this case the formulae are more complicated
since the second index of the whittaker function is an integer. 
We simply quote the result for the correlator
\eqn\twdg{
\eqalign{
\langle \CD^{(0)}_r\CD^{(0)}_{-r}\rangle=\qquad\qquad\qquad\qquad&
\cr
2Im\Biggl[{4(-i)^rr\over r^2-q^2}({(z_1z_2)^r\over r!^2}
\biggl({\Gamma(r+|q|+\half-i\mu)\over\Gamma(\half-i\mu)}&
\Psi(r+\half-i\mu)
-
{\Gamma(\half-i\mu)\over\Gamma(-r+\half-i\mu)}
\Psi(\half-i\mu)\biggr)\Biggr]\cr}
}
in fact the above expression is also analytic in $\mu$ for $q$ not
an integer. This is expected since, for momentum $q\not= r$ 
the operator cannot couple to the macroscopic loop of 
momentum $q$. Nevertheless the pole at $q=r$ should be interpreted 
as an extra logarithmic factor in $\mu$.
\foot{These last few remarks were pointed out to me by N. Seiberg.}

DISCUSS SPECTRUM
From the expansion of the bessel 
function we see that we will obtain terms of the form $z_1^{|q|+n}
z_2^{|q|+n}$, where $n\in \IZ_+$, from the bessel function $J_{|q|}$
and terms of the form $z_1^{r+n}z_2^{r+n}$ from th bessel 
functions $J_r$. 
Thus, if working at fixed $\xi$ is indeed working at 
fixed area, we can already conclude that the scaling operators 
of the 
theory, $\CO^{(n)}_{q}$, are labelled by
a smoothly varying part, $q$ together with an nonnegative
integer $n$ labelling the tower of ``gravitational descendents.'' 
In addition there is a discrete spectrum $\CD_m^{(n)}$ 
arising from the degenerate representations of 
the virasoro algebra at $c=1$, together with their gravitational
descendents. NO D's??

We use the double scaled free fermion field theory of 
the c=1 matrix model to compute the correlation functions
of eigenvalue densities and macroscopic loops. Integral 
representations for these quantities are found. We compare
the eigenvalue density correlators with the predictions of 
free relativistic field theory on the spectral curve and 
find differences, whose physical origin is explained. 
We obtain explicit expressions for the correlation function
of two macroscopic loops to all orders of perturbation 
theory. From this one may deduce the spectrum of the model. 
We find explicit expressions for the 3 and 4 point scattering
amplitudes of tachyon operators to all orders of perturbation theory.
It is found that for momenta of the order of the cosmological
constant the topological perturbation series ceases to be asymptotic,
explicitly realizing the transition in a string theory from 
a theory of strings to a theory of partons.




 
 
 
 


